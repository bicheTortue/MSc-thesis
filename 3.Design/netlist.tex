\section{Netlist generation}
\label{sec:netlist}

In order to be able to run the simulation with different kinds of \ac{NN} architecture. The point of the part is thus to explain how the netlist generator tool works.

This is done by generating a SPICE netlist using a python script. This script takes in a few parameters :

\begin{itemize}
  \item The first parameter is the number of input we use our system. This is the size of the input vector ($x_t$) from \cref{fig:lstmCell}.
  \item The next parameter is the number of time steps. Everything about the \ac{LSTM} time steps is all explained in \cref{sec:lstm}.
  \item The number of hidden state of the \ac{LSTM} network.
  \item The serial size of the crossbar arrays, as described in \cref{sec:xbarCircuit}, used in the \ac{LSTM} network.
  \item The files containing the weights for each part of the circuit. The weights in the files have to be organized using a specific model (\cref{subsec:weiStore}).
  \item Finally the name of the file in which the output of the script (the netlist) will be written to.
\end{itemize}

\subsection{Weights storage}\label{subsec:weiStore}

The weights are stored in a single file that is the save a numpy array. The first dimension contains the weights for the different layers. Depending on the layer used, the weights are stored a bit differently :

\begin{itemize}
  \item Dense layer : This is basically a \ac{VMM} so the weights are just sorted linearly in a list. The list is of size $n^2$, where $n$ is the size of the input vector. \Cref{mtrx:wei} shows the position ($i$) of the weight ($w_i$) in the matrix.
  \item \ac{LSTM} layer : An \ac{LSTM} layer contains four \acp{VMM} which can be assimilated and thus stored like a dense layer. The weights for each \ac{VMM} is then stored in a sub list.
\end{itemize}

\begin{equation}\label{mtrx:wei}
  \begin{bmatrix}
    w_{0} & w_{1} & \dots \\
    \vdots & w_i & \vdots \\
    \dots & w_{n^2-2} & w_{n^2-1}\\
  \end{bmatrix}
\end{equation}

All the code can be found on my github page \cite{lstmGen}.
