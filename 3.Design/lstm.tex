\section{\ac{LSTM} analog implementation}\label{sec:lstmCircuit}

\subsection{Circuit}

This section describes the circuit of an \ac{LSTM} with an input vector of size $n_i$, a $n_h$ hidden states, a serial size of $n_s$ and $n_{ts}$ time steps. $n_o=n_h/n_s$ is going to be used for future references in this section. In order for the crossbar array to be used $n_o$ must be an integer, in other words, $n_s$ must divide $n_h$. The circuit (shown in \cref{circt:lstm}) is pretty complex and contains numerous parts that require explaination.

\begin{figure}[H]
  \centering
  \includesvg[width=\textwidth,pretex=\tiny]{lstm/lstmCircuit}
  \caption{\ac{LSTM} circuit}
  \label{circt:lstm}
\end{figure}

The system is built using crossbar array (\cref{sec:xbarCircuit}) with ($n_i+n_h+1$,$n_o$, $n_s$) as parameters.

First of all, the different vectors and variables present in the schematic have to be described :

\begin{itemize}
  \item $\overrightarrow{x_t}$ : This is the input vector for the \ac{LSTM} circuit at time $t$. It has a size of $n_i$.
  \item $\overrightarrow{h_t}$ : This is the hidden layer vector for the feedback connections, it is defined as $\overrightarrow{h_t}=(\overrightarrow{h_{t,i}}) \forall i\in [\![0,n_o-1]\!]$, with $\overrightarrow{h_{t,i}}=(h_{t,i,j}) \forall j\in [\![0,n_s-1]\!]$.
  \item $\overrightarrow{z_t}$ is the input of the crossbar but not the input of the \ac{LSTM}. This vector is there to lighten the informations on the schematic (\cref{circt:lstm}). $\overrightarrow{z_t}$ is defined by $\overrightarrow{z_t}=(\overrightarrow{x_t},\overrightarrow{h_{t-1}},b)$.
  \item $e_{j,0}$ and $e_{j,1}$ are two enable flags that respectively represent the first and second half of $e_j$.
  \item $e_{in}$ and $e_{out}$ are the flags used to enable the hidden state values to go to the input (feedback connection) or to the output of the circuit.
  \item $e_{next}$ is the enable flag on in between two time steps.
  \item $R_{amp0}$ and $R_{amp1}$ are the two resistances used to amplify the output voltage of the voltage multipliers. The ratio of the resistances value must of $\frac{R_{amp1}}{R_{amp0}}=10$. The resistances must stay arround the values of the resistances used around the circuit, especially those of the memristors.
\end{itemize}

The wires coming into the crossbar are a bus of size $n_i+n_h+1$ and the output of the crossbar is a bus of size $n_o$ (\cref{sym:xbar}). This is why everything in the system apart from the crossbar arrays is only shown once in \cref{circt:lstm} but in reality those components are present $n_o$ times. Those extra components are needed in order for the parallel channels to work.

\subsection{Doubling memory cells}

\subsubsection{Feedback hidden states}

In order for the system to work in serial mode, the memory cells of the output need to be doubled. This is done in \cref{circt:lstm}, and allows for the hidden states to be saved for the next stage. Indeed, if using the system in serial mode with only one line of memory cells, the old hidden state ($\overrightarrow{h_{t-1}}$) is slowly overwritten by the current hidden state ($\overrightarrow{h_t}$). As soon as the first $n_o$ serial values are computed, they are going to override the old ones, still required for the following serial values. \Cref{tim:lstmMemcell} shows that the value in the memory cell changes too early in the cycle and gives a bad input for the next serial values. The value given to the input of the \ac{LSTM} should always be the old hidden state ($\overrightarrow{h_{t-1}}$).

\begin{figure}[H]
  \centering
  \begin{tikztimingtable}
    $e_{n_s-1}$ & 3H8Lx \\
    $e_0$ & 3L4H4Lx \\
    $e_1$ & 7L4Hx \\
    $m_{i,0}$ & 3D{$h_{t-1,i,0}$}8D{$h_{t,i,0}$}x \\
    $m_{i,1}$ & 7D{$h_{t-1,i,1}$}4D{$h_{t,i,1}$}x \\
    \extracode
    \tablerules
    %\draw (0,0) circle (2pt); % Origin
    \begin{pgfonlayer}{background}
      \vertlines[help lines]{3.05}
      %\vertlines[red]{1.6,5.6,15.6}
      %\vertlines[blue]{3.6,9.6,15.6}
    \end{pgfonlayer}
  \end{tikztimingtable}
  \caption{Time diagram of the values in the different memory cells ($m_{i,j}$ being the value stored in the $i^{th}$ parallel memory cells for the $j^{th}$ serial value). It is assumed that $n_s>1$ (the system is in serial mode). This graph does not take into account the pauses $e_{next}$ between the time steps.}
  \label{tim:lstmMemcell}
\end{figure}

Using two memory cells when in serial mode allows for this issue to be elevated. The values are transfered from one level to an other at the end of the time step. That way, the values are updated safely during each time step, without changing the final output.

When the system is running is a fully parallel mode, doubling the memory cells still causes a problem. The enable in and enable out flags can't be high at the same time when the memory cell has a feedback connection, the signal would interfere with it self and change its own value. Potential solutions are discussed in \cref{subsec:noDoubleMemcell}.

\subsubsection{Cell states}

Cell states have the same solution to a similar problem. Here the problem is that the memory cell has a feedback connection and needs to be activated within one serial step. The value of the old cell state needs to be used to compute the current values for the hidden states ($\overrightarrow{h_t}$) and cell states ($\overrightarrow{c_t}$). The issue is that at the value in the memory cell needs to be conserved until the next serial step.  (\cref{subsec:noDoubleMemcell}).

Once again, when the system is running in fully parallel mode, only one memory cell line is required for a normal behavior.

\subsection{Serialization/Parallelization}
\label{subsec:serpar}

Using an \ac{LSTM} in serial mode is very beneficial as it divides the number of point wise circuit by $n_s$.

This serial system has great advantages but comes with a few down sides. If the system we are working with has $n_h$ hidden states. This number of hidden states ($n_h$) is the major changing element in the final system(TODO ref final onchip area).

TODO : Rework
\begin{itemize}
  \item When the system is serialized, the inference time increases with a factor of $O(n_h)$.
  \item When the system is in parallel, the pointwise circuit's onChip area increases with a factor of $O(n_h^2)$.
    This is due to the fact that the onChip area of the pointwise part of the circuit increases with a factor of $O(n_h)$ and the onChip area of a crossbar array increase with a factor of $O(n_h^2)$. So the final increase of onChip area is $O(n_h^2+n_h)=O(n_h^2)$.
\end{itemize}

However, in both cases, the overall onChip area of the crossbar arrays increases with a factor of $O(n_h^2)$. That means that using the parallel system won't increase the onChip area by a lot as $n_h$ rises, because the $O(n_h)$ factor of the parallel version is absorbed as $O(n_h^2+n_h)=O(n_h^2)$.

That leads to the conclusion that the best thing to do is use an hybrid system (both serialized and parallel), using the limiting inference time for our system to set the maximum serial size. Depending on the system we're using this circuit for, we might have different time requirements.

\subsection{Symbol}

The symbol of the \ac{LSTM} circuit can be found in \cref{sym:lstm}. It depends on several parameters, the number of inputs ($n_i$), the number of hidden states ($n_h$), the serial size of the system ($n_s$) and the number of time steps ($n_{ts}$).

\begin{figure}[H]
  \centering
  \includesvg[height=2.5cm]{crossbar/xbarSymbol}
  \caption{Symbol used for the \ac{LSTM} circuit}
  \label{sym:lstm}
\end{figure}

The total onChip area for the crossbar circuit depends on the previously defined parameters.

\begin{equation}
  A_{xbar}(n_i,n_o,n_s)=
  \begin{cases}
    2\cdot n_i\cdot n_o \cdot A_{memristor}+n_o\cdot(2\cdot[A_{opAmp}+A_{R_r}]+A_{R_f}),\hfill n_s=1\\
    \begin{array}{2}
      2\cdot n_i\cdot n_o \cdot n_s\cdot A_{memristor}+2\cdot A_{CMOS}\cdot n_o\cdot n_s\\
      +n_o\cdot(2\cdot[A_{opAmp}+A_{R_r}]+A_{R_f})
    \end{array}
    ,\hfill \forall n_s>1\\
  \end{cases}
\end{equation}

With $A_{CMOS}$ being the same than for the memory cell (\cref{sec:memcell}), $A_{CMOS}=1.2\cdot 10^{-14}m^2$, and the feature size of typical memristor that can be fabricated at INESC is $3\mu m$, which would make the approximate area for a memristor $A_{memristor}=9\mu m^2=9\cdot 10^{-12} m^2$.%TODO check info and inesc citation
