\section{\ac{LSTM} analog implementation}\label{sec:lstmCircuit}

\subsection{Circuit}

This section describes the circuit of an \ac{LSTM} with an input vector of size $n_i$, a $n_h$ hidden states, a serial size of $n_s$ and $n_{ts}$ time steps. $n_o=n_h/n_s$ is going to be used for future references in this section. In order for the crossbar array to be used $n_o$ must be an integer, in other words, $n_s$ must divide $n_h$. The circuit (shown in \cref{circt:lstm}) is pretty complex and contains numerous parts that require explaination.

\begin{figure}[H]
  \centering
  \includesvg[width=\textwidth,pretex=\tiny]{lstm/lstmCircuit}
  \caption{\ac{LSTM} circuit}
  \label{circt:lstm}
\end{figure}

The system is built using crossbar array (\cref{sec:xbarCircuit}) with ($n_i+n_h+1$,$n_o$, $n_s$) as parameters.

First of all, the different vectors and variables present in the schematic have to be described :

\begin{itemize}
  \item $\overrightarrow{x_t}$ : This is the input vector for the \ac{LSTM} circuit at time $t$. It has a size of $n_i$.
  \item $\overrightarrow{h_t}$ : This is the hidden layer vector for the feedback connections, it is defined as $\overrightarrow{h_t}=(\overrightarrow{h_{t,i}}) \forall i\in [\![0,n_o-1]\!]$, with $\overrightarrow{h_{t,i}}=(\overrightarrow{h_{t,i,j}}) \forall j\in [\![0,n_s-1]\!]$.
  \item $\overrightarrow{z_t}$ is the input of the crossbar but not the input of the \ac{LSTM}. This vector is there to lighten the informations on the schematic (\cref{circt:lstm}). $\overrightarrow{z_t}$ is defined by $\overrightarrow{z_t}=(\overrightarrow{x_t},\overrightarrow{h_{t-1}},b)$.
  \item $e_{j,0}$ and $e_{j,1}$ are two enable flags that respectively represent the first and second half of $e_j$.
  \item $e_{in}$ and $e_{out}$ are the flags used to enable the hidden state values to go to the input (feedback connection) or to the output of the circuit.
  \item $e_{next}$ is the enable flag on in between two time steps.
\end{itemize}

The wires coming into the crossbar are a bus of size $n_i+n_h+1$ and the output of the crossbar is a bus of size $n_o$. This is why everything in the system apart from the crossbar arrays is only shown once in \cref{circt:lstm} but in reality, those components are present $n_o$ times. Those extra components are needed in order for the parallel channels to work.

\subsubsection{Memcells}

The memcells use and their number is going to be

\subsection{Inputs}%TODO oen section about datasets ?
