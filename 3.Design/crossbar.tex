\section{Voltage-driven crossbar circuit}
\label{sec:xbarCircuit}

The crossbar circuit theory has already been explained in the \cref{sec:crossbar}. This section describes how the crossbar circuit is actually implemented. The final circuit is the one in \cref{circt:xbar}. The circuit depends on three parameters :

\begin{itemize}
  \item $n_i$ : The number of input for our crossbar array (not including bias for a more general circuit).
  \item $n_o$ : The number of parallel output for our crossbar array.
  \item $n_s$ : The serial size of our crossbar system.
\end{itemize}

\begin{figure}[H]
  \centering
  \includesvg[width=\textwidth]{crossbar/crossbarUse}
  \caption{Circuit of the crossbar array used in the final system ($n_i$, $n_o$, $n_s$)}
  \label{circt:xbar}
\end{figure}

The parallel and serial are explained later (\cref{subsec:serpar}).

The enable flags ($e_j,\forall j\in[\![ 0,n_s]\!]$ and similarly $\overline{e_j}$) are there to show when the \ac{CMOS} switches are open, the states of those flags is shown in \cref{tim:serpar}.

\subsection{Two memristors per synapse}
\label{subsec:doubleMem}

First of all, a two memristors per synapse architecture has been used. This design is beneficial in two ways :

\begin{itemize}
  \item It doubles the range of weights (\cref{tab:synapses})
  \item It allows to easily use negative weights, as this design simply allows the output values to be centered around $V_{cm}$ to be compliant with the standard set in \cref{tab:valConv}.
\end{itemize}

This design is the one that is used in \cite{doubleMem}. Let's assume that a given memristor has a resistance range of $R\in[R_{min},R_{max}]$, that means it's conductance range is $\sigma \in [\sigma_{min},\sigma_{max}]$ (with $\sigma_{min}= \frac{1}{R_{max}}$ and $\sigma_{max}= \frac{1}{R_{min}}$). This design works using two \ac{opAmp} connected to $V_{cm}$ with the positive pin and the negative pin to the output of the crossbar array. \Cref{eq:doubleMem0,eq:doubleMem1,eq:doubleMem2} are describing how this architecture works. A simplified version of the double memristors per synapse circuit is also available in \cref{circt:doubleMem}.

\begin{figure}[H]
  \centering
  \includesvg[height=8cm,pretex=\scriptsize]{crossbar/doubleMem}
  \caption{Simplified circuit of a double memristor per synapse architecture}
  \label{circt:doubleMem}
\end{figure}

With $x_k$ being the voltage for the input line $k$. The highest \ac{opAmp} is identified as opamp0 and the lowest opamp1.

For the sake of simplicity, the following equations considers the ground to be $V_{cm}$.

\begin{equation}
  \label{eq:doubleMem0}
  V_{opAmp0}=-R_r\cdot i_+ \Leftrightarrow
\end{equation}
\begin{equation}
  \label{eq:doubleMem1}
  i_{R_f}=i_-+\frac{V_{opAmp0}}{R_r}=i_--i_+
\end{equation}
With $i_{opAmp}=0A$ because we assume an ideal \ac{opamp}.
\begin{equation}
  \label{eq:doubleMem2}
  V_{opAmp2}=y_0=-R_f\cdot(i_--i_+)=R_f\cdot(i_+-i_-)=R_f\cdot\sum_{k=0}^n(\sigma_{k+}-\sigma_{k-})\cdot x_k
\end{equation}
With $i_+=\sum_{k=0}^n\sigma_{k+}\cdot x_k$ and $i_-=\sum_{k=0}^n\sigma_{k-}\cdot x_k$.


\begin{table}[H]
  \centering
  \begin{tabular}{|c|c|c|}
    \cline{2-3}
    \rowcolor{gray}
    \multicolumn{1}{c|}{\cellcolor[HTML]{FFFFFF}} & Two memristors per synapse & One memristor per synapse \\
    \hline
    Maximum weight & $\sigma_{max}-\sigma_{min}$ & $\sigma_{max} -\overline{\sigma}$\\
    \hline
    Minimum weight & $\sigma_{min}-\sigma_{max}$ & $\sigma_{min} -\overline{\sigma}$\\
    \hline
    Range & $2\cdot(\sigma_{max}-\sigma_{min})$&$\sigma_{max}-\sigma_{min}$\\
    \hline
  \end{tabular}
  \caption{Synaptic weights precision (extracted from \cite{doubleMem})}
  \label{tab:synapses}
\end{table}

\subsection{Serialization/Parallelization}
\label{subsec:serpar}

The idea using the system in a serial mode came from \cite{thesisRef}. The principle of this system is to save onChip area by reducing the number of \ac{opAmp} required. Using the system in serial mode, increases the time it takes to compute the output of the the \ac{VMM}. Let's assume a serial size of $n_s\in\mathbb{N}$, the number of \ac{opAmp} and resistor is then divided by $n_s$ at the cost of multipying the computation time by $n_s$.

The \ac{CMOS} switches are here to open the necessary input gates when the output is required, the \ac{CMOS} switches are controlled as in \cref{tim:serpar}.

\begin{figure}[H]
  \centering
  \begin{tikztimingtable}%TODO change or find other way
    $e_0$ & x2H2L N(A1) 2L N(A2) 2L N(A3) 2L N(A4) 2Lx\\
    $e_1$ & x2L2H N(B1) 2L N(B2) 2L N(B3) 2L N(B4) 2Lx\\
    $e_i$ & x4L N(C1) 2L N(C2) 2H N(C3) 2L N(C4) 2Lx\\
    $e_{n_s-1}$ & x4L N(D1) 2L N(D2) 2L N(D3) 2L N(D4) 2Hx\\
    %foo & 2L N(A1)  4H N(A2) L\\
    \extracode
    \node[gap, at={($(A1|-A2)!0.5!(A2)$)}];
    \node[gap, at={($(A3|-A4)!0.5!(A4)$)}];
    \node[gap, at={($(B1|-B2)!0.5!(B2)$)}];
    \node[gap, at={($(B3|-B4)!0.5!(B4)$)}];
    \node[gap, at={($(C1|-C2)!0.5!(C2)$)}];
    \node[gap, at={($(C3|-C4)!0.5!(C4)$)}];
    \node[gap, at={($(D1|-D2)!0.5!(D2)$)}];
    \node[gap, at={($(D3|-D4)!0.5!(D4)$)}];
    \tablerules
    %\draw (0,0) circle (2pt); % Origin
    \begin{pgfonlayer}{background}
      \vertlines[help lines]{0.55,12.55}
      %\vertlines[red]{1.6,5.6,15.6}
      %\vertlines[blue]{3.6,9.6,15.6}
    \end{pgfonlayer}
  \end{tikztimingtable}
  \caption{Enable flags timing for any value of $n_s$ in a single time step}
  \label{tim:serpar}
\end{figure}

When the system is used fully in parallel, the \ac{CMOS} switches are not required can then be removed to lower the final onChip area.

%TODO understand better \subsection{Sneak path problem}
\subsection{Symbol}
The symbol (\cref{sym:xbar}) defined for the voltage based memristor crossbar array used in this thesis is more compact and helps the readability of the circuits that require a crossbar array. It depends on several parameters, the number of inputs ($n_i$), the number of outputs ($n_o$) and the serial size ($n_s$).

\begin{figure}[H]
  \centering
  \includesvg[height=2.5cm]{crossbar/xbarSymbol}
  \caption{Symbol used for the crossbar array, the input pin is a bus of size $n_i$ and the output pin is a bus of size $n_o$}
  \label{sym:xbar}
\end{figure}

The total onChip area for the crossbar circuit depends on the previously defined parameters. It also slightly changes depending on the value of $n_s$.

\begin{equation}
  A_{xbar}(n_i,n_o,n_s)=
  \begin{cases}
    2\cdot n_i\cdot n_o \cdot A_{memristor}+n_o\cdot(2\cdot[A_{opAmp}+A_{R_r}]+A_{R_f}),\hfill n_s=1\\
    \begin{array}{2}
      2\cdot n_i\cdot n_o \cdot n_s\cdot A_{memristor}+2\cdot A_{CMOS}\cdot n_o\cdot n_s\\
      +n_o\cdot(2\cdot[A_{opAmp}+A_{R_r}]+A_{R_f})
    \end{array}
    ,\hfill \forall n_s>1\\
  \end{cases}
\end{equation}

With $A_{CMOS}$ being the same than for the memory cell (\cref{sec:memcell}), $A_{CMOS}=1.2\cdot 10^{-14}m^2$, and the feature size of typical memristor that can be fabricated at INESC is $3\mu m$, which would make the approximate area for a memristor $A_{memristor}=9\mu m^2=9\cdot 10^{-12} m^2$.%TODO check info and inesc citation

\subsection{Usage}

\subsubsection{Input voltages}

The input voltages must be monitored very carefuly in order not to mess up the entire system. Indeed, memristors have a voltage threshold ($V_{threshold}$) above which the internal resistance changes. %TODO Find info in the model paper
This means that the input voltages must follow the inequality in \cref{eq:memThres}.

\begin{equation}\label{eq:memThres}
  |V_{mem}|= |x_i-V_{cm}|\le V_{threshold}
\end{equation}

Where $V_{mem}$ is the terminal voltage of a memristor.

\subsubsection{Dense layer}

A dense layer is simply a \ac{VMM}. This means that a dense layer in a \ac{NN} can be assimilated with a crossbar array. Dense layers are mostly used as by fully parallel voltage-driven crossbar array in this thesis to keep the timing simple, otherwise the timing of the enable flags, and later on the \ac{LSTM}'s memory cells (\cref{sec:lstm}), would get too complicated.

\subsubsection{Resistors vs Memristors}

The analog system being only simulated for the inference of the \ac{NN}. Furthermore, since the weights only change during the training phase, and the resistances are just a rendition of the weights (\cref{sec:wei2res}), the resistances don't need to change, the circuit used in the simulations is thus using resistors instead of memristors for the sake of simplicity, as importing resistors is way easier to import in Cadence's virutoso.
