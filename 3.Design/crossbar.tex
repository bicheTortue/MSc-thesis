\section{Crossbar circuit}
\label{sec:xbarCircuit}

The crossbar circuit has already been explained in the section \ref{sec:crossbar}. This section describes how the crossbar circuit is actually implemented. The final circuit is the one in figure \ref{fig:xbarCircuit}.

\begin{figure}[H]
  \centering
  \includesvg[width=\textwidth]{crossbar/crossbarUse}
  \caption{Crossbar circuit used in the final system}
  \label{fig:xbarCircuit}
\end{figure}

First of all, a design with two memristors per synapse has been used. This design is beneficial in two ways :

\begin{itemize}
  \item It doubles the range of weights
  \item It allows to easily use negative weights
\end{itemize}

This design is the one that is used in \cite{doubleMem}. Let's assume that a given memristor has a resistance range of $R\in[R_{min},R_{max}]$, that means it's conductance range is $\sigma \in [\sigma_{min},\sigma_{max}]$ (with $\sigma_{min}= \frac{1}{R_{max}}$ and $\sigma_{max}= \frac{1}{R_{min}}$). The use of a design such as the one specifically displayed if figure TODO. It works with two \ac{opAmp}.

\begin{equation}
  V_{opAmp1}=-R\cdot i_+
\end{equation}
\begin{equation}
  i_{R_f}=i_-+\frac{V_{opAmp1}}{R}=i_--i_+
\end{equation}
With $i_{opAmp}=0A$ because we assume an ideal \ac{opamp}.
\begin{equation}
  V_{opAmp2}=V_{out}=-R_f\cdot(i_--i_+)=R_f\cdot(i_+-i_-)=R_f\cdot\sum_{k=0}^n(\sigma_{k+}-\sigma_{k-})\cdot x_k
\end{equation}
With $i_+=\sum_{k=0}^n\sigma_{k+}\cdot x_k$ and $i_-=\sum_{k=0}^n\sigma_{k-}\cdot x_k$.

In the previous equations, $x_k$ is the voltage for the input line $k$.


\begin{table}[H]
  \centering
  \begin{tabular}{|c|c|c|}
    \cline{2-3}
    \rowcolor{gray}
    \multicolumn{1}{c|}{\cellcolor[HTML]{FFFFFF}} & Two memristors per synapse & One memristor per synapse \\
    \hline
    Maximum weight & $\sigma_{max}-\sigma_{min}$ & $\sigma_{max} -\overline{\sigma}$\\
    \hline
    Minimum weight & $\sigma_{min}-\sigma_{max}$ & $\sigma_{min} -\overline{\sigma}$\\
    \hline
    Range & $2\cdot(\sigma_{max}-\sigma_{min})$&$\sigma_{max}-\sigma_{min}$\\
    \hline
  \end{tabular}
  \caption{Synaptic weights precision (extracted from \cite{doubleMem})}
  \label{tab:synapses}
\end{table}


Since the analog system is only simulated in the inference phase of the \ac{NN}, the resistance doesn't need to change during this phase. The weights only change during the training phase, and the resistances are just a rendition of the weights (section TODO : ref weigths to res)
