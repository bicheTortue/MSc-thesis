\section{The activation functions}
\label{sec:af}

Activation functions play a great role in the results obtained \cite{af}. It is the reason why making good Activation function circuits is fundamental. Of course the easy way would be to simply design a hard sigmoid (TODO : Ref annex), the issue being that the hard sigmoid is much worse than the regular sigmoid, especially for regression problems \cite{hardSigm}. The same goes for the \ac{tanh} and hard \ac{tanh} functions.

Designing an analog activation function as close to the original is very important for the final result's quality.


\subsection{Circuit}

The circuit used is the same as the one in \cite{thesisRef}, the circuit is the one shown in figure \ref{fig:afCircuit}. The technology used being different, all the parameters had to be determined empirically to best fit a sigmoid shape. The parameters can be found in ( TODO : show af parameters ).

Due to the nature of the functions we want to generate, we will use the same circuit for both a sigmoid and a \ac{tanh} like functions. The two different functions are generated by changing two parameters.

The functions generated are the same shape and only differ by their output range.

\begin{figure}[H]
  \centering
  \includesvg[width=\textwidth]{activation/circuit}
  \caption{Activation functions circuit}
  \label{fig:afCircuit}
\end{figure}

The approximate onChip area of this circuit is :

\begin{equation}
  A_{af}=2\cdot A_{R_1} + 2\cdot A_{nmos} + 5\cdot A_{R_2} + A_{R_3} +2\cdot A_{opAmp}
\end{equation}
With $A_{nmos} = w\cdot l = 9\cdot 10^{-7} \cdot 6 \cdot 10^{-8} = 5.4 \cdot 10^{-14} m^2$
TODO : Finish calc

\subsection{Symbol}
The symbol for both the sigmoid and \ac{tanh}

\begin{figure}[H]
  \centering
  \hspace*{1.5cm}
  \subfloat[Sigmoid symbol]{\includesvg[height=2.5cm]{activation/sigmoidSymbol}}%
  \hfill
  \subfloat[\ac{tanh} symbol]{\includesvg[height=2.5cm]{activation/tanhSymbol}}%
  \hspace*{1.5cm}
  \caption{Activation functions symbols}
  \label{fig:afSymbol}
\end{figure}

\subsection{Results}

This circuit outputs a voltage that depends on the input voltage passed on. The relation between the two is shown in figure \ref{fig:afGraph}. The graph also shows the \ac{RMSE} of each graph compared to the ideal result.

The actual output of the circuit shown in figure \ref{fig:afCircuit} is shown in. TODO

\begin{figure}[H]
  \centering
  \includesvg[width=\textwidth]{activation/activGraph}
  \caption{Input/Output graph of the activation function circuit for both sigmoid and \ac{tanh} functions}
  \label{fig:afGraph}
\end{figure}
