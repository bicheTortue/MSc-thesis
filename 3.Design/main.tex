% %%%%%%%%%%%%%%%%%%%%%%%%%%%%%%%%%%%%%%%%%%%%%%%%%%%%%%%%%%%%%%%%%%%%%%
% Dummy Chapter:
% %%%%%%%%%%%%%%%%%%%%%%%%%%%%%%%%%%%%%%%%%%%%%%%%%%%%%%%%%%%%%%%%%%%%%%

% %%%%%%%%%%%%%%%%%%%%%%%%%%%%%%%%%%%%%%%%%%%%%%%%%%%%%%%%%%%%%%%%%%%%%%
% The Introduction:
% %%%%%%%%%%%%%%%%%%%%%%%%%%%%%%%%%%%%%%%%%%%%%%%%%%%%%%%%%%%%%%%%%%%%%%
\fancychapter{The design}
\label{cap:design}

In this chapter, I'm going to describe the different design decisions taken during the study of the thesis.

The system will be working with a $V_{dd}$ of $1.8V$. Such a value was chosen because this is a low power system.

The way values are encoded in the analog system will be descibed here as it serves for the entire thesis.
In order for the system to support negative numbers we're going to use a $V_{cm}$ set to $\frac{V_{dd}}{2}$. That means that $V_{cm}=\frac{V_{dd}}{2}=0.9V$. This $V_{cm}$ will then describe a zero. A step of one was chosen to be $0.1V$ in the analog circuit.
\Cref{tab:valConv} shows the conversions from a real number to it's voltage equivalent.

\begin{table}[H]
  \centering
  \begin{tabular}{|c|c|}
    \hline
    \rowcolor{gray}
    Real value & Voltage \\
    \hline
    $0$ & $0.9V$ \\
    \hline
    $1$ & $1.0V$ \\
    \hline
    $x$ & $\frac{x}{10}+V_{cm}$\\
    \hline
    $(V-0.9)\cdot 10$ & $V$\\
    \hline
  \end{tabular}
  \caption{Real/Voltage Conversion Table.}
  \label{tab:valConv}
\end{table}

Since the system cannot reach voltage outside of the operating range with the intended behavior, the voltage is then restricted to $V\in [0,1.8]$. This means that the range of real value that the systems can handle is $x\in [-9,9]$.

Any data inputed in the analog system should have been previously checked to make sure it stays in the range of accepted values.

\section{The activation functions}
\label{sec:af}

Activation functions play a great role in the results obtained \cite{af}. It is the reason why making good Activation function circuits is fundamental. Of course the easy way would be to simply design a hard sigmoid (\cref{apsec:hardFunc}), the issue being that the hard sigmoid is much worse than the regular sigmoid, especially for regression problems \cite{hardSigm}. The same goes for the \ac{tanh} and hard \ac{tanh} functions.

Designing an analog activation function as close to the original is very important for the final result's quality.

\begin{table}[H]
  \centering
  \begin{tabular}{|c|c|c|}
    \hline
    \rowcolor{gray}
    Parameter & Sigmoid & \ac{tanh} \\
    \hline
    $V_1$ & \multicolumn{2}{c|}{$1.1V$}\\
    \hline
    $V_2$ & \multicolumn{2}{c|}{$635mV$}\\
    \hline
    $V_3$ & $0.8V$ & $550mV$\\
    \hline
    $i_{dc}$ & \multicolumn{2}{c|}{$150uA$}\\
    \hline
    $w$ & \multicolumn{2}{c|}{$900nm$}\\
    \hline
    $l$ & \multicolumn{2}{c|}{$60nm$}\\
    \hline
    $R_1$ & \multicolumn{2}{c|}{$5k\Omega$}\\
    \hline
    $R_2$ & \multicolumn{2}{c|}{$10k\Omega$}\\
    \hline
    $R_3$ & $2k\Omega$ & $4k\Omega$\\
    \hline
  \end{tabular}
  \caption{Circuits parameters}
  \label{tab:afPar}
\end{table}

Here, $w$ and $l$ are, respectively, the width and length of the two NMOS of the circuit.

\subsection{Circuit}

The circuit used is the same as the one in \cite{thesisRef}, the circuit is the one shown in \cref{circt:af}. The technology used being different, all the parameters had to be determined empirically to best fit a sigmoid shape. The parameters can be found in \cref{tab:afPar}.

Due to the nature of the functions we want to generate, we will use the same circuit for both a sigmoid and a \ac{tanh} like functions. The two different functions are generated by changing two parameters.

The functions generated are the same shape and only differ by their output range.

\begin{figure}[H]
  \centering
  \includesvg[width=\textwidth]{activation/afCircuit}
  \caption{Activation functions circuit}
  \label{circt:af}
\end{figure}

The approximate onChip area of this circuit is :

\begin{equation}
  A_{af}=2\cdot A_{R_1} + 2\cdot A_{CMOS} + 5\cdot A_{R_2} + A_{R_3} +2\cdot A_{opAmp}
\end{equation}
With $A_{CMOS} = w\cdot l = 9\cdot 10^{-7} \cdot 6 \cdot 10^{-8} = 5.4 \cdot 10^{-14} m^2$
TODO : Finish calc



\subsection{Symbols}
The symbols for the sigmoid and the \ac{tanh} are separated for better understanding.

\begin{figure}[H]
  \centering
  \hspace*{0.8cm}
  \subfloat[Sigmoid symbol]{\includesvg[height=2.5cm]{activation/sigmoidSymbol}}%
  \hfill
  \subfloat[\ac{tanh} symbol]{\includesvg[height=2.5cm]{activation/tanhSymbol}}%
  \hspace*{0.8cm}
  \caption{Activation functions symbols with the input and output pins on either side depending on the flow of the current for better readability}
  \label{fig:afSymbol}
\end{figure}

\subsection{Usage}

This circuit outputs a voltage that depends on the input voltage passed on. The relation between the two is shown in \cref{fig:afGraph}. The graph also shows the \ac{RMSE} of each graph compared to the ideal result.

Note that the actual output of the circuit in \cref{circt:af} is inverted arround $V_{cm}$ a simple circuit such as the one shown in  (TODO show circuit in annex) does inverts the

\begin{figure}[H]
  \centering
  \includesvg[width=\textwidth]{activation/afGraph}
  \caption{Input/Output graph of the activation function circuit for both sigmoid and \acs{tanh} functions}
  \label{fig:afGraph}
\end{figure}

These functions are still a bit different from the original functions (especially for the \ac{tanh}). However that doesn't matter too much as the trainning will be happening within the final circuit, all weights will be set in the circuit. This is the reason why such a difference doesn't matter. As long as the curves have the similar shape, the result won't be drastically affected.

\section{Memory cells}
\label{sec:memcell}

The memory cell is a circuit that is able to store an analog value for a limited time. It works using capacitors that have the ability to store a voltage for a short time.

The circuit is shown in figure \ref{fig:memcellCircuit}. The value/voltage is trapped in the capacitor using CMOS switches.


I chose to use a two CMOS switches design to avoid voltage leakage through the swicthes. As figure \ref{fig:memcellLoss} shows that only using 1 CMOS swicth leads to a large memory leak due to the high voltage difference on both sides of the CMOS swicth. Using two CMOS switches allows for this difference to be mitigated.

\begin{figure}[H]
  \centering
  \includesvg[width=\textwidth]{memcell/data-loss}
  \caption{Memory conservation in a memory cell with 1 CMOS switch vs 2 CMOS swicthes}
  \label{fig:memcellLoss}
\end{figure}


\begin{figure}[H]
  \centering
  \includesvg[width=\textwidth]{memcell/circuit2}
  \caption{Memory cell circuit}
  \label{fig:memcellCircuit}
\end{figure}

\section{Voltage-driven crossbar circuit}
\label{sec:xbarCircuit}

The crossbar circuit theory has already been explained in the \cref{sec:crossbar}. This section describes how the crossbar circuit is actually implemented. The final circuit is the one in \cref{circt:xbar}. The circuit depends on three parameters :

\begin{itemize}
  \item $n_i$ : The number of input for our crossbar array (not including bias for a more general circuit).
  \item $n_o$ : The number of parallel output for our crossbar array.
  \item $n_s$ : The serial size of our crossbar system.
\end{itemize}

\begin{figure}[H]
  \centering
  \includesvg[width=\textwidth]{crossbar/crossbarUse}
  \caption{Circuit of the crossbar array used in the final system ($n_i$, $n_o$, $n_s$)}
  \label{circt:xbar}
\end{figure}

The parallel and serial are explained later (\cref{subsec:serpar}).

The enable flags ($e_j,\forall j\in[\![ 0,n_s]\!]$ and similarly $\overline{e_j}$) are there to show when the \ac{CMOS} switches are open, the states of those flags is shown in \cref{tim:serpar}.

\subsection{Two memristors per synapse}
\label{subsec:doubleMem}

First of all, a two memristors per synapse architecture has been used. This design is beneficial in two ways :

\begin{itemize}
  \item It doubles the range of weights (\cref{tab:synapses})
  \item It allows to easily use negative weights, as this design simply allows the output values to be centered around $V_{cm}$ to be compliant with the standard set in \cref{tab:valConv}.
\end{itemize}

This design is the one that is used in \cite{doubleMem}. Let's assume that a given memristor has a resistance range of $R\in[R_{min},R_{max}]$, that means it's conductance range is $\sigma \in [\sigma_{min},\sigma_{max}]$ (with $\sigma_{min}= \frac{1}{R_{max}}$ and $\sigma_{max}= \frac{1}{R_{min}}$). This design works using two \ac{opAmp} connected to $V_{cm}$ with the positive pin and the negative pin to the output of the crossbar array. \Cref{eq:doubleMem0,eq:doubleMem1,eq:doubleMem2} are describing how this architecture works. A simplified version of the double memristors per synapse circuit is also available in \cref{circt:doubleMem}.

\begin{figure}[H]
  \centering
  \includesvg[height=8cm]{crossbar/doubleMem}
  \caption{Simplified circuit of a double memristor per synapse architecture}
  \label{circt:doubleMem}
\end{figure}

With $x_k$ being the voltage for the input line $k$. The highest \ac{opAmp} is identified as opamp0 and the lowest opamp1.

For the sake of simplicity, the following equations considers the ground to be $V_{cm}$.

\begin{equation}
  \label{eq:doubleMem0}
  V_{opAmp0}=-R\cdot i_+ \Leftrightarrow
\end{equation}
\begin{equation}
  \label{eq:doubleMem1}
  i_{R_f}=i_-+\frac{V_{opAmp0}}{R}=i_--i_+
\end{equation}
With $i_{opAmp}=0A$ because we assume an ideal \ac{opamp}.
\begin{equation}
  \label{eq:doubleMem2}
  V_{opAmp2}=y_0=-R_f\cdot(i_--i_+)=R_f\cdot(i_+-i_-)=R_f\cdot\sum_{k=0}^n(\sigma_{k+}-\sigma_{k-})\cdot x_k
\end{equation}
With $i_+=\sum_{k=0}^n\sigma_{k+}\cdot x_k$ and $i_-=\sum_{k=0}^n\sigma_{k-}\cdot x_k$.


\begin{table}[H]
  \centering
  \begin{tabular}{|c|c|c|}
    \cline{2-3}
    \rowcolor{gray}
    \multicolumn{1}{c|}{\cellcolor[HTML]{FFFFFF}} & Two memristors per synapse & One memristor per synapse \\
    \hline
    Maximum weight & $\sigma_{max}-\sigma_{min}$ & $\sigma_{max} -\overline{\sigma}$\\
    \hline
    Minimum weight & $\sigma_{min}-\sigma_{max}$ & $\sigma_{min} -\overline{\sigma}$\\
    \hline
    Range & $2\cdot(\sigma_{max}-\sigma_{min})$&$\sigma_{max}-\sigma_{min}$\\
    \hline
  \end{tabular}
  \caption{Synaptic weights precision (extracted from \cite{doubleMem})}
  \label{tab:synapses}
\end{table}

\subsection{Serialization/Parallelization}
\label{subsec:serpar}

The idea using the system in a serial mode came from \cite{thesisRef}. The principle of this system is to save onChip area by reducing the number of \ac{opAmp} required. It also, and mainly reduces the point wise part of the \ac{LSTM} circuit.

This serial system has great advantages but comes with a few down sides. Let's assume the number of hidden state ($n_h$) is the major changing element in the final system(TODO ref final onchip area).

\begin{itemize}
  \item When the system is serialized, the inference time increases with a factor of $O(n_h)$.
  \item When the system is in parallel, the circuit onChip area increases with a factor of $O(n_h)$.
\end{itemize}

However, in both cases, the overall onChip area increases with a factor of $O(n_h^2)$. That means that using the parallel system won't increase the onChip area by a lot as $n_h$ rises, because the $O(n_h)$ factor of the parallel version is absorbed as $O(n_h^2+n_h)=O(n_h^2)$.

That leads to the conclusion that the best thing to do is use a both serialized and parallel system, using the limiting inference time for our system to set the maximum serial size. Depending on the system we're using this circuit for, we might have different time requirements.

The \ac{CMOS} switches are here to open the necessary input gates when the output is required, the gates are controlled as in \cref{tim:serpar}.
\begin{figure}[H]
  \centering
  \begin{tikztimingtable}%TODO change or find other way
    $e_0$ & xH3Lx\\
    $e_1$ & xLH2Lx\\
    $e_2$ & x2LHLx\\
    $e_3$ & x3LHx\\
    \extracode
    \tablerules
    %\draw (0,0) circle (2pt); % Origin
    \begin{pgfonlayer}{background}
      \vertlines[help lines]{0.6,4.6}
      %\vertlines[red]{1.6,5.6,15.6}
      %\vertlines[blue]{3.6,9.6,15.6}
    \end{pgfonlayer}
  \end{tikztimingtable}
  \caption{Example of enable flags timing with $n_s=4$ in a single time step}
  \label{tim:serpar}
\end{figure}

%TODO understand better \subsection{Sneak path problem}
\subsection{Symbol}
The symbol (\cref{sym:xbar}) defined for the voltage based memristor crossbar array used in this thesis is more compact and helps the readability of the final circuit. It contains the size of each parameters on its symbol. It depends on several parameters, the number of inputs ($n_i$), the number of outputs ($n_o$) and the serial size ($n_s$).

\begin{figure}[H]
  \centering
  \includesvg[height=2.5cm]{crossbar/xbarSymbol}
  \caption{Symbol used for the crossbar array}
  \label{sym:xbar}
\end{figure}

The total onChip area for the crossbar circuit depends on the previously defined parameters.

\begin{equation}
  A_{xbar}(n_i,n_o,n_s)=2\cdot n_i\cdot n_o \cdot n_s\cdot A_{memristor}+2\cdot A_{CMOS}\cdot n_o\cdot n_s +n_o\cdot(2\cdot[A_{opAmp}+A_R]+A_{R_f})
\end{equation}

With $A_{CMOS}$ being the same than for the memory cell (\cref{sec:memcell}), $A_{CMOS}=1.2\cdot 10^{-14}m^2$, and the feature size of typical memristor that can be fabricated at INESC is $3\mu m$, which would make the approximate area for a memristor $A_{memristor}=9\mu m^2=9\cdot 10^{-12} m^2$.%TODO check info and inesc citation

\subsection{Usage}

%TODO Show off results ??


The analog system is only simulated in the inference phase of the \ac{NN}. And since the weights only change during the training phase, and the resistances are just a rendition of the weights (\cref{sec:wei2res}), the resistances don't need to change, the circuit used in the simulations is thus using resistors instead of memristors for the sake of simplicity, as importing resistors is way easier to import in Cadence's virutoso.

\section{Inverter}
\label{sec:inv}

This section simply describes a very well know electrical component, the inverter. There is nothing special about this specific inverter. When it a voltage of $V_{dd}$ is applied to the input, the output is grounded and the same goes for the other way around.

\begin{figure}[H]
  \centering
  \hspace*{2.5cm}
  \subfloat[Inverter's circuit]{\label{fig:invCircuit}\includesvg[height=3cm]{inverter/invCircuit}}%
  \hfill
  \subfloat[Inverter's symbol]{\label{fig:invSymbol}\includesvg[height=2.5cm]{inverter/invSymbol}}%
  \hspace*{1.5cm}
  \caption{}
  \label{fig:inv}
\end{figure}

Figure \ref{fig:inv} combines the circuit and the symbol of the inverter used in this thesis.

\section{Verilog models}
\label{sec:models}

Due to a lack of time, some of the more common component were not designed by me but instead were simulated using a Verilog model. The only components that use a verilog model are the voltage multiplier and the \ac{opAmp}.

\subsection{Operational amplifier}

This component is the very famous \ac{opAmp}. I didn't design a specific \ac{opAmp} because of the current range of the system I'm trying to build.

This is the code (figure TODO : in annex) of a perfect \ac{opAmp}. TODO : more description.
TODO : add symbol.

\subsection{Voltage multiplier}

This component while a bit less popular than the latter, is just as useful for our specific use. It allows us to multiply, as its name implies, two voltages. It is used to compute the pointwise multiplications of the \ac{LSTM} (figure \ref{fig:lstmCell}).

It must follow the following equation :

\begin{equation}
  V_{out}=(V_{in_1}-V_{cm})\cdot (V_{in_2}-V_{cm}) + V_{cm}
\end{equation}

Where $V_{out}$ is the output voltage of the circuit and $V_{in_1}$ and $V_{in_2}$ are the input voltages.

However, the output of the voltage multiplier needs to be ajusted after because of the voltage value of a unit. Indeed, we need to have $1\cdot 1=1$, but if we set $V_{in_1}=V_{in_2}=1V$, we would find $V_{out}=0.91V$. To fix this issue, a small circuit added following the output of the voltage multiplier, this circuit needs to do the following :

\begin{equation}
  V_{amp}=(V_{out}-V_{cm})\cdot10+V_{cm}
\end{equation}

Where $V_{amp}$ is the output of the amplificator and $V_{out}$ is the output of the voltage multiplier and the input of the amplifier circuit.
This circuit is shown in appendix \ref{apsec:amp}.

\section{Netlist generation}
\label{sec:netlist}

In order to be able to run the simulation with different kinds of \ac{NN} architecture. The point of the part is thus to explain how the netlist generator tool works. The tool also work to generate any architecture with the supported layers (listed in the README.md of \cite{lstmGen}).

This is done by generating a SPICE netlist using a python script. This script takes in a few parameters :

\begin{itemize}
  \item The first parameter is the number of input we use our system. This is the size of the first input vector ($x_t$).
  \item The next parameter is the number of time steps. Everything about the \ac{LSTM} time steps is all explained in \cref{sec:lstm}.
  \item The serial size of the crossbar arrays, as described in \cref{sec:xbarCircuit}, used in the \ac{LSTM} network. This parameter must divide the number of hidden states in the \acp{LSTM} layers.
  \item The files containing different info about the model. This file contains the type of \ac{NN} architecture and the weights associated with each layer. The weights in the files have to be organized using a specific model (\cref{subsec:weiStore}).
  \item Finally the name of the file in which the output of the script (the netlist) will be written to.
\end{itemize}

\subsection{Weights storage}\label{subsec:weiStore}


The weights are stored in a single file that is the save a numpy array. This file also contains a brief description of the architecture being used. The first index stores this desciption. the following indexes are for the actual weights, in the order given in the description. Depending on the layer used, the weights are stored a bit differently :

\begin{itemize}
  \item Dense layer : This is basically a \ac{VMM} so the weights are just sorted linearly in a list. The list is of size $n^2$, where $n$ is the size of the input vector. \Cref{mtrx:wei} shows the position ($i$) of the weight ($w_i$) in the matrix. This is represented in the description as "Dense($n_o$)" where $n_o$ is the number of output
  \item \ac{LSTM} layer : An \ac{LSTM} layer contains four \acp{VMM} which can be assimilated and thus stored like a dense layer. The weights for each \ac{VMM} is then stored in a sub list.
\end{itemize}

\begin{equation}\label{mtrx:wei}
  \begin{bmatrix}
    w_{0} & w_{1} & \dots \\
    \vdots & w_i & \vdots \\
    \dots & w_{n^2-2} & w_{n^2-1}\\
  \end{bmatrix}
\end{equation}

All the code can be found on my github page \cite{lstmGen}.

\section{Conversion from weights to resistance}
\label{sec:wei2res}

This chapter will contain all the information on the weight conversion to a resistance value.

The weight to resistances equations has been found using simple math.

\begin{equation}
  \label{eq:wei2res0}
  w=R_f\cdot(\sigma_+-\sigma_-)=\frac{R_f}{R_+}-\frac{R_f}{R_-}
\end{equation}
First, we know from \cref{eq:doubleMem2} that the weight is represented by \cref{eq:wei2res0}. This means that the weights are limited in values to :

\begin{itemize}
  \item $w_{max}=R_f\cdot(\sigma_{max}-\sigma_{min})$
  \item $w_{min}=-w_{max}=R_f\cdot(\sigma_{min}-\sigma_{max})$
\end{itemize}

For the rest of this chapter we consider that $R_f$ is set to the middle point of $R_{max}$ and $R_{min}$ meaning that $R_f=\frac{R_{min}+R_{max}}{2}$.

Since we only have one equation (\cref{eq:wei2res0}) for 2 unknowns ($R_+$ and $R_-$), we need to set a second equation. This is done by centering the resistances around $R_f$, this means that \cref{eq:wei2res1} is the second equation, that makes our problem now solvable.
%TODO explain how in future work the weight doesn't have to be centered around R_f, to gain time only one of the two resistance could change
\begin{equation}
  \label{eq:wei2res1}
  R_f=\frac{R_-+R_+}{2}
\end{equation}
\begin{equation}
  \label{eq:wei2res01}
  \begin{cases}
    w=\frac{R_f}{R_+}-\frac{R_f}{R_-}\\
    R_f=\frac{R_-+R_+}{2}
  \end{cases}
\end{equation}

By solving \cref{eq:wei2res01}, we find the \cref{eq:wei2res2} that gives the values for $R_-$ and $R_+$. All the steps for solving the equations can be found in .%TODO solve equations in appendix

The real value to voltage conversion (\cref{tab:valConv}) doesn't affect this part of the system as the crossbar array already works in voltages. Replacing the resistor variable in \cref{eq:doubleMem2} by $w$ using \cref{eq:wei2res0} would not change the results.

\begin{equation}
  \label{eq:wei2res2}
  \begin{cases}
    R_+= (w+1-\sqrt{w^2+1})\cdot\frac{R_f}{w}\\
    R_-=2\cdot R_f -R_+
  \end{cases}
\end{equation}

For the thesis, this step is done in python (TODO show function). Since the resistance resolution is of about%TODO check memristor precision
, the python code then reproduces this by limiting the amount of significant digit of the resistance.

\cleardoublepage
