% %%%%%%%%%%%%%%%%%%%%%%%%%%%%%%%%%%%%%%%%%%%%%%%%%%%%%%%%%%%%%%%%%%%%%%
% Dummy Chapter:
% %%%%%%%%%%%%%%%%%%%%%%%%%%%%%%%%%%%%%%%%%%%%%%%%%%%%%%%%%%%%%%%%%%%%%%

% %%%%%%%%%%%%%%%%%%%%%%%%%%%%%%%%%%%%%%%%%%%%%%%%%%%%%%%%%%%%%%%%%%%%%%
% The Introduction:
% %%%%%%%%%%%%%%%%%%%%%%%%%%%%%%%%%%%%%%%%%%%%%%%%%%%%%%%%%%%%%%%%%%%%%%
\fancychapter{The design}
\label{cap:design}

In this chapter, I'm going to describe the different design decisions taken during the study of the thesis.

The system will be working with a $V_{dd}$ of $1.8V$. Such a value was chosen because this is a low power system.

The way values are encoded in the analog system will be descibed here as it serves for the entire thesis.
In order for the system to support negative numbers we're going to use a $V_{cm}$ set to $\frac{V_{dd}}{2}$. That means that $V_{cm}=\frac{V_{dd}}{2}=0.9V$. This $V_{cm}$ will then describe a zero. A step of one was chosen to be $0.1V$ in the analog circuit.
Table \ref{tab:valConv} shows the conversions from a real number to it's voltage equivalent.

\begin{table}[H]
  \centering
  \caption{Real/Voltage Conversion Table.}
  \begin{tabular}{|c|c|}
    \hline
    \rowcolor{gray}
    Real value & Voltage \\
    \hline
    $0$ & $0.9V$ \\
    \hline
    $1$ & $1.0V$ \\
    \hline
    $x$ & $\frac{x}{10}+V_{cm}$\\
    \hline
    $(V-0.9)\cdot 10$ & $V$\\
    \hline
  \end{tabular}
  \label{tab:valConv}
\end{table}






Since the system cannot reach voltage outside of the operating range with the intended behavior, the voltage is then restricted to $V\in [0,1.8]$. This means that the range of real value that the systems can handle is $x\in [-9,9]$.

Any data inputed in the analog system should have been previously checked to make sure it stays in the range of accepted values.

\section{The activation functions}
\label{sec:af}

Activation functions play a great role in the results obtained \cite{af}. It is the reason why making good Activation function circuits is fundamental. Of course the easy way would be to simply design a hard sigmoid (\cref{apsec:hardFunc}), the issue being that the hard sigmoid is much worse than the regular sigmoid, especially for regression problems \cite{hardSigm}. The same goes for the \ac{tanh} and hard \ac{tanh} functions.

Designing an analog activation function as close to the original is very important for the final result's quality.

\begin{table}[H]
  \centering
  \begin{tabular}{|c|c|c|}
    \hline
    \rowcolor{gray}
    Parameter & Sigmoid & \ac{tanh} \\
    \hline
    $V_1$ & \multicolumn{2}{c|}{$1.1V$}\\
    \hline
    $V_2$ & \multicolumn{2}{c|}{$635mV$}\\
    \hline
    $V_3$ & $0.8V$ & $550mV$\\
    \hline
    $i_{dc}$ & \multicolumn{2}{c|}{$150uA$}\\
    \hline
    $w$ & \multicolumn{2}{c|}{$900nm$}\\
    \hline
    $l$ & \multicolumn{2}{c|}{$60nm$}\\
    \hline
    $R_1$ & \multicolumn{2}{c|}{$5k\Omega$}\\
    \hline
    $R_2$ & \multicolumn{2}{c|}{$10k\Omega$}\\
    \hline
    $R_3$ & $2k\Omega$ & $4k\Omega$\\
    \hline
  \end{tabular}
  \caption{Circuits parameters}
  \label{tab:afPar}
\end{table}

Here, $w$ and $l$ are, respectively, the width and length of the two NMOS of the circuit.

\subsection{Circuit}

The circuit used is the same as the one in \cite{thesisRef}, the circuit is the one shown in \cref{circt:af}. The technology used being different, all the parameters had to be determined empirically to best fit a sigmoid shape. The parameters can be found in \cref{tab:afPar}.

Due to the nature of the functions we want to generate, we will use the same circuit for both a sigmoid and a \ac{tanh} like functions. The two different functions are generated by changing two parameters.

The functions generated are the same shape and only differ by their output range.

\begin{figure}[H]
  \centering
  \includesvg[width=\textwidth]{activation/afCircuit}
  \caption{Activation functions circuit}
  \label{circt:af}
\end{figure}

The approximate onChip area of this circuit is :

\begin{equation}
  A_{af}=2\cdot A_{R_1} + 2\cdot A_{CMOS} + 5\cdot A_{R_2} + A_{R_3} +2\cdot A_{opAmp}
\end{equation}
With $A_{CMOS} = w\cdot l = 9\cdot 10^{-7} \cdot 6 \cdot 10^{-8} = 5.4 \cdot 10^{-14} m^2$
TODO : Finish calc



\subsection{Symbols}
The symbols for the sigmoid and the \ac{tanh} are separated for better understanding.

\begin{figure}[H]
  \centering
  \hspace*{0.8cm}
  \subfloat[Sigmoid symbol]{\includesvg[height=2.5cm]{activation/sigmoidSymbol}}%
  \hfill
  \subfloat[\ac{tanh} symbol]{\includesvg[height=2.5cm]{activation/tanhSymbol}}%
  \hspace*{0.8cm}
  \caption{Activation functions symbols with the input and output pins on either side depending on the flow of the current for better readability}
  \label{fig:afSymbol}
\end{figure}

\subsection{Usage}

This circuit outputs a voltage that depends on the input voltage passed on. The relation between the two is shown in \cref{fig:afGraph}. The graph also shows the \ac{RMSE} of each graph compared to the ideal result.

Note that the actual output of the circuit in \cref{circt:af} is inverted arround $V_{cm}$ a simple circuit such as the one shown in  (TODO show circuit in annex) does inverts the

\begin{figure}[H]
  \centering
  \includesvg[width=\textwidth]{activation/afGraph}
  \caption{Input/Output graph of the activation function circuit for both sigmoid and \acs{tanh} functions}
  \label{fig:afGraph}
\end{figure}

These functions are still a bit different from the original functions (especially for the \ac{tanh}). However that doesn't matter too much as the trainning will be happening within the final circuit, all weights will be set in the circuit. This is the reason why such a difference doesn't matter. As long as the curves have the similar shape, the result won't be drastically affected.

\section{Memory cells}
\label{sec:memcell}

The memory cell is a circuit that is able to store an analog value for a limited time. It works using capacitors that have the ability to store a voltage for a short time.

The circuit is shown in figure \ref{fig:memcellCircuit}. The value/voltage is trapped in the capacitor using CMOS switches.


I chose to use a two CMOS switches design to avoid voltage leakage through the swicthes. As figure \ref{fig:memcellLoss} shows that only using 1 CMOS swicth leads to a large memory leak due to the high voltage difference on both sides of the CMOS swicth. Using two CMOS switches allows for this difference to be mitigated.

\begin{figure}[H]
  \centering
  \includesvg[width=\textwidth]{memcell/data-loss}
  \caption{Memory conservation in a memory cell with 1 CMOS switch vs 2 CMOS swicthes}
  \label{fig:memcellLoss}
\end{figure}


\begin{figure}[H]
  \centering
  \includesvg[width=\textwidth]{memcell/circuit2}
  \caption{Memory cell circuit}
  \label{fig:memcellCircuit}
\end{figure}

\section{Inverter}
\label{sec:inv}

This section simply describes a very well know electrical component, the inverter. There is nothing special about this specific inverter. When it a voltage of $V_{dd}$ is applied to the input, the output is grounded and the same goes for the other way around.

\begin{figure}[H]
  \centering
  \hspace*{2.5cm}
  \subfloat[Inverter's circuit]{\label{fig:invCircuit}\includesvg[height=3cm]{inverter/invCircuit}}%
  \hfill
  \subfloat[Inverter's symbol]{\label{fig:invSymbol}\includesvg[height=2.5cm]{inverter/invSymbol}}%
  \hspace*{1.5cm}
  \caption{}
  \label{fig:inv}
\end{figure}

Figure \ref{fig:inv} combines the circuit and the symbol of the inverter used in this thesis.

\section{Verilog models}
\label{sec:models}

Due to a lack of time, some of the more common component were not designed by me but instead were simulated using a Verilog model. The only components that use a verilog model are the voltage multiplier and the \ac{opAmp}.

\subsection{Operational amplifier}

This component is the very famous \ac{opAmp}. I didn't design a specific \ac{opAmp} because of the current range of the system I'm trying to build.

This is the code (figure TODO : in annex) of a perfect \ac{opAmp}. TODO : more description.
TODO : add symbol.

\subsection{Voltage multiplier}

This component while a bit less popular than the latter, is just as useful for our specific use. It allows us to multiply, as its name implies, two voltages. It is used to compute the pointwise multiplications of the \ac{LSTM} (figure \ref{fig:lstmCell}).

It must follow the following equation :

\begin{equation}
  V_{out}=(V_{in_1}-V_{cm})\cdot (V_{in_2}-V_{cm}) + V_{cm}
\end{equation}

Where $V_{out}$ is the output voltage of the circuit and $V_{in_1}$ and $V_{in_2}$ are the input voltages.

However, the output of the voltage multiplier needs to be ajusted after because of the voltage value of a unit. Indeed, we need to have $1\cdot 1=1$, but if we set $V_{in_1}=V_{in_2}=1V$, we would find $V_{out}=0.91V$. To fix this issue, a small circuit added following the output of the voltage multiplier, this circuit needs to do the following :

\begin{equation}
  V_{amp}=(V_{out}-V_{cm})\cdot10+V_{cm}
\end{equation}

Where $V_{amp}$ is the output of the amplificator and $V_{out}$ is the output of the voltage multiplier and the input of the amplifier circuit.
This circuit is shown in appendix \ref{apsec:amp}.

\cleardoublepage
