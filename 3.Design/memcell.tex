\section{Memory cells}
\label{sec:memcell}

The memory cell is a circuit that is able to store an analog value for a limited time. It works using capacitors that have the ability to store a voltage for a short period of time.

\subsection{Circuit}

The circuit is shown in figure \ref{fig:memcellCircuit}. The value/voltage is trapped in the capacitor using CMOS switches.

\begin{figure}[H]
  \centering
  \includesvg[width=\textwidth]{memcell/memCircuit}
  \caption{Memory cell circuit}
  \label{fig:memcellCircuit}
\end{figure}

The cirucit has a two CMOS switches design to avoid voltage leakage through the swicthes. The system has voltage leakage when only one CMOS swicth, and thus leads to a large memory leak. This is due to the high voltage difference between the two sides of the CMOS swicth. Using two CMOS switches allows for this difference to be mitigated (figure \ref{fig:memcellLoss}).

\begin{figure}[H]
  \centering
  \includesvg[width=\textwidth]{memcell/data-loss}
  \caption{Memory conservation in a memory cell with 1 CMOS switch vs 2 CMOS swicthes}
  \label{fig:memcellLoss}
\end{figure}

For this circuit the approximate onChip area :

\begin{equation}
  A_{memcell}=8\cdot A_{CMOS}+2\cdot A_{inv}+A_{opAmp}+A_{cap}
\end{equation}

With $A_{CMOS} = w\cdot l = 9\cdot 10^{-7} \cdot 6 \cdot 10^{-8} = 5.4 \cdot 10^{-14} m^2$ TODO check value w et l

\subsection{Symbol}

The symbol for this circuit is designed to show a capacitor because it is its memory mechanism.

\begin{figure}[H]
  \centering
  \includesvg[height=2.5cm]{memcell/memcellSymbol}
  \caption{Memory cell symbol with the input pin (left), the output pin (right), the input enable pin (top) and the output enable pin (bottom)}
  \label{fig:memcellCircuit}
\end{figure}

\subsection{Usage}

This circuit is pretty straight forward, when the input enable pin is high (up to $V_{dd}$), then the first switch is open and the capacitor is being charged, if it is low, then the switch is closed and the capacitor keeps the voltage and thus the analog data. When the output enable is high, the buffer (appendix \ref{apsec:buffer}) forwards the voltage stored in the capactor to the output pin without emptying the capacitor.

\begin{figure}[H]
  \centering
  \begin{tikztimingtable}
    Input data & x6D{$x_1$}6D{$x_2$}6D{$x_3$}x \\
    Input enable & xL2H2L3H7L2HLx\\
    Output enable & x3L2H4L5H1L3Hx\\
    Memory & xU4D{$x_1$}10D{$x_2$}3D{$x_3$}x \\
    Output data & x3U2D{$x_1$}4U5D{$x_2$}U3D{$x_3$}x \\
    \extracode
    \tablerules
    %\draw (0,0) circle (2pt); % Origin
    \begin{pgfonlayer}{background}
      \vertlines[help lines]{0.6,18.6}
      %\vertlines[red]{1.6,5.6,15.6}
      %\vertlines[blue]{3.6,9.6,15.6}
    \end{pgfonlayer}
  \end{tikztimingtable}
  \caption{Time diagram that shows the logic of the memory cell}
  \label{tim:memcell}
\end{figure}
