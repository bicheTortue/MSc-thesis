\section{Memory cells}
\label{sec:memcell}

The memory cell is a circuit that is able to store an analog value for a limited time. It works using capacitors that have the ability to store a voltage for a short time.

The circuit is shown in figure \ref{fig:memcellCircuit}. The value/voltage is trapped in the capacitor using CMOS switches.


I chose to use a two CMOS switches design to avoid voltage leakage through the swicthes. As figure \ref{fig:memcellLoss} shows that only using 1 CMOS swicth leads to a large memory leak due to the high voltage difference on both sides of the CMOS swicth. Using two CMOS switches allows for this difference to be mitigated.

\begin{figure}[H]
  \centering
  \includesvg[width=\textwidth]{memcell/data-loss}
  \caption{Memory conservation in a memory cell with 1 CMOS switch vs 2 CMOS swicthes}
  \label{fig:memcellLoss}
\end{figure}


\begin{figure}[H]
  \centering
  \includesvg[width=\textwidth]{memcell/circuit2}
  \caption{Memory cell circuit}
  \label{fig:memcellCircuit}
\end{figure}
