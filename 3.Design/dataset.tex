\section{The datasets}\label{sec:dataset}

\subsection{The format}

The values for the data needs to be set not to exceed 1. This is to make sure that the terminal voltage doesn't reach the threshold as explained in \cref{sec:xbarCircuit}.

\subsection{Airline}

The first dataset contains a time series of international airline passengers from January 1949 to December 1960, the data is recorded monthly and is in thousands. The dataset then contains twelve years of monthly data so $144$ sample points. The results of such a problem aren't useful, it was used in this thesis because of its simplicity to check whether the analog circuit is working. It is extracted from the tutorial available at \cite{airline}. The dataset is available at \cite{datasets}, the name of the file containing the dataset is \textit{airline-passengers.csv}.

This is a regression problem, the role of the \ac{LSTM} is to predict the number of passenger flying the next month being given previous months' passenger count.

\begin{figure}[H]
  \centering
  \includesvg[width=\textwidth]{datasets/airline}
  \caption{the airline dataset in a grphic. This is the curve that the regression is trying to reproduce. The vertical lines represent a full year.}
  \label{graph:airline}
\end{figure}

\subsubsection{Data format}

The dataset contains $144$ sample points. It has been transformed into $142$ data points for training. This has been done by taking each values three by three. The first two values are two timesteps for the input vector and the third value being used the target value. Two third of the dataset is being used for training and the other third is used for validation.

\subsubsection{Network configuration}

The layers used to solve this problem are listed below :

\begin{itemize}
  \item An \ac{LSTM} with four hidden states ($n_h=4$) and an input with feature size of one and two time steps.
  \item A Dense layer with an output size of one.
\end{itemize}

\Cref{fig:airlineModel} is a graphical representation of the model just described.

\begin{figure}[H]
  \centering
  \includesvg[width=\textwidth]{datasets/airlineModel}
  \caption{Model used to solve the airline passengers problem}
  \label{fig:airlineModel}
\end{figure}

\subsubsection{Results} %TODO Move to results obviously

The weights are trained using both the custom activation function (\cref{sec:genwei}) and without ones and compare their output.

After training on a digital computer using the settings described in \cref{sec:genwei}.

\subsection{\ac{C. elegans}}

This dataset is far more interresting than the latter. This data set aims to use \acp{LSTM} to mimic the behavior of real neurons. As explained in \cite{celegans}, \acf{C. elegans} are simple organisms that are getting very popular for whole brain organization studies. The point of this problem is to reproduce the nervous system of the \ac{C. elegans}. This is done using recorded data of the input of 4 neurons and the output of 4 other neurons.

The dataset is great for our study because :
\begin{itemize}
  \item It comes from a very recent paper, and means the research is happening right now.
  \item It aims at reproducing the behavior of the brain of a simple organism (\ac{C. elegans}), domain in which having a small and low powered system to mimic is an advantage.
  \item The paper was written by Dr. Barbulescu who is one of my supervisor.
\end{itemize}

\subsubsection{Data format}

The dataset contains $144$ sample points. It has been transformed into $142$ data points for training. This has been done by taking each values three by three. The first two values are two timesteps for the input vector and the third value being used the target value. Two third of the dataset is being used for training and the other third is used for validation.

\subsubsection{Network configuration}

The layers used to solve this problem are listed below :

\begin{itemize}
  \item An \ac{LSTM} with eight hidden states ($n_h=8$) and an input with feature size of four (because of the four input neurons that are considered) and one thousand time steps.
  \item A Dense layer with an output size of one.
\end{itemize}

\Cref{fig:airlineModel} is a graphical representation of the model just described.

\begin{figure}[H]
  \centering
  \includesvg[width=\textwidth]{datasets/celegansModel}
  \caption{Model used to solve the \ac{C. elegans} problem}
  \label{fig:celegansModel}
\end{figure}
