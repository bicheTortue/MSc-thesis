\section{\aclp{NN}}\label{sec:nn}

\acfp{NN} are a set of units known as neurons. Those neurons are linked to each other with arcs known as synapses, thoses synapses each have a weight associated to them. The set of neurons interconnected with their synapses is what is called a \acl{NN}.
\Cref{fig:snn}, shows a simple representation of a \ac{NN}, the artificial neurons are the represented by the colored circles. On \cref{fig:snn} each arrow represent a synapse.

\begin{figure}[h!]
  \centering
  \includesvg[height=8cm]{NN_explained.svg}
  \caption{Simple \acl{NN}}
  \label{fig:snn}
\end{figure}

\acp{NN} contains several layers :

\begin{itemize}
  \item Input layer : This layer is simply the different inputs.
  \item Hidden layer : This layer can be (and usually is) wider than the one in figure \ref{fig:snn}. This is the layer that can be modified the most, by adding layers or increasing the amount of neurons in a layer.
  \item Output layer : This layer is where you can find the result from the \ac{NN}.
\end{itemize}

The model presented in \cref{fig:snn} is modular and can be scaled up as much as required. It is generally accepted that the more neurons a \ac{NN} has, the best it can learn and execute the given task.

The weights of the synapses have to be multiplied with the previous neuron and then added to each other to produce the next stage. Using the names defined in figure \ref{fig:snn}, the output is linked to the input by \cref{eq:nnHid,eq:nnOut}. First, the hidden layers' neurons need to be computed (\cref{eq:nnHid}).

\begin{equation}\label{eq:nnHid}
  \begin{bmatrix}
    H_1\\ H_2\\ H_3\\ H_4\\
  \end{bmatrix}
  =
  \begin{bmatrix}
    w_h_{1,1} & w_h_{1,2} & w_h_{1,3}\\
    w_h_{2,1} & w_h_{2,2} & w_h_{2,3}\\
    w_h_{3,1} & w_h_{3,2} & w_h_{3,3}\\
    w_h_{4,1} & w_h_{4,2} & w_h_{4,3}\\
  \end{bmatrix}
  \cdot
  \begin{bmatrix}
    I_1\\ I_2\\ I_3\\
  \end{bmatrix}
\end{equation}

Similarly the output is computed like in \cref{eq:nnOut}

\begin{equation}\label{eq:nnOut}
  \begin{bmatrix}
    O_1\\ O_2
  \end{bmatrix}
  =
  \begin{bmatrix}
    w_o_{1,1} & w_o_{1,2} & w_o_{1,3} & w_o_{1,4}\\
    w_o_{2,1} & w_o_{2,2} & w_o_{2,3} & w_o_{2,4}\\
  \end{bmatrix}
  \cdot
  \begin{bmatrix}
    H_1\\ H_2\\ H_3\\ H_4\\
  \end{bmatrix}
\end{equation}

Those matrix multiplication are called \ac{VMM} because it is the result of the multiplication of a vector and a matrix, thus giving us another vector.

\subsection{Training weights}

The weights values are obtained through training. This is done in a few steps :

\begin{enumerate}
  \item Generate random values for each weight.
  \item Run the \ac{NN} to get an output vector.\label{step:restart}
  \item Measure error rate using the chosen error algorithm.
  \item Use the
  \item go back to \cref{step:restart}
\end{enumerate}
