\section{\acs{GRU}}
The \acf{GRU} is another type of \ac{RNN}. It is also known to reduce the effect of the vanishing gradient problem. It was first introduced to improve translation techniques \cite{gru}.

The \ac{GRU} is very often compared to the \ac{LSTM}, it is sometimes assimilated as a type of \ac{LSTM} \cite{nbLSTM}. Their performance was found to be very similar in most situations, making those two types of \acp{RNN} coexistant in the modern machine learning world.

There are two versions of the \ac{GRU}, both are found on the internet, they are known as the encoder and decoder version \cite{gru}. They were originally designed to encode the message to translate and then decode in the translation. PyTorch only supports the decoder version \cite{gruPyTorch}, while the keras library supports both \cite{gruKeras} chosen by changing an argument.

\subsection{Encoder \ac{GRU}}

The encoder \ac{GRU} is the version of the \ac{GRU} most widely described on the internet. It contains an update gate (\cref{eq:updateG}), a reset gate (\cref{eq:resetG}), a candidate activation gate (\cref{eq:candActivG}). The hidden state is then computed (\cref{eq:gruHidG}) using the previous results.

\begin{equation}\label{eq:updateG}
  \overrightarrow{z_t}=\sigma ([\overrightarrow{x_t},\overrightarrow{h_{t-1}}] \cdot w_z + \overrightarrow{b_z})
\end{equation}
\begin{equation}\label{eq:resetG}
  \overrightarrow{r_t}=\sigma ([\overrightarrow{x_t},\overrightarrow{h_{t-1}}] \cdot w_r + \overrightarrow{b_r})
\end{equation}
\begin{equation}\label{eq:candActivG}
  \overrightarrow{\hat{h_t}}=tanh(\overrightarrow{x_t}\cdot w_{hx}+(\overrightarrow{r_t}\odot\overrightarrow{h_{t-1}}) \cdot w_{hh} + \overrightarrow{b_h})
\end{equation}
\begin{equation}\label{eq:gruHidG}
  \overrightarrow{h_t}=(\overrightarrow{1}-\overrightarrow{z_t})\odot \overrightarrow{h_{t-1}} + \overrightarrow{z_t}\odot \overrightarrow{\hat{h_t}}
\end{equation}

Where ($w_z$,$\overrightarrow{b_z}$), ($w_r$,$\overrightarrow{b_r}$),($w_{hx}$,$w_{hh}$,$\overrightarrow{b_h}$) are the weights matrixes and bias vectors for the update, reset and candidate activation gates respectively.

\begin{figure}[H]
  \centering
  \includesvg[width=\textwidth,pretex=\large]{gru/encoderCell.svg}
  \label{fig:lstmCell}
  \caption{Encoder \acs{GRU} cell, legend in \cref{leg:cells}}
\end{figure}

Sometimes the \cref{eq:gruHidG} can be found in another form (\cref{eq:gruHidG1})\cite{gruPyTorch}. This, however, has no impact on the final results, it means the weights are going to be trained differently for the update gate.

\begin{equation}\label{eq:gruHidG1}
  \overrightarrow{h_t}=\overrightarrow{z_t}\odot \overrightarrow{h_{t-1}} + (\overrightarrow{1}-\overrightarrow{z_t})\odot \overrightarrow{\hat{h_t}}
\end{equation}

\subsection{Decoder \ac{GRU}}

The decoder \ac{GRU}, while being less described, is the version used in pyTorch, which is getting very popular.

The candidate activation gate (\cref{eq:candActivG1}) is the only difference from the encoder \ac{GRU}.

\begin{equation}\label{eq:candActivG1}
  \overrightarrow{\hat{h_t}}=tanh(\overrightarrow{x_t}\cdot w_{hx}+ \overrightarrow{b_{hx}}+\overrightarrow{r_t}\odot[\overrightarrow{h_{t-1}} \cdot w_{hh} + \overrightarrow{b_{hh}}])
\end{equation}

\begin{figure}[H]
  \centering
  \includesvg[width=\textwidth,pretex=\large]{gru/decoderCell.svg}
  \label{fig:lstmCell}
  \caption{Encoder \acs{GRU} cell, legend in \cref{leg:cells}}
\end{figure}

\subsection{Similarities with \ac{LSTM}}

The \ac{CIFG} \ac{LSTM} or more commonly known as \ac{GRU}. \cite{gru,gruKeras,gruPyTorch} As it's name implies, \acl{CIFG}, links the input and forget gates as such, $f_t=1-i_t$. Other differences are :
\begin{itemize}
  \item No peepholes and output activation function.
  \item Combined hidden state and cell state.
  \item The candidate cell state is point wise multiplied with the output gate before the activation function.
\end{itemize}

This is clearer with the math equations :
\begin{equation}\label{eq:cellGGRU}
  \tilde{c}_t = tanh(w_{cx}\cdot x_t + w_{ch}\cdot(o_t \odot h_{t-1}) + b_c)
\end{equation}
\begin{equation}\label{eq:hiddenGRU}
  h_t=(1-i_t)\cdot h_{t-1} + i_t \cdot \tilde{c}_t
\end{equation}
