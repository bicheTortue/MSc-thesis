\section{Memristor's model}\label{sec:model}

There are several ways to model a memristor \cite{memristorFab,memTEAMmodel, memVTEAMmodel, memristorSpiceModels}. The way the memristor works depends on the materials used to fabricate the memristor. Each type of memristor has slightly different behavior. Models were thus created to simulate, as accurately as possible, the memristor in electrical circuit simulation software such as \textbf{Cadence}'s \textit{Virtuoso} or SPICE.

Some examples of memristor's model are :

\begin{itemize}
  \item \acf{LIDM}, the model of the first memristor \cite{memristorFab}.
  \item \acf{STBM}, is another model of the $TiO_x$ memristor created in 2008 by \textbf{HP} \cite{memristorFab, memristorSpiceModels}.
  \item \acf{TEAM}, is a model that can easily adapt to several types of memristive devices while focusing on fast computation \cite{memTEAMmodel}.
  \item \acf{VTEAM}, is a later improvement of the \ac{TEAM}. The internal resistance depends on voltage, unlike \ac{TEAM} which depends on current \cite{memVTEAMmodel}.
\end{itemize}

The \ac{VTEAM} being the most versatile model, its implementation would fit most memristor types.

\subsection{Equations}

It has been established that the internal resistance also depends on an internal state $x$ (\cref{eq:memristiveDev}).

\begin{equation}
  \frac{dx}{dt}=f(x,v)
\end{equation}

Where $v$ is the voltage, $t$ is time and $f$ is a function.

The function $f$ is described in its simplest form as in \cref{eq:funcDesc}.

\begin{equation}\label{eq:funcDesc}
  f(x,v)=g(x)\cdot s(v)
\end{equation}

Where $g$ is the window function and $s$ the sensitivity function.

\begin{itemize}
  \item The window function delimits the operational boundaries of the memristor.
  \item The sensitivity function, also known as voltage sensitivity describes the effect of voltage on the internal state's variations.
\end{itemize}

The model is based on previous work \cite{memCadenceModel} that aims to reproduce as many memristive device types. They have chosen to use the resistive state ($R$) as the internal state \cite{memModelOrigin}. Based on this assumption the \cref{eq:memModel0,eq:memModel1} are extracted from the memristive device. This model is of course very basic and ignores some physical dependencies such as temperature.

\begin{equation}\label{eq:memModel0}
  i(R,v)=
  \begin{cases}
    \frac{a_p}{R}\cdot sinh(b_p\cdot v) & v\ge 0\\
    \frac{a_n}{R}\cdot sinh(b_n\cdot v) & v<0\\
  \end{cases}
\end{equation}

\begin{equation}\label{eq:memModel1}
  \frac{dR}{dt}=f(R,v)=s(v)\cdot g(R,v)
\end{equation}

The sensitivity function was chosen to be a voltage-dependent exponential function, like described in \cref{eq:memModel2}.

\begin{equation}\label{eq:memModel2}
  s(v)=
  \begin{cases}
    A_p\cdot (e^{t_p\cdot |v|}-1)& v>0\\
    A_n\cdot (e^{t_n\cdot |v|}-1)& v<0\\
    0 &  v=0\\
  \end{cases}
\end{equation}

The window function used is a state-dependent quadratic function as shown in \cref{eq:memModel3}.

\begin{equation}\label{eq:memModel3}
  g(R,v)=
  \begin{cases}
    (R-r_p(v))^2& v>0\\
    (R-r_n(v))^2& v<0\\
    0 &  v=0\\
  \end{cases}
\end{equation}

Where $r_p$ and $r_n$ are voltage dependent in a polynomial manner. They represent the boundaries of the state variable. All the other parameters in \cref{eq:memModel0,eq:memModel1,eq:memModel2,eq:memModel3} that have not been declared are free fitting variables. They have to be set depending on the type of memristor used.

\subsection{Verilog-A integration}

The model now needs to be adapted to work as Verilog-A code.
Verilog-A is a hardware desciption language, very popular in the industry for its simplicity and flexibility when using it with circuit simulator such as \textbf{Cadence}'s \textit{Virtuoso}.

The focus of this specific implementation will be the resistance range $[20k\Omega, 120k\Omega]$. The boundaries functions are found in \cref{eq:memModel4}.

\begin{equation}\label{eq:memModel4}
  \begin{cases}
    r_p(v)= r_{p,0} + r_{p,1}\cdot v& v>0\\
    r_n(v)= r_{r,0} + r_{n,1}\cdot v& v\le 0\\
  \end{cases}
\end{equation}

Where $r_{p,0}$, $r_{p,1}$, $r_{n,0}$ and $r_{n,1}$ are free fitting variables that need to be extracted from the physical device.

The changes of resistive states from the initial resistance ($R_0$) are computed by \cref{eq:memModel5} extracted from \cite{memCadenceModel}. It uses the bias voltage ($V_b$) of the pulse applied to the memristor. The pulse has a duration of $t$.

\begin{equation}\label{eq:memModel5}
  R(t)_{|V_b} =
  \begin{cases}
    \frac{R_0+s(V_b)\cdot r_p(V_b)\cdot(r_p(V_b)-R_0)\cdot t}{1+s(V_b)\cdot (r_p(V_b)-R_0)\cdot t} & V_b>0, R<r_p(V_b)\\
    \frac{R_0+s(V_b)\cdot r_n(V_b)\cdot(r_n(V_b)-R_0)\cdot t}{1+s(V_b)\cdot (r_n(V_b)-R_0)\cdot t} & V_b\le 0, R>r_n(V_b)\\
    R_0 & else\\
  \end{cases}
\end{equation}

The final Verilog-A code is available in \cite{memCadenceModel}.

\subsection{Behavior}\label{subsec:memModelBehave}

The model implements, as with physical memristors, a threshold voltage value from which its resistive states starts changing. This threshold value is $V_{read}$. For this specific model the threshold voltage is set to $V_{read}=500mV$.

Every pulse is composed of 2 parts and lasts for a total of $1 ms$ :
\begin{itemize}
  \item The reading of the resistive state, by applying a triangular wave that is bound by the threshold value $V_{read}$. The triangular wave starts from $0V$, goes to $V_{read}$ then to $-V_{read}$ before going back to $0V$.
  \item The writing to the memristor. During this part the resistive state will change. This is done by applying a square wave of variable width ($t_{\Delta}$) with a set bias voltage ($V_{bias}$).
\end{itemize}

\begin{figure}[H]
  \centering
  \includesvg[width=\textwidth]{memristor/input.svg}
  \caption{Pulse shape graph}
  \label{graph:memInput}
\end{figure}

The \cref{graph:memInput} is the visual representation of what each pulse, the value is read during the first half of the pulse.

The resistive state of the memristor changes depending on the pulse width ($t_{\Delta}$) and the bias voltage ($V_{bias}$) used for each pulse.

\begin{figure}[H]
  \centering
  \includesvg[width=\textwidth]{memristor/memPulseChange.svg}
  \caption{Memristor's resistive state under different pulses width (Fixed bias voltage $V_{bias}=1.8V$)}
  \label{graph:pulseChange}
\end{figure}

The effect of the pulse width variation is shown in \cref{graph:pulseChange}. Indeed, the larger the pulse widths, the faster the internal resistance reaches the bounding resistance (\cref{eq:memModel4}). The bounding resistance being the same as all curves use the same bias voltage ($V_{bias}=1.8V$), the curves will all eventually reach the max resistance.

\begin{figure}[H]
  \centering
  \includesvg[width=\textwidth]{memristor/memVoltChange.svg}
  \caption{Memristor's resistive state under different voltages (Fixed pulse width $t_{\Delta}=100\mu s$)}
  \label{graph:voltChange}
\end{figure}

The voltage dependence of the memristor's model is described in \cref{graph:voltChange}. The influence of the bias voltage is, as \cref{eq:memModel3,eq:memModel4} show, the way to change the resistance range of the memristor, until the physical limit is reached of course. Applying higher voltage also allows to reach a set resistance faster as the slopes are bigger the higher the bias voltage gets. The resistive state can also be lowered by applying negative voltages.%TODO

%\subsection{\acf{LIDM}}

%The \ac{LIDM} is the model of the first memristor eveer created, it was created to fit the behavior of the $TiO_x$ memristor they had just fabricated \cite{memristorFab}. The device, of width $D$, has two parts. The doped part and the undoped
