\section{Memristors}
\label{sec:memristors}

Memristors are the lesser known fourth fundamental passive component of electronics, among resistors, capacitors and inductor.
It was first theorized in 1971 by L. Chua from UC Berkeley, in \cite{TheoMemristor}. The name comes from the blend of \textit{memory} and \textit{resistance}.
The theory behind this component was extracted from a missing component to link the four fundamental circuit variables, voltage (\symv), charge (\symq), current (\symq) and flux (\symphi). \Cref{fig:fundComp} shows the four fundamental variables are on each side of the square, with the ones on opposite sides being linked by the following equations :

\begin{equation}
  d\symphi = \symv\cdot d\symt
\end{equation}

\begin{equation}
  d\symq = \symi\cdot d\symt
\end{equation}

Resistors, capacitors and inductors were already very established and well known components, so it was theorized that a fourth device should then exist to physically link flux (\symphi) and charge (\symq).  The flux in this case is not a magnetic flux and is defined as such : $d\symphi=\symv\cdot d\symt \Rightarrow \symphi =  \int \symv \,d\symt$.

The component stayed theoretical until 2008 when it was implemented in a physical device for the first time \cite{memristorFab}. It took 37 years to have an actual working device.

There is then an extension of this theoretical device to another, the memristive device. It was theorized in 1976 by L. Chua and S. M. Kang \cite{memrestiveDev}. The difference between the memristor and the memristive devices is its internal behavior. Memristive device are commonly referred to as memristors as well.

\begin{figure}[H]
  \centering
  \includesvg[width=0.55\textwidth]{memristor/memristor}
  \caption{Fundamental passive components, adapted from \cite{memWiki}}
  \label{fig:fundComp}
\end{figure}

\subsection{Equations}
A memristor links the flux (\symphi) and charge (\symq) by the memristance.
This memristance is defined with the following equation :
\begin{equation}
  \symM(\symq)=\frac{d\symphi}{d\symq}
\end{equation}
It can be compared with the other fundamental components like resistor ($\symR(\symi)=\frac{d\symv}{d\symi}$), capacitor ($\frac{1}{\symC(\symq)}=\frac{d\symv}{d\symq}$) and inductor ($\symL(\symi)=\frac{d\symphi}{d\symi}$).
We can then extract a more useful equation in an actual circuit :
\begin{equation}
  \symv(\symt)=\symM(\symq(\symt))\cdot \symi(\symt)
\end{equation}
Similarly, a memductance can be defined as such :
\begin{equation}
  \symW(\symphi)=\frac{d\symq}{d\symphi}
\end{equation}
A memristive device is slightly differently defined, it still uses a memristance, but here the memristance also depends on an internal state called $x$. This gives us this equation :
\begin{equation}\label{eq:memristiveDev}
  \symv(\symt)=\symM(x,\symi)\cdot \symi(\symt)
\end{equation}
The internal state ($x$) is not linked to flux or charge in the case of a memristive device.\\
Once again, we can also define the memristive device using a memductance :
\begin{equation}
  \symi(\symt)=\symW(x,\symv)\cdot \symv(\symt)
\end{equation}
In all of the previous equations, \symv is the voltage in Volt ($V$), \symi is the current in Ampere ($A$), \symphi is the flux in Weber ($Wb$), \symq is the charge in Coulomb ($C$), \symM is the memristance in Ohm ($\Omega$) and \symW is the memductance in Siemens ($S$ or $\Omega^{-1}$).

\subsection{Behavior}

A memristor is defined as a non-linear two-terminal fundamental electrical component. It behaves as a resistance with memory (hence its name), meaning that it changes its resistance based on how much charge went through it. This enables us to manipulate the resistance of the component.
The huge benefit of memristors is the ability to retain its internal resistance, the device can be left without power for a long period of time (retention time of minimum 10 years according to \cite{memRetention}). When the device is powered backup, it will have the same resistance it had before.

Memristive devices have a similar behavior, the memristive device's resistance will change depending on the internal state ($x$). That internal state changes based on how much and how long voltage signals or currents are applied to the memristive device.

\subsection{Usage}
The main research for memristor usage is using them as ReRAM. The idea behind ReRAM is to use memristors as Non-Volatile memory. It uses two states of the device with known resistances ($R_{on}$ and $R_{off}$), giving it binary property. Reading the memory simply requires setting a voltage and reading the output current. It is better than current solutions (HDD, SSD) as it has a much lower latency. It is better than traditionnal DRAM because it keeps the information even when turned off. This makes ReRAM a good replacement for both RAM and HDDs/SSDs, thus eliminating von Neumann bottleneck due to the von Neumann architecture that is used in all modern computers.

They can also be used to set a resistance to be able to perform analog multiplication. Setting them in a crossbar array makes them a very strong candidate to be used in neuromorphic computing.
