\section{Environmental and societal impact}

\subsection{Personnal environmental impact}

\subsection{Global impact}

The subject of this thesis being purely theoretical, running simulations only, the environmental and societal impact will focus on the system if it were to become real. This assumes that the entire system is working and is being fabricated for embedded use. The final product can be used in both comsumer and business oriented product.

\subsubsection{Environmental impact}

The system would be found in the form of a chip, they would need to be fabricated using a very precise method. Chip fabrication does not come cheap for the environment. Indeed, the manufacturing process of computer chips, while integral to technological advancements, poses significant environmental challenges. Chip manufacturing involves several stages, including silicon wafer production, lithography, etching, and assembly. These processes consume vast amounts of energy and water while generating substantial waste. Semiconductor fabrication demands clean rooms with controlled environments, necessitating extensive energy consumption for air conditioning and filtration systems. Additionally, the use of hazardous chemicals, such as solvents and acids, in chip etching and cleaning stages contributes to chemical waste and potential environmental pollution if not managed and disposed of properly. Moreover, the disposal of silicon scraps and other manufacturing byproducts requires careful handling to prevent environmental contamination. The machines used to fabricate the chip are not taken into account here as their impact is negligible next to the fabrication itself. For example, TSMC's electricity consumption in 2021 accounted for 5 \% of Taiwan total electricity consumption \cite{chipEnv}. Furthermore, in a report from greenpeace, the projected emissions in 2030 of the chip manufacturing industry is of 86 million metric tons of carbon dioxide equivalent, which is more than the total emissions of Portugal over 2021 and 2022 \cite{greenReport,worldEmissions}.

When the product reaches end of life, it needs to be disposed of properly to avoid soil and water contamination due to the release of hazardous substances. Recycling computer chips is a critical facet of sustainable electronic waste management, aiming to mitigate the environmental repercussions of discarded electronic devices. The recycling process typically involves several stages, including collection, sorting, dismantling, and extraction of valuable materials like silicon, copper, gold, and other metals. However, chip recycling poses challenges due to the complex composition of electronic components and the intricate structures of integrated circuits. These challenges often necessitate advanced techniques such as shredding, mechanical separation, and chemical treatments to extract and purify the reusable materials. Although chip recycling reduces the need for raw materials and minimizes landfill accumulation, the process itself demands significant energy and resources. Therefore, while chip recycling offers a sustainable approach to managing electronic waste, continuous advancements in recycling technologies and effective waste management policies are imperative to diminish the environmental impact associated with both chip manufacturing and disposal.

The potential chip that would be created has been analyzed, there are also other aspects to take into account. The system being designed to run \acp{NN}, the training of the \ac{NN} also consumes energy

Subtantial gain in power consumption.

\subsubsection{Societal impact}

\subsection{Host structure's policy}
