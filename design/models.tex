\section{Verilog models}
\label{sec:models}

Due to a lack of time, some of the more common component were not designed by me but instead were simulated using a Verilog model. Verilog is a hardware description laguage very popular in the industry. The only components that use a verilog model are the voltage multiplier and the \ac{opAmp}.

\subsection{Operational amplifier}\label{subsec:opamp}

This component is the very famous \ac{opAmp}. This specific component wasn't designed for the thesis, it required a current range that didn't allow to use ones already made by members of the research group. The choice was then to use a verilog-A model, and then design one if time allows it.

\begin{equation}
  \label{eq:opAmp}
  V_{out}=\mu \cdot (V_+-V_-)
\end{equation}

\Cref{eq:opAmp} is the equation used in the verilog-A model. It makes it so it behaves like an ideal \ac{opAmp}. For this thesis $\mu$ has been set to $\mu=10^5$.

\subsection{Voltage multiplier}\label{subsec:voltmult}

This component while far less popular than the latter, is just as useful for our specific use. It allows us to multiply, as its name implies, two voltages. It is used to compute the pointwise multiplications of the \ac{LSTM} (\cref{fig:lstmCell}).

However this part is tricky. Indeed, because of the voltage conversion (\cref{tab:valConv}), the equation needs to take that into consideration. The final equation needs to be \cref{eq:finalVoltMult} in order to have $1\cdot 1=1$ ($V_{in_1}\cdot V_{in_2}=1V$, with $V_{in_1}=V_{in_2}=1V$)

\begin{equation}\label{eq:finalVoltMult}
  V_{out}=10\cdot(V_{in_1}-V_{cm})\cdot (V_{in_2}-V_{cm}) + V_{cm}
\end{equation}

This is taken care of using in reality two parts, the actual voltage multiplier (\cref{eq:voltMult}) and a non inverting amplifier (\cref{eq:invAmp}), the circuit of which is available in \cref{apsec:invAmp}.

\begin{equation}\label{eq:voltMult}
  V_{voltMult}=-(V_{in_1}-V_{cm})\cdot (V_{in_2}-V_{cm}) + V_{cm}
\end{equation}
\begin{equation}\label{eq:invAmp}
  V_{out}=-(V_{voltMult}-V_{cm})\cdot10+V_{cm}
\end{equation}

Where $V_{out}$ is the output voltage the inverting amplifier (\cref{apsec:invAmp}), $V_{voltMult}$ is the out voltage of the voltage multiplier itself and $V_{in_1}$ and $V_{in_2}$ are the input voltages.

The \cref{eq:voltMult} is assumed possible because of the actual voltage multiplier's datasheet available at \cite{actualVoltMult}.

\subsection{Symbols}

The symbols for those two components are the ones in \cref{sym:models}.

\begin{figure}[H]
  \centering
  \hspace*{2cm}
  \subfloat[\ac{opAmp}'s symbol\label{sym:opAmp}]{\includesvg[height=2.5cm]{models/opAmpSymbol}}%
  \hfill
  \subfloat[Voltage multiplier's symbol\label{sym:voltMult}]{\includesvg[height=2.5cm]{models/voltMultSymbol}}%
  \hspace*{2cm}
  \caption{Symbols used for the verilog-A components}
  \label{sym:models}
\end{figure}

The \ac{opAmp} uses its classic symbol (\cref{sym:opAmp}) while the voltage multiplier uses a custom symbol (\cref{sym:voltMult}).
