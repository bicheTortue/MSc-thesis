\section{The datasets}\label{sec:dataset}

To make sure that the terminal voltage doesn't reach the threshold as explained in \cref{sec:xbarCircuit}. The voltage threshold ($V_{threshold}$) was chosen to be $V_{threshold}=0.1V$ because the memristors used in this thesis are purely theorical. The dataset needs to be formated not to exceed this value when converted to voltage (\cref{tab:valConv}), the dataset needs to be trained while not exceeding $1$.

\subsection{Airline}

The first dataset contains a time series of international airline passengers from January 1949 to December 1960, the data is recorded monthly and is in thousands. The dataset then contains twelve years of monthly data so $144$ sample points. The results of such a problem aren't useful, it was used in this thesis because of its simplicity to check whether the analog circuit is working. It is extracted from the tutorial available at \cite{airline}. The dataset is available at \cite{datasets}, the name of the file containing the dataset is \textit{airline-passengers.csv}.

This is a regression problem, the role of the \ac{LSTM} is to predict the number of passenger flying the next month being given previous months' passenger count.

\begin{figure}[H]
  \centering
  \includesvg[width=\textwidth]{datasets/airline}
  \caption{The airline dataset. The vertical lines represent a full year.}
  \label{graph:airline}
\end{figure}

\subsubsection{Data format}

The dataset contains $144$ sample points. It has been transformed into $142$ data points for training. This has been done by taking each values three by three. The first two values are two timesteps for the input vector and the third value being used the target value. Two third of the dataset is being used for training and the other third is used for validation.

\subsection{\ac{C. elegans}}

This dataset is far more interresting than the latter. This data set aims to use \acp{LSTM} to mimic the behavior of real neurons. As explained in \cite{celegans}, \acf{C. elegans} are simple organisms that are getting very popular for whole brain organization studies. The point of this problem is to reproduce the nervous system of the \ac{C. elegans}. This is done using recorded data of the input of 4 neurons and the output of 4 other neurons.

The dataset is great for our study because :
\begin{itemize}
  \item It comes from a very recent paper, and means the research is current.
  \item It aims at reproducing the behavior of the brain of a simple organism (\ac{C. elegans}). In the (very) long term, a full parts of the human brain could be replaced by a very low powered chip.
  \item The paper was written by Dr. Barbulescu who happens to be one of my supervisor.
\end{itemize}

\begin{figure}[H]
  \centering
  \begin{minipage}{\columnwidth}
    \subfloat[Training dataset sample\label{graph:io5Celegans}]{\includesvg[width=0.5\textwidth, pretex=\tiny]{datasets/celegans/ioTrain0}}%
    \hfill
    \subfloat[Test dataset sample]{\includesvg[width=0.5\textwidth,pretex=\tiny]{datasets/celegans/ioTest0}}%
  \end{minipage}
  \begin{minipage}{\columnwidth}
    \subfloat[Training dataset sample]{\includesvg[width=0.5\textwidth, pretex=\tiny]{datasets/celegans/ioTrain1.svg}}%
    \hfill
    \subfloat[Validation dataset sample]{\includesvg[width=0.5\textwidth,pretex=\tiny]{datasets/celegans/ioValid0}}%
  \end{minipage}
  \caption{\ac{C. elegans} dataset samples}
  \label{graph:celegans}
\end{figure}

\subsubsection{Dataset format}

The dataset contains a set of fourty set of input/output. Each input/output set contains $500ms$ of data with a time step of $0.5ms$, making a thousand points for each neurons. This data is recorded after applying current to the input neurons (PLML2, PLMR, AVBL, AVBR), and monitoring the output of four neurons (DB1, LUAL, PVR, VB1) that are known to have strong activity during  a specific behavior of the nematode known as \ac{FCM}. The simulation is done using a known model of \ac{C. elegans} connectome (complete overview of the brains connections) \cite{celegans}.

\Cref{graph:celegans} shows four sets of input/output our of the fourty that were recorded for the dataset.

The inputs for the systems are then going to be the current values for the four input neurons at every of the thousand time steps and plans to get an output as close as possible to the modeled output for all time steps.
