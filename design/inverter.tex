\section{Inverter}
\label{sec:inv}

This section simply describes a very well know electrical component, the inverter. When it a voltage of $V_{dd}$ is applied to the input, the output is grounded and the same goes for the other way around.

\begin{figure}[H]
  \centering
  \hspace*{2.5cm}
  \subfloat[Inverter's circuit]{\label{circt:inv}\includesvg[height=3cm]{inverter/invCircuit}}%
  \hfill
  \subfloat[Inverter's symbol]{\label{sym:inv}\includesvg[height=2.5cm]{inverter/invSymbol}}%
  \hspace*{1.5cm}
  \caption{}
  \label{fig:inv}
\end{figure}

Like mentionned, the circuit for the inverter is very straight forward (\cref{circt:inv}). The symbol is nothing fancy as well as it is the classic symbol (\cref{sym:inv}) for such a circuit.

The approximate onChip area for the inverter is :
\begin{equation}
  A_{inv}=2\cdot A_{CMOS}
\end{equation}

With $A_{CMOS}= w\cdot l=2\cdot 10^{-7}\cdot6\cdot10^{-8} =1.2\cdot10^{-14}m^2$.

The onChip area of the inverter is then $A_{inv}=2.4\cdot10^{-14}m^2$.
