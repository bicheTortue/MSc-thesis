% %%%%%%%%%%%%%%%%%%%%%%%%%%%%%%%%%%%%%%%%%%%%%%%%%%%%%%%%%%%%%%%%%%%%%%
% Dummy Chapter:
% %%%%%%%%%%%%%%%%%%%%%%%%%%%%%%%%%%%%%%%%%%%%%%%%%%%%%%%%%%%%%%%%%%%%%%

% %%%%%%%%%%%%%%%%%%%%%%%%%%%%%%%%%%%%%%%%%%%%%%%%%%%%%%%%%%%%%%%%%%%%%%
% The Introduction:
% %%%%%%%%%%%%%%%%%%%%%%%%%%%%%%%%%%%%%%%%%%%%%%%%%%%%%%%%%%%%%%%%%%%%%%
\fancychapter{The design}
\label{cap:design}

In this chapter, I'm going to describe the different design decisions taken during the study of the thesis.

The system will be working with a $V_{dd}$ of $1.8V$. Such a value was chosen because this is a low power system.

The way values are encoded in the analog system will be descibed here as it serves for the entire thesis.
In order for the system to support negative numbers we're going to use a $V_{cm}$ set to $\frac{V_{dd}}{2}$. That means that $V_{cm}=\frac{V_{dd}}{2}=0.9V$. This $V_{cm}$ will then describe a zero. A step of one was chosen to be $0.1V$ in the analog circuit.
\Cref{tab:valConv} shows the conversions from a real number to it's voltage equivalent.

\begin{table}[H]
  \centering
  \begin{tabular}{|c|c|}
    \hline
    \rowcolor{gray}
    Real value & Voltage \\
    \hline
    $0$ & $0.9V$ \\
    \hline
    $1$ & $1.0V$ \\
    \hline
    $x$ & $V(x)=\frac{x}{10}+V_{cm}$\\
    \hline
    $real(v)=(v-0.9)\cdot 10$ & $v$\\
    \hline
  \end{tabular}
  \caption{Real/Voltage Conversion Table.}
  \label{tab:valConv}
\end{table}

Since the system cannot reach voltage outside of the operating range with the intended behavior, the voltage is then restricted to $V\in [0,1.8]$. This means that the range of real value that the systems can handle is $x\in [-9,9]$.

Any data inputed in the analog system should have been previously checked to make sure it stays in the range of accepted values.

In order to create a fully functional \ac{LSTM} or \ac{GRU} circuit, there are a few components that are required.

An analog activation function circuit to have an analog equivalent to the activation function necessary to the good behavior of \acp{LSTM} and \acp{GRU}.

A memory cell to allow to store the analog value from one time step to the next.

A circuit for the crossbar array, to have a more general circuit that can easily be adapted.

Other, less significant, components required for the circuits to work.

\section{The activation functions}
\label{sec:af}

Activation functions play a great role in the results obtained \cite{af}. It is the reason why making good Activation function circuits is fundamental. Of course the easy way would be to simply design a hard sigmoid (\cref{apsec:hardFunc}), the issue being that the hard sigmoid is much worse than the regular sigmoid, especially for regression problems \cite{hardSigm}. The same goes for the \ac{tanh} and hard \ac{tanh} functions.

Designing an analog activation function as close to the original is very important for the final result's quality.

\begin{table}[H]
  \centering
  \begin{tabular}{|c|c|c|}
    \hline
    \rowcolor{gray}
    Parameter & Sigmoid & \ac{tanh} \\
    \hline
    $V_1$ & \multicolumn{2}{c|}{$1.1V$}\\
    \hline
    $V_2$ & \multicolumn{2}{c|}{$635mV$}\\
    \hline
    $V_3$ & $0.8V$ & $550mV$\\
    \hline
    $i_{dc}$ & \multicolumn{2}{c|}{$150uA$}\\
    \hline
    $w$ & \multicolumn{2}{c|}{$900nm$}\\
    \hline
    $l$ & \multicolumn{2}{c|}{$60nm$}\\
    \hline
    $R_1$ & \multicolumn{2}{c|}{$5k\Omega$}\\
    \hline
    $R_2$ & \multicolumn{2}{c|}{$10k\Omega$}\\
    \hline
    $R_3$ & $2k\Omega$ & $4k\Omega$\\
    \hline
  \end{tabular}
  \caption{Circuits parameters}
  \label{tab:afPar}
\end{table}

Here, $w$ and $l$ are, respectively, the width and length of the two NMOS of the circuit.

\subsection{Circuit}

The circuit used is the same as the one in \cite{thesisRef}, the circuit is the one shown in \cref{circt:af}. The technology used being different, all the parameters had to be determined empirically to best fit a sigmoid shape. The parameters can be found in \cref{tab:afPar}.

Due to the nature of the functions we want to generate, we will use the same circuit for both a sigmoid and a \ac{tanh} like functions. The two different functions are generated by changing two parameters.

The functions generated are the same shape and only differ by their output range.

\begin{figure}[H]
  \centering
  \includesvg[width=\textwidth]{activation/afCircuit}
  \caption{Activation functions circuit}
  \label{circt:af}
\end{figure}

\subsection{Symbols}
The symbols for the sigmoid and the \ac{tanh} are separated for better understanding.

\begin{figure}[H]
  \centering
  \hspace*{0.8cm}
  \subfloat[Sigmoid symbol]{\includesvg[height=2.5cm]{activation/sigmoidSymbol}}%
  \hfill
  \subfloat[\ac{tanh} symbol]{\includesvg[height=2.5cm]{activation/tanhSymbol}}%
  \hspace*{0.8cm}
  \caption{Activation functions symbols with the input and output pins on either side depending on the flow of the current for better readability}
  \label{fig:afSymbol}
\end{figure}

The approximate onChip area of this circuit is :

\begin{equation}
  A_{af}=2\cdot A_{R_1} + 2\cdot A_{CMOS} + 5\cdot A_{R_2} + A_{R_3} +2\cdot A_{opAmp}
\end{equation}
With $A_{CMOS} = w\cdot l = 9\cdot 10^{-7} \cdot 6 \cdot 10^{-8} = 5.4 \cdot 10^{-14} m^2$
TODO : Finish calc

In order to combine the the onChip area of both the sigmoid and \ac{tanh} activation functions, it is assumed that the variation of area taken by the resistor ($R_3$) is negligible compared to th overall area.

\subsection{Usage}

This circuit outputs a voltage that depends on the input voltage passed on. The relation between the two is shown in \cref{fig:afGraph}. The graph also shows the \ac{RMSE} of each graph compared to the ideal result.

Note that the actual output of the circuit in \cref{circt:af} is inverted arround $V_{cm}$ a simple circuit such as the one shown in  (TODO show circuit in annex) does inverts the

\begin{figure}[H]
  \centering
  \includesvg[width=\textwidth]{activation/afGraph}
  \caption{Input/Output graph of the activation function circuit for both sigmoid and \acs{tanh} functions}
  \label{fig:afGraph}
\end{figure}

These functions are still a bit different from the original functions (especially for the \ac{tanh}). However that doesn't matter too much as the trainning will be happening within the final circuit, all weights will be set in the circuit. This is the reason why such a difference doesn't matter. As long as the curves have the similar shape, the result won't be drastically affected.

\section{Memory cells}
\label{sec:memcell}

The memory cell is a circuit that is able to store an analog value for a limited time. It works using capacitors that have the ability to store a voltage for a short period of time.

\subsection{Circuit}

The circuit is shown in \cref{circt:memcell}. The value/voltage is trapped in the capacitor using \ac{CMOS} switches.

\begin{figure}[H]
  \centering
  \includesvg[width=\textwidth]{memcell/memCircuit}
  \caption{Memory cell circuit}
  \label{circt:memcell}
\end{figure}

The cirucit has a two \ac{CMOS} switches design to avoid voltage leakage through the swicthes. The system has voltage leakage when only one \ac{CMOS} swicth, and thus leads to a large memory leak. This is due to the high voltage difference between the two sides of the \ac{CMOS} swicth. Using two \ac{CMOS} switches allows for this difference to be mitigated (\cref{fig:memcellLoss}).

\begin{figure}[H]
  \centering
  \includesvg[width=\textwidth]{memcell/data-loss}
  \caption{Memory conservation in a memory cell with 1 \ac{CMOS} switch vs 2 \ac{CMOS} swicthes}
  \label{fig:memcellLoss}
\end{figure}

\subsection{Symbol}

The symbol for this circuit is designed to show a capacitor because it is its memory mechanism.

\begin{figure}[H]
  \centering
  \includesvg[height=2.5cm]{memcell/memcellSymbol}
  \caption{Memory cell symbol with the input enable pin (top) and the output enable pin (bottom). The left and right pins are interchangebly the input and ouput for the sake of readability}
  \label{circt:memcell}
\end{figure}

For this circuit the approximate onChip area is :

\begin{equation}
  A_{memcell}=8\cdot A_{CMOS}+2\cdot A_{inv}+A_{opAmp}+A_{cap}
\end{equation}

With $A_{CMOS} = w\cdot l = 2\cdot 10^{-7} \cdot 6 \cdot 10^{-8} = 1.2 \cdot 10^{-14} m^2$

\subsection{Usage}

This circuit is pretty straight forward, when the input enable pin is high (up to $V_{dd}$), then the first switch is open and the capacitor is being charged, if it is low, then the switch is closed and the capacitor keeps the voltage and thus the analog data. When the output enable is high, the buffer (\cref{apsec:buffer}) forwards the voltage stored in the capactor to the output pin without emptying the capacitor.

\begin{figure}[H]
  \centering
  \begin{tikztimingtable}
    Input data & x6D{$x_1$}6D{$x_2$}6D{$x_3$}x \\
    Input enable & xL2H2L3H7L2HLx\\
    Output enable & x3L2H4L5H1L3Hx\\
    Memory & xU4D{$x_1$}10D{$x_2$}3D{$x_3$}x \\
    Output data & x3U2D{$x_1$}4U5D{$x_2$}U3D{$x_3$}x \\
    \extracode
    \tablerules
    %\draw (0,0) circle (2pt); % Origin
    \begin{pgfonlayer}{background}
      \vertlines[help lines]{0.6,18.6}
      %\vertlines[red]{1.6,5.6,15.6}
      %\vertlines[blue]{3.6,9.6,15.6}
    \end{pgfonlayer}
  \end{tikztimingtable}
  \caption{Time diagram that shows the logic of the memory cell}
  \label{tim:memcell}
\end{figure}

\section{Voltage-driven crossbar circuit}
\label{sec:xbarCircuit}

The crossbar circuit theory has already been explained in the \cref{sec:crossbar}. This section describes how the crossbar circuit is actually implemented. The final circuit is the one in \cref{circt:xbar}. The circuit depends on three parameters :

\begin{itemize}
  \item $n_i$ : The number of input for our crossbar array (not including bias for a more general circuit).
  \item $n_o$ : The number of parallel output for our crossbar array.
  \item $n_s$ : The serial size of our crossbar system.
\end{itemize}

\begin{figure}[H]
  \centering
  \includesvg[width=\textwidth]{crossbar/crossbarUse}
  \caption{Circuit of the crossbar array used in the final system ($n_i$, $n_o$, $n_s$)}
  \label{circt:xbar}
\end{figure}

The parallel and serial are explained later (\cref{subsec:serpar}).

The enable flags ($e_j,\forall j\in[\![ 0,n_s]\!]$ and similarly $\overline{e_j}$) are there to show when the \ac{CMOS} switches are open, the states of those flags is shown in \cref{tim:serpar}.

\subsection{Two memristors per synapse}
\label{subsec:doubleMem}

We could choose between one or two memristor per synapse. Using two memristor per synapse doubles the area but doubles the weight range (\cref{tab:synapses}) and allows to easily use negavite weights. The output voltage will be centered around $V_{cm}$ and be compliant with the standard set in \cref{tab:valConv}.

This design is the one that is used in \cite{doubleMem}. Let's assume that a given memristor has a resistance range of $R\in[R_{min},R_{max}]$, that means its conductance range is $ G \in [ G_{min}, G_{max}]$ (with $ G_{min}= \frac{1}{R_{max}}$ and $ G_{max}= \frac{1}{R_{min}}$). This design works using two \ac{opAmp} connected to $V_{cm}$ with the positive pin and the negative pin to the output of the crossbar array. \Cref{eq:doubleMem0,eq:doubleMem1,eq:doubleMem2} are describing how this architecture works. A simplified version of the double memristors per synapse circuit is also available in \cref{circt:doubleMem}.

\begin{figure}[H]
  \centering
  \includesvg[width=\textwidth,pretex=\scriptsize]{crossbar/doubleMem}
  \caption{Simplified circuit of a double memristor per synapse architecture}
  \label{circt:doubleMem}
\end{figure}

Using $x_k$ as the voltage for the input line $k$. The highest \ac{opAmp} is identified as $opAmp_0$ and the lowest $opAmp_1$.

For the sake of simplicity, the following equations considers the ground to be $V_{cm}$.

\begin{equation}
  \label{eq:doubleMem0}
  V_{opAmp_0}=-R_r\cdot i_+
\end{equation}
\begin{equation}
  \label{eq:doubleMem1}
  i_{R_f}=i_-+\frac{V_{opAmp_0}}{R_r}=i_--i_+
\end{equation}
\begin{equation}
  \label{eq:doubleMem2}
  V_{opAmp_1}=y_0=-R_f\cdot(i_--i_+)=R_f\cdot(i_+-i_-)=R_f\cdot\sum_{k=0}^n( G_{k+}- G_{k-})\cdot x_k
\end{equation}
With $i_+=\sum_{k=0}^n G_{k+}\cdot x_k$ and $i_-=\sum_{k=0}^n G_{k-}\cdot x_k$.


\begin{table}[H]
  \centering
  \begin{tabular}{|c|c|c|}
    \cline{2-3}
    \rowcolor{gray}
    \multicolumn{1}{c|}{\cellcolor[HTML]{FFFFFF}} & Two memristors per synapse & One memristor per synapse \\
    \hline
    Maximum weight & $ G_{max}- G_{min}$ & $ G_{max} -\overline{ G}$\\
    \hline
    Minimum weight & $ G_{min}- G_{max}$ & $ G_{min} -\overline{ G}$\\
    \hline
    Range & $2\cdot( G_{max}- G_{min})$&$ G_{max}- G_{min}$\\
    \hline
  \end{tabular}
  \caption{Synaptic weights precision (extracted from \cite{doubleMem})}
  \label{tab:synapses}
\end{table}

\subsection{Serialization}
\label{subsec:serpar}

This circuit has the option to be serialized with varying degrees. The idea of serializing the circuit came from \cite{thesisRef}. Serializing the circuit reduces the number of components required and thus reduces the final onChip area. Serializing the system increases the time it takes to compute the output.

Serializing the system means not computing all values of the output vector at the same time, but instead computing group by group, the groups' size are $n_o$. The first output group is computed during $e_0$ and the $i^{th}$ group is computed during $e_i$. The timing of when the outputs are available is found in \cref{tim:serpar}. Those flag control the \ac{CMOS} swicthes present in \cref{circt:xbar}, the switches control which output group is outputed.

The \ac{CMOS} switches are here to open the necessary input gates when the output is required, the \ac{CMOS} switches are controlled as in \cref{tim:serpar}.

\begin{figure}[H]
  \centering
  \begin{tikztimingtable}
    $e_0$ & x2H2L N(A1) L N(A2) 2L N(A3) L N(A4) 2Lx\\
    $e_1$ & x2L2H N(B1) L N(B2) 2L N(B3) L N(B4) 2Lx\\
    $e_i$ & x4L N(C1) L N(C2) 2H N(C3) L N(C4) 2Lx\\
    $e_{n_s-1}$ & x4L N(D1) L N(D2) 2L N(D3) L N(D4) 2Hx\\
    %foo & 2L N(A1)  4H N(A2) L\\
    \extracode
    \node[gap, at={($(A1|-A2)!0.5!(A2)$)}];
    \node[gap, at={($(A3|-A4)!0.5!(A4)$)}];
    \node[gap, at={($(B1|-B2)!0.5!(B2)$)}];
    \node[gap, at={($(B3|-B4)!0.5!(B4)$)}];
    \node[gap, at={($(C1|-C2)!0.5!(C2)$)}];
    \node[gap, at={($(C3|-C4)!0.5!(C4)$)}];
    \node[gap, at={($(D1|-D2)!0.5!(D2)$)}];
    \node[gap, at={($(D3|-D4)!0.5!(D4)$)}];
    \tablerules
    %\draw (0,0) circle (2pt); % Origin
    \begin{pgfonlayer}{background}
      \vertlines[help lines]{0.55,10.55}
      %\vertlines[red]{1.6,5.6,15.6}
      %\vertlines[blue]{3.6,9.6,15.6}
    \end{pgfonlayer}
  \end{tikztimingtable}
  \caption{Enable flags timing for any value of $n_s$ in a single time step}
  \label{tim:serpar}
\end{figure}

When the system is used fully in parallel, the \ac{CMOS} switches are not required can then be removed to lower the final onChip area.

%TODO understand better \subsection{Sneak path problem}
\subsection{Symbol}
The symbol (\cref{sym:xbar}) defined for the voltage based memristor crossbar array used in this thesis is more compact and helps the readability of the circuits that require a crossbar array. It depends on several parameters, the number of inputs ($n_i$), the number of outputs ($n_o$) and the serial size ($n_s$).

\begin{figure}[H]
  \centering
  \includesvg[height=2.5cm]{crossbar/xbarSymbol}
  \caption{Symbol used for the crossbar array, the input pin is a bus of size $n_i$ and the output pin is a bus of size $n_o$}
  \label{sym:xbar}
\end{figure}

The approximate onChip area for the crossbar circuit depends on the previously defined parameters.

\begin{equation}
  A_{xbar}(n_i,n_o,n_s)=
  \begin{cases}
    2\cdot n_i\cdot n_o \cdot A_{memristor}+n_o\cdot(2\cdot[A_{opAmp}+A_{R_r}]+A_{R_f})& n_s=1\\
    \begin{array}{2}
      2\cdot n_i\cdot n_o \cdot n_s\cdot A_{memristor}+2\cdot A_{CMOS}\cdot n_o\cdot n_s\\
      +n_o\cdot(2\cdot[A_{opAmp}+A_{R_r}]+A_{R_f})
    \end{array}
    & \forall n_s>1\\
  \end{cases}
\end{equation}

With $A_{CMOS}=w\cdot l$.

\subsection{Usage}

\subsubsection{Input voltages}

The input voltages must be monitored very carefuly in order not to accidentally change a memristor's internal resistance. Indeed, memristors have a voltage threshold ($V_{read}$) above which the internal resistance changes \cite{memristorSpiceModels,memCadenceModel,memTEAMmodel,memVTEAMmodel}. This means that the input voltages must follow the inequality in \cref{eq:memThres}.

\begin{equation}\label{eq:memThres}
  |V_{mem}|= |x_i-V_{cm}|\le V_{read}
\end{equation}

Where $V_{mem}$ is the terminal voltage of a memristor.

\subsubsection{Dense layer}

A dense layer is simply a \ac{VMM}. This means that a dense layer in a \ac{NN} can be assimilated with a crossbar array. Dense layers are mostly used as by fully parallel voltage-driven crossbar array in this thesis to keep the timing simple, otherwise the timing of the enable flags, and later on the \ac{LSTM}'s memory cells (\cref{sec:lstm}), would get too complicated.

\subsubsection{Resistors vs Memristors}

In this work, the analog system will be simulated for the inference of the \ac{NN}, thus the weights will not have to change during the simulation. Because the weights are represented in the analog circuit by the internal resistances of the memristors, the memristors can be replaced by resistors with a set resistance for the simulation.

\section{Inverter}
\label{sec:inv}

This section simply describes a very well know electrical component, the inverter. When it a voltage of $V_{dd}$ is applied to the input, the output is grounded and the same goes for the other way around.

\begin{figure}[H]
  \centering
  \hspace*{2.5cm}
  \subfloat[Inverter's circuit]{\label{circt:inv}\includesvg[height=3cm]{inverter/invCircuit}}%
  \hfill
  \subfloat[Inverter's symbol]{\label{sym:inv}\includesvg[height=2.5cm]{inverter/invSymbol}}%
  \hspace*{1.5cm}
  \caption{}
  \label{fig:inv}
\end{figure}

Like mentionned, the circuit for the inverter is very straight forward (\cref{circt:inv}). The symbol is nothing fancy as well as it is the classic symbol (\cref{sym:inv}) for such a circuit.

The approximate onChip area for the inverter is :
\begin{equation}
  \symA_{inv}=2\cdot \symA_{CMOS}
\end{equation}

With $\symA_{CMOS}= w\cdot l=2\cdot 10^{-7}\cdot6\cdot10^{-8} =1.2\cdot10^{-14}m^2$.

The onChip area of the inverter is then $\symA_{inv}=2.4\cdot10^{-14}m^2$.

\section{Verilog models}
\label{sec:models}

Due to a lack of time, some of the more common component were not designed by me but instead were simulated using a Verilog model. Verilog is a hardware description laguage very popular in the industry. The only components that use a verilog model are the voltage multiplier and the \ac{opAmp}.

\subsection{Operational amplifier}\label{subsec:opamp}

This component is the very famous \ac{opAmp}. This specific component wasn't designed for the thesis, it required a current range that didn't allow to use ones already made by members of the research group. The choice was then to use a verilog-A model, and then design one if time allows it.

\begin{equation}
  \label{eq:opAmp}
  V_{out}=\mu \cdot (V_+-V_-)
\end{equation}

\Cref{eq:opAmp} is the equation used in the verilog-A model. It makes it so it behaves like an ideal \ac{opAmp}. For this thesis $\mu$ has been set to $\mu=10^5$.

\subsection{Voltage multiplier}\label{subsec:voltmult}

This component while far less popular than the latter, is just as useful for our specific use. It allows us to multiply, as its name implies, two voltages. It is used to compute the pointwise multiplications of the \ac{LSTM} (\cref{fig:lstmCell}).

However this part is tricky. Indeed, because of the voltage conversion (\cref{tab:valConv}), the equation needs to take that into consideration. The final equation needs to be \cref{eq:finalVoltMult} in order to have $1\cdot 1=1$ ($V_{in_1}\cdot V_{in_2}=1V$, with $V_{in_1}=V_{in_2}=1V$)

\begin{equation}\label{eq:finalVoltMult}
  V_{out}=10\cdot(V_{in_1}-V_{cm})\cdot (V_{in_2}-V_{cm}) + V_{cm}
\end{equation}

This is taken care of using in reality two parts, the actual voltage multiplier (\cref{eq:voltMult}) and a non inverting amplifier (\cref{eq:invAmp}), the circuit of which is available in \cref{apsec:invAmp}.

\begin{equation}\label{eq:voltMult}
  V_{voltMult}=-(V_{in_1}-V_{cm})\cdot (V_{in_2}-V_{cm}) + V_{cm}
\end{equation}
\begin{equation}\label{eq:invAmp}
  V_{out}=-(V_{voltMult}-V_{cm})\cdot10+V_{cm}
\end{equation}

Where $V_{out}$ is the output voltage the inverting amplifier (\cref{apsec:invAmp}), $V_{voltMult}$ is the out voltage of the voltage multiplier itself and $V_{in_1}$ and $V_{in_2}$ are the input voltages.

The \cref{eq:voltMult} is assumed possible because of the actual voltage multiplier's datasheet available at \cite{actualVoltMult}.

\subsection{Symbols}

The symbols for those two components are the ones in \cref{sym:models}.

\begin{figure}[H]
  \centering
  \hspace*{2cm}
  \subfloat[\ac{opAmp}'s symbol\label{sym:opAmp}]{\includesvg[height=2.5cm]{models/opAmpSymbol}}%
  \hfill
  \subfloat[Voltage multiplier's symbol\label{sym:voltMult}]{\includesvg[height=2.5cm]{models/voltMultSymbol}}%
  \hspace*{2cm}
  \caption{Symbols used for the verilog-A components}
  \label{sym:models}
\end{figure}

The \ac{opAmp} uses its classic symbol (\cref{sym:opAmp}) while the voltage multiplier uses a custom symbol (\cref{sym:voltMult}).

\section{\ac{LSTM} analog implementation}\label{sec:lstmCircuit}

\subsection{Circuit}

This section describes the circuit of an \ac{LSTM} with an input vector of size $n_i$, a $n_h$ hidden states, a serial size of $n_s$ and $n_{ts}$ time steps. $n_o=n_h/n_s$ is going to be used for future references in this section. In order for the crossbar array to be used $n_o$ must be an integer, in other words, $n_s$ must divide $n_h$. The circuit (shown in \cref{circt:lstm}) is pretty complex and contains numerous parts that require explaination.

\begin{figure}[H]
  \centering
  \includesvg[width=\textwidth,pretex=\tiny]{lstm/lstmCircuit}
  \caption{\ac{LSTM} circuit}
  \label{circt:lstm}
\end{figure}

The system is built using crossbar array (\cref{sec:xbarCircuit}) with ($n_i+n_h+1$,$n_o$, $n_s$) as parameters.

First of all, the different vectors and variables present in the schematic have to be described :

\begin{itemize}
  \item $\overrightarrow{x_t}$ : This is the input vector for the \ac{LSTM} circuit at time $t$. It has a size of $n_i$.
  \item $\overrightarrow{h_t}$ : This is the hidden layer vector for the feedback connections, it is defined as $\overrightarrow{h_t}=(\overrightarrow{h_{t,i}}) \forall i\in [\![0,n_o-1]\!]$, with $\overrightarrow{h_{t,i}}=(h_{t,i,j}) \forall j\in [\![0,n_s-1]\!]$.
  \item $\overrightarrow{z_t}$ is the input of the crossbar but not the input of the \ac{LSTM}. This vector is there to lighten the informations on the schematic (\cref{circt:lstm}). $\overrightarrow{z_t}$ is defined by $\overrightarrow{z_t}=(\overrightarrow{x_t},\overrightarrow{h_{t-1}},b)$.
  \item $e_{j,0}$ and $e_{j,1}$ are two enable flags that respectively represent the first and second half of $e_j$.
  \item $e_{in}$ and $e_{out}$ are the flags used to enable the hidden state values to go to the input (feedback connection) or to the output of the circuit.
  \item $e_{next}$ is the enable flag on in between two time steps.
  \item $R_{amp0}$ and $R_{amp1}$ are the two resistances used to amplify the output voltage of the voltage multipliers. The ratio of the resistances value must of $\frac{R_{amp1}}{R_{amp0}}=10$. The resistances must stay arround the values of the resistances used around the circuit, especially those of the memristors.
\end{itemize}

The wires coming into the crossbar are a bus of size $n_i+n_h+1$ and the output of the crossbar is a bus of size $n_o$ (\cref{sym:xbar}). This is why everything in the system apart from the crossbar arrays is only shown once in \cref{circt:lstm} but in reality those components are present $n_o$ times. Those extra components are needed in order for the parallel channels to work.

\subsection{Doubling memory cells}

\subsubsection{Feedback hidden states}

In order for the system to work in serial mode, the memory cells of the output need to be doubled. This is done in \cref{circt:lstm}, and allows for the hidden states to be saved for the next stage. Indeed, if using the system in serial mode with only one line of memory cells, the old hidden state ($\overrightarrow{h_{t-1}}$) is slowly overwritten by the current hidden state ($\overrightarrow{h_t}$). As soon as the first $n_o$ serial values are computed, they are going to override the old ones, still required for the following serial values. \Cref{tim:lstmMemcell} shows that the value in the memory cell changes too early in the cycle and gives a bad input for the next serial values. The value given to the input of the \ac{LSTM} should always be the old hidden state ($\overrightarrow{h_{t-1}}$).

\begin{figure}[H]
  \centering
  \begin{tikztimingtable}
    $e_{n_s-1}$ & 3H8Lx \\
    $e_0$ & 3L4H4Lx \\
    $e_1$ & 7L4Hx \\
    $m_{i,0}$ & 3D{$h_{t-1,i,0}$}8D{$h_{t,i,0}$}x \\
    $m_{i,1}$ & 7D{$h_{t-1,i,1}$}4D{$h_{t,i,1}$}x \\
    \extracode
    \tablerules
    %\draw (0,0) circle (2pt); % Origin
    \begin{pgfonlayer}{background}
      \vertlines[help lines]{3.05}
      %\vertlines[red]{1.6,5.6,15.6}
      %\vertlines[blue]{3.6,9.6,15.6}
    \end{pgfonlayer}
  \end{tikztimingtable}
  \caption{Time diagram of the values in the different memory cells ($m_{i,j}$ being the value stored in the $i^{th}$ parallel memory cells for the $j^{th}$ serial value). It is assumed that $n_s>1$ (the system is in serial mode). This graph does not take into account the pauses $e_{next}$ between the time steps.}
  \label{tim:lstmMemcell}
\end{figure}

Using two memory cells when in serial mode allows for this issue to be elevated. The values are transfered from one level to an other at the end of the time step. That way, the values are updated safely during each time step, without changing the final output.

When the system is running is a fully parallel mode, doubling the memory cells still causes a problem. The enable in and enable out flags can't be high at the same time when the memory cell has a feedback connection, the signal would interfere with it self and change its own value. Potential solutions are discussed in \cref{subsec:noDoubleMemcell}.

\subsubsection{Cell states}

Cell states have the same solution to a similar problem. Here the problem is that the memory cell has a feedback connection and needs to be activated within one serial step. The value of the old cell state needs to be used to compute the current values for the hidden states ($\overrightarrow{h_t}$) and cell states ($\overrightarrow{c_t}$). The issue is that at the value in the memory cell needs to be conserved until the next serial step.  (\cref{subsec:noDoubleMemcell}).

Once again, when the system is running in fully parallel mode, only one memory cell line is required for a normal behavior.

\subsection{Serialization/Parallelization}
\label{subsec:serpar}

Using an \ac{LSTM} in serial mode is very beneficial as it divides the number of point wise circuit by $n_s$.

There is a sweet spot for the serial size ($n_s$) :

\begin{itemize}
  \item When the serial size ($n_s)$ increases, the inference time increases with a factor of $O(n_s)$ while the onChip area decrease with the same factor.
  \item When the serial size ($n_s)$ decreases, similarly, the onChip area increases with a factor of $O(n_s)$ while the tinference time decreases by the same factor.
\end{itemize}

The overall onChip area is linked to the number of hidden states ($n_h$) by $O(n_h^2)$. This means that the serial size ($n_s$) will be set depending on the limiting factor of our system (onChip area or inference time).

\subsection{Symbol}

The symbol of the \ac{LSTM} circuit can be found in \cref{sym:lstm}. It depends on several parameters, the number of inputs ($n_i$), the number of hidden states ($n_h$), the serial size of the system ($n_s$) and the number of time steps ($n_{ts}$).

\begin{figure}[H]
  \centering
  \includesvg[height=2.5cm]{lstm/lstmSymbol}
  \caption{Symbol used for the \ac{LSTM} circuit}
  \label{sym:lstm}
\end{figure}

The approximate onChip area for the crossbar circuit depends on the previously defined parameters.

\begin{equation}
  \begin{array}{3}
    A_{lstm}(n_i,n_h,n_s)=4\cdot A_{xbar}(n_i,\frac{n_h}{n_s},n_s)\\+ \frac{n_h}{n_s}\cdot (5\cdot A_{af}+3\cdot A_{voltMult}+3\cdot A_{R_{amp0}}+2\cdot A_{R_{amp1}}+2\cdot A_{opAmp})\\+5\cdot n_h\cdot A_{memcell}
  \end{array}
\end{equation}

\section{\acs{GRU} analog implementation}\label{sec:gruCircuit}

\subsection{Circuit}

This section describes the circuit of an encoder \ac{GRU} with an input vector of size $n_i$, a $n_h$ hidden states and $n_{ts}$ time steps. The decoder \ac{GRU} could also be implemented in an analog circuit, but the choice was made to focus on the encoder \ac{GRU}. The \ac{GRU} is by its nature very similar to the \ac{LSTM}.

\begin{figure}[H]
  \centering
  \includesvg[width=\textwidth,pretex=\tiny]{gru/gruCircuit}
  \caption{\ac{LSTM} circuit}
  \label{circt:lstm}
\end{figure}

The system is built, once again, using crossbar array (\cref{sec:xbarCircuit}) with ($n_i+n_h+1$,$n_o$, $n_s$) as parameters.

First of all, the different vectors and variables present in the schematic have to be described :

\begin{itemize}
  \item $\overrightarrow{x_t}$ : This is the input vector for the \ac{LSTM} circuit at time $t$. It has a size of $n_i$.
  \item $\overrightarrow{h_t}$ : This is the hidden layer vector for the feedback connections, it is defined as $\overrightarrow{h_t}=(\overrightarrow{h_{t,i}}) \forall i\in [\![0,n_o-1]\!]$, with $\overrightarrow{h_{t,i}}=(h_{t,i,j}) \forall j\in [\![0,n_s-1]\!]$.
  \item $\overrightarrow{z_t}$ is the input of the crossbar but not the input of the \ac{LSTM}. This vector is there to lighten the informations on the schematic (\cref{circt:lstm}). $\overrightarrow{z_t}$ is defined by $\overrightarrow{z_t}=(\overrightarrow{x_t},\overrightarrow{h_{t-1}},b)$.
  \item $e_{j,0}$ and $e_{j,1}$ are two enable flags that respectively represent the first and second half of $e_j$.
  \item $e_{in}$ and $e_{out}$ are the flags used to enable the hidden state values to go to the input (feedback connection) or to the output of the circuit.
  \item $e_{next}$ is the enable flag on in between two time steps.
  \item $R_{amp0}$ and $R_{amp1}$ are the two resistances used to amplify the output voltage of the voltage multipliers. The ratio of the resistances value must of $\frac{R_{amp1}}{R_{amp0}}=10$. The resistances must stay arround the values of the resistances used around the circuit, especially those of the memristors.
\end{itemize}

The wires coming into the crossbar are a bus of size $n_i+n_h+1$ and the output of the crossbar is a bus of size $n_h$ (\cref{sym:xbar}). This is why everything in the system apart from the crossbar arrays is only shown once in \cref{circt:lstm} but in reality those components are present $n_o$ times. Those extra components are needed in order for the parallel channels to work.

\subsection{Serialization/Parallelization}
\label{subsec:gruSerPar}

As opposed to the to the \ac{LSTM}, the \ac{GRU} can not be serialized. This is due to a mathematical issue.

Indeed in the equation of the candidate hidden state is a problem for the serialization of the circuit. The \cref{eq:candActivG} from \cref{sec:gru} is repeated in \cref{eq:serProb} for better lisibility of the thesis.

\begin{equation}\label{eq:serProb}
  \overrightarrow{\hat{h_t}}=tanh(\overrightarrow{x_t}\cdot w_{hx}+(\overrightarrow{r_t}\odot\overrightarrow{h_{t-1}}) \cdot w_{hh} + \overrightarrow{b_h})
\end{equation}

As \cref{eq:serProb} shows, the reset gate's results vector needs to be multiplied with the the hidden state's previous vector. However, while the hidden state's previous vector is fully available in the memory cells, the reset gate's vector needs to be computed using the current inputs.

If the circuit were serialized, a full cycle would be required to compute the reset gate, that is itself required to compute the candidate hidden state.

That is why, the \ac{GRU} would take twice as much time to compute a time step if it were serialized. It is not worth it to double the computation time to save a bit of onChip area.

\subsection{Symbol}

The symbol of the \ac{GRU} circuit is available in \cref{sym:lstm}. It depends on a few parameters namely, the number of inputs ($n_i$), the number of hidden states ($n_h$) and the number of time steps ($n_{ts}$).

\begin{figure}[H]
  \centering
  \includesvg[height=2.5cm]{gru/gruSymbol}
  \caption{Symbol used for the \ac{GRU} circuit}
  \label{sym:lstm}
\end{figure}

The total onChip area for the crossbar circuit depends on the previously defined parameters.

\begin{equation}
  \begin{array}{3}
    A_{gru}(n_i,n_h)=4\cdot A_{xbar}(n_i,\frac{n_h}{n_s},n_s)\\+ \frac{n_h}{n_s}\cdot (5\cdot A_{af}+3\cdot A_{voltMult}+3\cdot A_{R_{amp0}}+2\cdot A_{R_{amp1}}+2\cdot A_{opAmp})\\+5\cdot n_h\cdot A_{memcell}
  \end{array}
\end{equation}

\section{Weights generation}
\label{sec:genwei}

The choice for the weight generation was to use Keras API with the tensorflow framework. However, both tensorflow and pyTorch were tried and were giving similar results, the final choice of using tensorflow was made because it is the most popular among the research group.

\subsection{Training the weights}

Generating the weights requires to train a \ac{NN}. This is done in Keras by creating the model, in other words the architecture required to solve the problem. This model contains the description of the \ac{NN}, such as the type of \ac{RNN} and the Dense layers that would come before of after the \ac{RNN}. In Keras, there are only two of the \ac{LSTM} variants (\cref{sec:lstm}) available :

\begin{itemize}
  \item The \ac{NP} \ac{LSTM} : This is most commonly used version and by such one of the two available in Keras.
  \item The \ac{GRU} : This one is present here because it is considered to be different from an \ac{LSTM} \ac{NN} despite their similarities.
\end{itemize}

The other kinds of \ac{LSTM} are not present in the default Keras librairies.

Training can only be done outside of the circuit as this state of the project. However, in order to train the weights as closely as they would have been on a real circuit, the activation functions used for the training are the custom activation functions generated by the dedicated circuit (\cref{sec:af}). In other words, the activation functions used are the ones shown in \cref{fig:afGraph}. This allows to train the weights as they almost (considering the activation functions are recreated using 51 simulated points) as they would have been in the real circuit.

All the weight trainings are done using \ac{MSE} loss function and Adam optimizer \cite{adamOpti}.

\subsection{Weights constaint}

The weights need to be contrained during the training to make sure that the voltage of the analog circuit stays within its operating voltage range ($[0,V_{dd}]$). To avoid any unwanted behavior when the system supposedly goes out of range, the weights in the \ac{LSTM} are contrained. This constrain is not fixed and depends on the architecture it is generated with.

The worst case scenario for an \ac{LSTM} is when every value reaches the voltage threshold ($V_{threshold}$) (\cref{sec:xbarCircuit}) and every weight is maximized ($w_{max}$). When that is the case the output of the \acp{VMM} is found in \cref{eq:weiCons}.

\begin{equation}\label{eq:weiCons}
  V_{threshold}\cdot w_{max} \cdot(n_i+n_h)+w_{max}+V_{cm}= V_{dd}
\end{equation}

We can then determine the maximum acceptable value for the weights ($w_{max}$), this value has thus been determined to be the one in \cref{eq:weiConsRes}.

\begin{equation}\label{eq:weiConsRes}
  w_{max}=\frac{V_{dd}-V_{cm}}{V_{threshold}\cdot(n_i+n_h)+1}
\end{equation}

The analog system being completly centered and symetrical around $V_{cm}$ the value for the minimum weight is the opposite to the maximum one (\cref{eq:weiConsRes1})

\begin{equation}\label{eq:weiConsRes1}
  w_{min}=-w_{max}=-\frac{V_{dd}-V_{cm}}{V_{threshold}\cdot(n_i+n_h)+1}
\end{equation}

A way to remove those constraints is discussed in \cref{subsec:noCons}. Having a constraint on the weights limits the performance of the final \ac{NN}.

\subsection{Exporting the weights}

Once all the parameters (number of hidden states, number of dense layers, etc) have been set. The weights can be exported using the required weight repartition layed out in \cref{sec:netlist}. A description of the architecture is also saved along the weights. This file can now be used as the input for the netlist generator (\cref{sec:netlist}).

The code used for all the weights generations is available at \cite{lstmWei}. The code containing the finctions to save the weights to a file is available in \ac{apsec:saveWei}.

\section{Netlist generation}
\label{sec:netlist}

In order to be able to run the simulation with different kinds of \ac{NN} architecture. The point of the part is thus to explain how the netlist generator tool works. The tool also work to generate any architecture with the supported layers (listed in the README.md of \cite{lstmGen}).

This is done by generating a SPICE netlist using a python script. A SPICE netlist was chosen because of Cadence's virtuoso limitations. Indeed, Verilog netlists can be imported with a downside, component's parameters can't be imported. This is very limiting for this thesis' use case (required to set the resistors' resistance and thus the weights). SPICE netlists can import parameters for the components and is open source and by such well documented. The generator script takes in a few parameters :

\begin{itemize}
  \item The first parameter is the number of input we use our system. This is the size of the first input vector ($x_t$).
  \item The next parameter is the number of time steps. Everything about the \ac{LSTM} time steps is all explained in \cref{sec:lstm}.
  \item The serial size of the crossbar arrays, as described in \cref{sec:xbarCircuit}, used in the \ac{LSTM} network. This parameter must divide the number of hidden states in the \acp{LSTM} layers.
  \item The files containing different informations about the model. This file contains the type of \ac{NN} architecture and the weights associated with each layer. The weights in the files have to be organized using a specific model (\cref{subsec:weiStore}).
  \item Finally the name of the file in which the output of the script (the netlist) will be written to.
\end{itemize}

\subsection{Available layers}

As of the writing of the thesis, the netlist generator can generate the netlist for the \ac{NP} \ac{LSTM} and the \ac{GRU}. Other type of \acp{NN} can be added to the code.

\subsection{Weights storage}\label{subsec:weiStore}

The weights are stored in a single file. This file contains a brief description of the architecture being used. The first index stores this desciption. The following indexes are for the weights of each layer, in the order given in the description. Depending on the layer used, the weights are stored a bit differently :

\begin{itemize}
  \item Dense layer : This is basically a \ac{VMM} so the weights are just sorted linearly in a list. The list is of size $n^2$, where $n$ is the size of the input vector. \Cref{mtrx:wei} shows the position ($i$) of the weight ($w_i$) in the matrix. This is represented in the description as "Dense($n_o$)" where $n_o$ is the number of output of the layer.
  \item \ac{LSTM} layer : An \ac{LSTM} layer contains four \acp{VMM} which can be assimilated and thus stored like a dense layer. The weights for each \ac{VMM} is then stored in a sub list. This is represented in the description as "LSTM($n_h$)" where $n_h$ is the number of hidden states of the layer.
\end{itemize}

\begin{equation}\label{mtrx:wei}
  \begin{bmatrix}
    w_{0} & w_{1} & \dots \\
    \vdots & w_i & \vdots \\
    \dots & w_{n^2-2} & w_{n^2-1}\\
  \end{bmatrix}
\end{equation}

All the code can be found on my github page \cite{lstmGen}.

\section{Conversion from weights to resistance}
\label{sec:wei2res}

The weight to resistances equations has been found using \cref{eq:wei2res0}.

\begin{equation}
  \label{eq:wei2res0}
  \symw=\symR_f\cdot( \symG_+- \symG_-)=\frac{\symR_f}{\symR_+}-\frac{\symR_f}{\symR_-}
\end{equation}
First, we know from \cref{eq:doubleMem2} that the weight is represented by \cref{eq:wei2res0}. This means that the weights are limited in values to :

\begin{itemize}
  \item $\symw_{max}=\symR_f\cdot( \symG_{max}- \symG_{min})$
  \item $\symw_{min}=-\symw_{max}=\symR_f\cdot( \symG_{min}- \symG_{max})$
\end{itemize}

For the rest of this chapter we consider that $\symR_f$ is set to the middle point of $\symR_{max}$ and $\symR_{min}$ meaning that $\symR_f=\frac{\symR_{min}+\symR_{max}}{2}$.

Since we only have one equation (\cref{eq:wei2res0}) for 2 unknowns ($\symR_+$ and $\symR_-$), we need to set a second equation. This is done by centering the resistances around $\symR_f$, this means that \cref{eq:wei2res1} is the second equation, that makes our problem now solvable.

\begin{equation}
  \label{eq:wei2res1}
  \symR_f=\frac{\symR_-+\symR_+}{2}
\end{equation}
\begin{equation}
  \label{eq:wei2res01}
  \begin{cases}
    \symw=\frac{\symR_f}{\symR_+}-\frac{\symR_f}{\symR_-}\\
    \symR_f=\frac{\symR_-+\symR_+}{2}
  \end{cases}
\end{equation}

By solving \cref{eq:wei2res01}, we find the \cref{eq:wei2res2} that gives the values for $\symR_-$ and $\symR_+$. All the steps for solving the equations can be found in \cref{apsec:wei2res}.

The real value to voltage conversion (\cref{tab:valConv}) does not affect this part of the system as the crossbar array already works in voltages. Replacing the resistor variable in \cref{eq:doubleMem2} by \symw using \cref{eq:wei2res0} would not change the results.

\begin{equation}
  \label{eq:wei2res2}
  \begin{cases}
    \symR_+= (\symw+1-\sqrt{\symw^2+1})\cdot\frac{\symR_f}{\symw}\\
    \symR_-=2\cdot \symR_f -\symR_+
  \end{cases}
\end{equation}

While solving the system (\cref{apsec:wei2res}), we get two potential equations for $\symR_+$. Graphing them shows that one of them is outside of the memristor's resistance range ($[\symR_{min},\symR_{max}]$). Leaving us with only one equation because the other one is physically unreachable.

In the present work, this step is done in python and integrated in the netlist generator script (\cref{sec:netlist}). The resolution of the memristor (the precision of the resistances at which the memristor can be set) is simulated by limiting the generated resistances values to having only two significant figures.

The python implementation is available in \cref{apsec:wei2res}.

\section{The datasets}\label{sec:dataset}

To make sure that the terminal voltage does not reach the threshold as explained in \cref{sec:xbarCircuit}. The voltage threshold ($V_{read}$) was chosen to be $V_{read}=0.1V$ because the memristors used in this thesis are purely theorical. The dataset needs to be formated not to exceed this value when converted to voltage (\cref{tab:valConv}), the dataset needs to be trained while not exceeding $1$.

\subsection{Airline passengers}

The first dataset contains a time series of international airline passengers from January 1949 to December 1960, the data is recorded monthly and is in thousands. The dataset then contains twelve years of monthly data so $144$ sample points. Although the results of this problem are not necessarely tailored to the system, in the sense that it is not time sensitive information, it is nonetheless a problem that can, in a straightforward manner, demonstrate the predictive capacity of the proposed model. It is extracted from the tutorial available at \cite{airline}. The dataset is available at \cite{datasets}, the name of the file containing the dataset is \textit{airline-passengers.csv}.

This is a regression problem, the role of the \ac{LSTM} is to predict the number of passenger flying the next month being given previous months' passenger count.

\begin{figure}[H]
  \centering
  \includesvg[width=\textwidth]{datasets/airline}
  \caption{The airline dataset. The vertical lines represent a full year.}
  \label{graph:airline}
\end{figure}

\subsubsection{Data format}

The dataset contains $144$ sample points. It has been transformed into $142$ data points for training. This has been done by taking each values three by three. The first two values are two timesteps for the input vector and the third value being used the target value. Two third of the dataset is being used for training and the other third is used for validation.

\subsection{\ac{C. elegans}}\label{subsec:celegans}

This dataset is far more interesting and complex than the latter. This data set aims to use \acp{LSTM} to mimic the behavior of real neurons. As explained in \cite{celegans}, \acf{C. elegans} are simple organisms that are getting very popular for whole brain organization studies. The point of this problem is to reproduce the nervous system of the \ac{C. elegans}. This is done using recorded data of the input of 4 neurons and the output of 4 other neurons.

The dataset is great for our study because :
\begin{itemize}
  \item It comes from a very recent paper from the research group in which this work is also being developed.
  \item It aims at reproducing the behavior of the brain of a simple organism (\ac{C. elegans}). In the (very) long term, a full parts of the human brain could be replaced by a very low powered chip.
\end{itemize}

\begin{figure}[H]
  \centering
  \begin{minipage}{\columnwidth}
    \subfloat[Training dataset sample\label{graph:io5Celegans}]{\includesvg[width=0.5\textwidth, pretex=\tiny]{datasets/celegans/ioTrain0}}%
    \hfill
    \subfloat[Test dataset sample]{\includesvg[width=0.5\textwidth,pretex=\tiny]{datasets/celegans/ioTest0}}%
  \end{minipage}
  \begin{minipage}{\columnwidth}
    \subfloat[Training dataset sample]{\includesvg[width=0.5\textwidth, pretex=\tiny]{datasets/celegans/ioTrain1.svg}}%
    \hfill
    \subfloat[Validation dataset sample\label{graph:io15Celegans}]{\includesvg[width=0.5\textwidth,pretex=\tiny]{datasets/celegans/ioValid0}}%
  \end{minipage}
  \caption{\ac{C. elegans} dataset samples}
  \label{graph:celegans}
\end{figure}

\subsubsection{Dataset format}

The dataset contains a set of fourty set of input/output. Each input/output set contains $500ms$ of data with a time step of $0.5ms$, making a thousand points for each neurons. This data is recorded after applying current to the input neurons (PLML2, PLMR, AVBL, AVBR), and monitoring the output of four neurons (DB1, LUAL, PVR, VB1) that are known to have strong activity during  a specific behavior of the nematode known as \ac{FCM}. The simulation is done using a known model of \ac{C. elegans} connectome (complete overview of the brains connections) \cite{celegans}.

\Cref{graph:celegans} shows four sets of input/output out of the fourty that were recorded for the dataset.

The inputs for the systems are then going to be the current values for the four input neurons at every of the thousand time steps and plans to get an output as close as possible to the modeled output for all time steps.

\section{Running the simulation}\label{sec:run}

The simulation is ran using \textbf{Cadence}'s \textit{Virtuoso} simulator, using a parametric simulation to simulate all the inputs. The use of the parametric simulation allows to change the values for the data we want to use as input.

In order to run the simulation some variables need to be set. The time step ($T$) value was chosen to be $T=8\mu s$, the pause between two \ac{LSTM} steps is set to be $\frac{T}{8}=1\mu s$. The value for the maximum and minimum resistances value of the memristor used ($R_{max}$ and $R_{min}$, respectively) are set to be in acordance with the \textbf{KNOWM}\textregistered{}'s memristors \cite{Knowm}. The resistance values used for the simulation are thus $R_{max}=10^6 \Omega=1M\Omega$ and $R_{min}=10^4 \Omega = 10k\Omega$.

Once everything is set, the simulation can be started. The result are recorded by looking at the output net when at the time of output activation. The name of the output nets is outputed by the netlist generator script (\cref{sec:netlist}).

\cleardoublepage
