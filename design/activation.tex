\section{The activation functions}
\label{sec:af}

Activation functions play a great role in the results obtained \cite{af}. It is the reason why making good Activation function circuits is fundamental. Of course the easy way would be to simply design a hard sigmoid (\cref{apsec:hardFunc}), the issue being that the hard sigmoid is much worse than the regular sigmoid, especially for regression problems \cite{hardSigm}. The same goes for the \ac{tanh} and hard \ac{tanh} functions.

Designing an analog activation function as close to the original is very important for the final result's quality.

\begin{table}[H]
  \centering
  \begin{tabular}{|c|c|c|}
    \hline
    \rowcolor{gray}
    Parameter & Sigmoid & \ac{tanh} \\
    \hline
    $V_1$ & \multicolumn{2}{c|}{$1.1V$}\\
    \hline
    $V_2$ & \multicolumn{2}{c|}{$635mV$}\\
    \hline
    $V_3$ & $0.8V$ & $550mV$\\
    \hline
    $i_{dc}$ & \multicolumn{2}{c|}{$150uA$}\\
    \hline
    $w$ & \multicolumn{2}{c|}{$900nm$}\\
    \hline
    $l$ & \multicolumn{2}{c|}{$60nm$}\\
    \hline
    $R_1$ & \multicolumn{2}{c|}{$5k\Omega$}\\
    \hline
    $R_2$ & \multicolumn{2}{c|}{$10k\Omega$}\\
    \hline
    $R_3$ & $2k\Omega$ & $4k\Omega$\\
    \hline
  \end{tabular}
  \caption{Circuits parameters}
  \label{tab:afPar}
\end{table}

Here, $w$ and $l$ are, respectively, the width and length of the two NMOS of the circuit.

\subsection{Circuit}

The circuit used is the same as the one in \cite{thesisRef}, the circuit is the one shown in \cref{circt:af}. The technology used being different, all the parameters had to be determined empirically to best fit a sigmoid shape. The parameters can be found in \cref{tab:afPar}.

Due to the nature of the functions we want to generate, we will use the same circuit for both a sigmoid and a \ac{tanh} like functions. The two different functions are generated by changing two parameters.

The functions generated are the same shape and only differ by their output range.

\begin{figure}[H]
  \centering
  \includesvg[width=\textwidth]{activation/afCircuit}
  \caption{Activation functions circuit}
  \label{circt:af}
\end{figure}

\subsection{Symbols}
The symbols for the sigmoid and the \ac{tanh} are separated for better understanding.

\begin{figure}[H]
  \centering
  \hspace*{0.8cm}
  \subfloat[Sigmoid symbol]{\includesvg[height=2.5cm]{activation/sigmoidSymbol}}%
  \hfill
  \subfloat[\ac{tanh} symbol]{\includesvg[height=2.5cm]{activation/tanhSymbol}}%
  \hspace*{0.8cm}
  \caption{Activation functions symbols with the input and output pins on either side depending on the flow of the current for better readability}
  \label{fig:afSymbol}
\end{figure}

The approximate onChip area of this circuit is :

\begin{equation}
  A_{af}=2\cdot A_{R_1} + 2\cdot A_{CMOS} + 5\cdot A_{R_2} + A_{R_3} +2\cdot A_{opAmp}
\end{equation}
With $A_{CMOS} = w\cdot l$

In order to combine the the onChip area of both the sigmoid and \ac{tanh} activation functions, it is assumed that the variation of area taken by the resistor ($R_3$) is negligible compared to th overall area.

\subsection{Usage}

This circuit outputs a voltage that depends on the input voltage passed on. The relation between the two is shown in \cref{fig:afGraph}. The graph also shows the \ac{RMSE} of each graph compared to the ideal result.

Note that the actual output of the circuit in \cref{circt:af} is inverted arround $V_{cm}$ a simple circuit such as the one shown in  (TODO show circuit in annex) does inverts the

\begin{figure}[H]
  \centering
  \includesvg[width=\textwidth]{activation/afGraph}
  \caption{Input/Output graph of the activation function circuit for both sigmoid and \acs{tanh} functions}
  \label{fig:afGraph}
\end{figure}

These functions are still a bit different from the original functions (especially for the \ac{tanh}). However that doesn't matter too much as the trainning will be happening within the final circuit, all weights will be set in the circuit. This is the reason why such a difference doesn't matter. As long as the curves have the similar shape, the result won't be drastically affected.
