\section{Running the simulation}\label{sec:run}

The simulation is ran using Cadence's virtuoso simulator, using a corner simulation to simulate all the inputs.

In case the \ac{LSTM} we want to run requires several outputs, the use of the corner functionality. Corners are intended to be used for small variations of the variables in the circuit. However, it used differently in this thesis. It runs with the value for each input variable changing for the current \ac{LSTM} run and then moves on to the next with the variables changed.

In order to run the simulation some variables need to be set. The time step ($T$) value was chosen to be $T=8\mu s$, the pause between two \ac{LSTM} steps is set to be $\frac{T}{8}=1\mu s$. The value for the maximum and minimum resistances value of the memristor used ($R_{max}$ and $R_{min}$, respectively) are set to an arbitrary value of $R_{max}=10^6 \Omega=1M\Omega$ and $R_{min}=10^4 \Omega = 10k\Omega$.

Once everything is set, the simulation can be started. The result are recorded by looking at the output net when at the time of output activation. The name of the output nets is outputed by the netlist generator script (\cref{sec:netlist}).
