\section{Conversion from weights to resistance}
\label{sec:wei2res}

The weight to resistances equations has been found using \cref{eq:wei2res0}.

\begin{equation}
  \label{eq:wei2res0}
  \symw=\symR_f\cdot( \symG_+- \symG_-)=\frac{\symR_f}{\symR_+}-\frac{\symR_f}{\symR_-}
\end{equation}
First, we know from \cref{eq:doubleMem2} that the weight is represented by \cref{eq:wei2res0}. This means that the weights are limited in values to :

\begin{itemize}
  \item $\symw_{max}=\symR_f\cdot( \symG_{max}- \symG_{min})$
  \item $\symw_{min}=-\symw_{max}=\symR_f\cdot( \symG_{min}- \symG_{max})$
\end{itemize}

For the rest of this chapter we consider that $\symR_f$ is set to the middle point of $\symR_{max}$ and $\symR_{min}$ meaning that $\symR_f=\frac{\symR_{min}+\symR_{max}}{2}$.

Since we only have one equation (\cref{eq:wei2res0}) for 2 unknowns ($\symR_+$ and $\symR_-$), we need to set a second equation. This is done by centering the resistances around $\symR_f$, this means that \cref{eq:wei2res1} is the second equation, that makes our problem now solvable.

\begin{equation}
  \label{eq:wei2res1}
  \symR_f=\frac{\symR_-+\symR_+}{2}
\end{equation}
\begin{equation}
  \label{eq:wei2res01}
  \begin{cases}
    \symw=\frac{\symR_f}{\symR_+}-\frac{\symR_f}{\symR_-}\\
    \symR_f=\frac{\symR_-+\symR_+}{2}
  \end{cases}
\end{equation}

By solving \cref{eq:wei2res01}, we find the \cref{eq:wei2res2} that gives the values for $\symR_-$ and $\symR_+$. All the steps for solving the equations can be found in \cref{apsec:wei2res}.

The real value to voltage conversion (\cref{tab:valConv}) does not affect this part of the system as the crossbar array already works in voltages. Replacing the resistor variable in \cref{eq:doubleMem2} by \symw using \cref{eq:wei2res0} would not change the results.

\begin{equation}
  \label{eq:wei2res2}
  \begin{cases}
    \symR_+= (\symw+1-\sqrt{\symw^2+1})\cdot\frac{\symR_f}{\symw}\\
    \symR_-=2\cdot \symR_f -\symR_+
  \end{cases}
\end{equation}

While solving the system (\cref{apsec:wei2res}), we get two potential equations for $\symR_+$. Graphing them shows that one of them is outside of the memristor's resistance range ($[\symR_{min},\symR_{max}]$). Leaving us with only one equation because the other one is physically unreachable.

In the present work, this step is done in python and integrated in the netlist generator script (\cref{sec:netlist}). The resolution of the memristor (the precision of the resistances at which the memristor can be set) is simulated by limiting the generated resistances values to having only two significant figures.

The python implementation is available in \cref{apsec:wei2res}.
