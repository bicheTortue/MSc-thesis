\section{Datasets}

\subsection{Airline passengers}

The first dataset contains a time series of international airline passengers from January 1949 to December 1960, the data is recorded monthly and is in thousands. The dataset then contains twelve years of monthly data so $144$ sample points. Although the results of this problem are not necessarely tailored to the system, in the sense that it is not time sensitive information, it is nonetheless a problem that can, in a straightforward manner, demonstrate the predictive capacity of the proposed model. The dataset is graphed in \cref{graph:airline}.

This is a regression problem, the role of the \ac{LSTM} is to predict the number of passenger flying the next month being given previous months' passenger count.

\begin{figure}[h]
  \centering
  \includesvg[width=\columnwidth,pretex=\scriptsize]{datasets/airline}
  \caption{The airline dataset. The vertical lines represent a full year.}
  \label{graph:airline}
\end{figure}

The $144$ sample points have been transformed into $142$ data points for training. The data is grouped by three, two for the input and one the target output. Two third of the dataset is being used for training and the other third is used for validation.

\subsection{C. elegans}

This dataset is far more interesting and complex than the latter. This data set aims to use \acp{LSTM} to mimic the behavior of real neurons. As explained in \cite{celegans}, Caenorhabditis elegans are simple organisms that are getting very popular for whole brain organization studies. The point of this problem is to reproduce the nervous system of the \ac{C. elegans}. This is done using recorded data of the input of 4 neurons and the output of 4 other neurons.

The dataset is great for our study because :
\begin{itemize}
  \item It comes from a very recent paper from the research group in which this work is also being developed.
  \item It aims at reproducing the behavior of the brain of a simple organism (\ac{C. elegans}). In the (very) long term, a full parts of the human brain could be replaced by a very low powered chip.
\end{itemize}

An example of an input/output sequence is shown in \cref{graph:celegansAnalog1} along with the digital and analog predictions.

The dataset contains a set of fourty set of input/output sequences. Each set of input and output set contains $500ms$ of data with a time step of $0.5ms$, making a thousand points for each neurons. This data is recorded after applying current to the input neurons (PLML2, PLMR, AVBL, AVBR), and monitoring the output of four neurons (DB1, LUAL, PVR, VB1) that are known to have strong activity in response to the four inputs neurons. The simulation is done using a known model of \ac{C. elegans} connectome (complete overview of the brains connections) \cite{celegans}.
