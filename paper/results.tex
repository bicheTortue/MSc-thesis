\section{Results}

The target for our circuit is to reproduce, as closely as possible the digital prediction.

\subsection{Airline passengers}

The problem is solved using an LSTM or GRU with four hidden state connected to a Dense layer with an output size of one. The weights were trained for three hundreds epochs.

\begin{figure}[h]
  \centering
  \includesvg[width=\columnwidth, pretex=\scriptsize]{results/airline/analog}
  \caption{Analog predictions trained with the analog activation functions.}
  \label{graph:airlineAnalog}
\end{figure}

The analog results obtained in \cref{graph:airlineAnalog} are very close to the digital predictions. This is confirmed by the values in \cref{tab:airlineAnalog}. The results the closest to the target are the ones generated using a serial size of one ($n_s=1$). Indeed, its error ($28.4$) is lower than the error from the digital results ($52.3$) to the original target values, almost half of it. The other ones are still quite close to the targeted curve, but are still lower. The errors are higher when running the system in serial mode. This higher inaccuracy is probably due to the data staying for longer in the memory cells as the time steps get longer.

\begin{table}[h]
  \caption{\acp{RMSE} of each analog prediction to their associated digital prediction}
  \label{tab:airlineAnalog}
  \centering
  \begin{tabular}{|c|c|c|c|c|}
    \cline{2-5}
    \multicolumn{1}{c}{}& \multicolumn{3}{|c|}{Analog prediction} &Target\\
    \cline{1-4}
    Serial size & $n_s=1$ & $n_s=2$ & $n_s=4$ & data\\
    \hline
    Digital prediction & $28.4$ & $92.6$ & $68.5$ & $52.3$\\
    \hline
  \end{tabular}
\end{table}

The GRU circuit not being fully working produces very off predictions. For this reason, the GRU results are not included.

\subsection{C. elegans}

The problem is solved using an LSTM or GRU with eights hidden state connected to a Dense layer with an output size of four. The weights were trained for one thousand epochs.

\begin{figure}[h]
  \centering
  \includesvg[width=\columnwidth,pretex=\scriptsize]{results/celegans/out5}
  \caption{\acs{C. elegans} analog responses with their digital counterparts}
  \label{graph:celegansAnalog0}
\end{figure}

\Cref{graph:celegansAnalog0} combines the digital and analog of all four output neurons. However the analog predictions are quite unstable and barely readable. The origin of this issue being unknown, the output are thus artificially smoothed out using software. The

\begin{figure}[h]
  \centering
  \includesvg[width=\columnwidth,pretex=\scriptsize]{results/celegans/smooth5DB1}
  \caption{Averaged analog prediction forscriptsize DB1 neuron\label{graph:smooth5BD1Celegans}}
\end{figure}
\begin{figure}[h]
  \centering
  \includesvg[width=\columnwidth,pretex=\scriptsize]{results/celegans/smooth5LUAL}
  \caption{Averaged analog prediction forscriptsize LUAL neuron\label{graph:smooth5LUALCelegans}}
\end{figure}
\begin{figure}[h]
  \includesvg[width=\columnwidth,pretex=\scriptsize]{results/celegans/smooth5PVR}
  \caption{Averaged analog prediction forscriptsize PVR neuron\label{graph:smooth5PVRCelegans}}
\end{figure}
\begin{figure}[h]
  \includesvg[width=\columnwidth,pretex=\scriptsize]{results/celegans/smooth5VB1}
  \caption{Averaged analog prediction for the VB1 neuron\label{graph:smooth5VB1Celegans}}
\end{figure}

