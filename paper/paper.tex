\documentclass[conference]{IEEEtran}

\IEEEoverridecommandlockouts
% The preceding line is only needed to identify funding in the first footnote. If that is unneeded, please comment it out.
\usepackage{cite}
\usepackage{amsmath,amssymb,amsfonts}
\usepackage{algorithmic}
\usepackage{graphicx}
\usepackage{textcomp}
\usepackage{tikz-timing}
\usepackage[final]{listofsymbols}
\usepackage[table,dvipsnames]{xcolor}
\usepackage{cleveref}
\crefname{figure}{Fig.}{Figs.}
\Crefname{figure}{Fig.}{Figs.}
\crefname{table}{TABLE}{TABLES}
\Crefname{table}{TABLE}{TABLES}
\crefname{equation}{}{}
\Crefname{equation}{Equation}{Equations}
\usepackage{svg}
\usepackage{subfig}
\def\BibTeX{{\rm B\kern-.05em{\sc i\kern-.025em b}\kern-.08em
T\kern-.1667em\lower.7ex\hbox{E}\kern-.125emX}}
\graphicspath{{../figures/}}
\usetikzlibrary{calc}

\tikzset{
  alias path picture bounding box/.code=%https://tex.stackexchange.com/q/395628
    \pgfnodealias{#1}{path picture bounding box},
    gap/.style={
      circle,
      inner sep=0pt,
      minimum size=#1,
      node contents={},
      path picture={
        \tikzset{alias path picture bounding box=@}
        \fill [white] (@.265) to[out=40,in=220] (@.70) --
          (@.85)  to[out=220,in=40] (@.250) -- cycle;
        \draw [very thin] (@.265) to[out=40,in=220] (@.70)
          (@.85)  to[out=220,in=40] (@.250);
      }
    },
    gap/.default=10pt
}

\opensymdef
\newsym[Pointwise multiplication operator]{omult}{\odot}
\newsym[Unit vector, containing only ones]{vunit}{\overrightarrow{1}}

\newsym[Area]{symA}{A}
\newsym[Neural Network bias vector]{symvb}{\overrightarrow{b}}
\newsym[LSTM cell state vector]{symvc}{\overrightarrow{c}}
\newsym[Electrical capacitance]{symC}{C}
\newsym[LSTM candidate cell state vector]{symvct}{\overrightarrow{cc}}
\newsym[LSTM forget gate vector]{symvf}{\overrightarrow{f}}
\newsym[Electrical conductance]{symG}{G}
\newsym[Recurrent Neural Network hidden state vector]{symvh}{\overrightarrow{h}}
\newsym[GRU candidate hidden state vector]{symvhh}{\overrightarrow{ch}}
\newsym[Neural Network hidden perceptron]{symH}{H}
\newsym[Electrical current]{symi}{i}
\newsym[LSTM input gate vector]{symvi}{\overrightarrow{i}}
\newsym[Neural Network input perceptron]{symI}{I}
\newsym[Electrical inductance]{symL}{L}
\newsym[Memristor's internal resistance, memristance]{symM}{M}
\newsym[LSTM output gate vector]{symvo}{\overrightarrow{o}}
\newsym[Neural Network output perceptron]{symO}{O}
\newsym[LSTM peephole weight vector]{symvp}{\overrightarrow{p}}
\newsym[Electrical charge]{symq}{q}
\newsym[GRU reset gate vector]{symvr}{\overrightarrow{r}}
\newsym[Electrical resistance]{symR}{R}
\newsym[Time]{symt}{t}
\newsym[Electrical voltage]{symv}{v}
\newsym[Neural Network weight]{symw}{w}
\newsym[Neural Network weight matrix]{symmw}{\mathit{W}}
\newsym[Neural Network input vector]{symvx}{\overrightarrow{x}}
\newsym[GRU update gate vector]{symvz}{\overrightarrow{z}}
\newsym[Electrical flux]{symphi}{\phi}
\newsym[Memristor's internal conductance, memductance]{symW}{\kappa}
\closesymdef
\markasused{omult}


\begin{document}

\title{Memristor-based recurrent modules for neural computing}


\author{\IEEEauthorblockN{Valentin BARBAZA}
%\IEEEauthorblockA{\textit{Master student} \\
%\textit{name of organization (of Aff.)}\\
Lisbon, Portugal \\
valentin@barbaza.org}


\maketitle

\begin{abstract}
  In this work we propose an analog structure for memristor based recurrent modules targeting neural computing. The system is fully analog and implements a working \ac{LSTM} circuit block and a work in progress \ac{GRU} circuit block. Both of those blocks contain memristors to be used as weights in a analog \ac{VMM} capable circuit. These circuit blocks allow to run very fast computation of \aclp{RNNs} of any size, in a relatively small integrated circuit. As part of the \ac{LSTM} and \ac{GRU} blocks, an analog activation function circuit was designed. This specific circuit is capable of reproducing sigmoid and \ac{tanh} like functions, with similar shapes and the same output ranges. The work also include the implementation of a memory cell used to store an analog value for a short period of time. The \ac{LSTM} block can be serialized or not with the ability to choose the level of serialization. Serializing the system allows to save onChip area at the cost of execution time. To the author's knwoledge this is the first analog implementation of the behavior of \acs{C. elegans} using the \ac{LSTM} block. Such an analog system provides ground for real time implementation of nervous systems.
\end{abstract}

\begin{IEEEkeywords}
  Memristors, Crossbar array, VMM, Machine Learning, RNN, LSTM, GRU, Analog computing, Analog circuits, Analog activation functions, Tensorflow, Keras, Embedded systems, Neural computing
\end{IEEEkeywords}

\section{Introduction}

AIs (Artificial Intelligences) are one of the main research domain in computer sciences and is used in other scientific domains. One great application of AI, is in embedded systems. Lots of small electronic devices around us could benefit us. \textbf{Google} and \textbf{Apple} are pushing AI in phones with their in house designed ARM processors having, very advertised, tensor cores.
Being able to run low powered AIs is thus one of the biggest computer objective of the deceny.
Analog computers are known to be a low powered technology that most importantly give almost instant results. An analog ASIC (Application Specific Integrated Circuit) could then be fabricated for specific embedded applications. This would definitely improve digital technologies in terms of time, and maybe in power consumption as well, these being two prerequisite for embedded system use.
For this reason this work aims at creating a functioning software simulation of analog circuit capable of running AI. The thesis will focus on the software simulation, the first step of the work in order to have a fully working chip able to run AI algorithms.

\section{State of the art}\label{sec:state}

\subsection{Recurrent Neural Networks (RNN)}

RNNs are a family of neural networks that differentiate themselves by having feedback connections. It is often used with sequences of data \cite{rnn}, sometimes, with varying amount of input data. RNNs are used for handwriting recognition and language translation \cite{gru}.

The feedback connection is caracterized by the hidden state vector ($\overrightarrow{h_t}$) which serves as the output and as part of the input for the next time step, a RNN is defined as \cref{eq:rnn}.

\begin{equation}\label{eq:rnn}
  \overrightarrow{h_t}=f(\overrightarrow{x_t},\overrightarrow{h_{t-1}})
\end{equation}

The simple version of the RNN is defined by \cref{eq:srnn}.

\begin{equation}\label{eq:srnn}
  \overrightarrow{h_t}=tanh([\overrightarrow{x_t},\overrightarrow{h_{t-1}}]\cdot w + \overrightarrow{b})
\end{equation}

Where ($w$,$\overrightarrow{b}$) are the pair of weights matrix and bias vectors. $\overrightarrow{x_t}$ is the input vector and $\overrightarrow{h_t}$ is the hidden state vector.

The graphical representation of the simple RNN cell is shown in \cref{fig:rnnCell}.

\begin{figure}[t]
  \centering
  \begin{minipage}{\columnwidth}
    \subfloat[Simple \acs{RNN} cell\label{fig:rnnCell}]{\includesvg[width=\columnwidth]{rnn/rnnCell.svg}}
  \end{minipage}
  \begin{minipage}{\columnwidth}
    \subfloat[Legend\label{leg:cells}]{\includesvg[width=\columnwidth,pretex=\footnotesize]{cellsLegend.svg}}
  \end{minipage}
  \caption{}
\end{figure}

\subsection{Long Short Term Memory (LSTM)}

LSTMs are a type of RNN used to solve the vanishing gradiant problem \cite{firstLSTM}. It was improved a few times before becoming the modern LSTM \cite{improvLSTM}. The LSTM differs from a simple RNN because of its second feedback variable being the cell state ($\overrightarrow{c_t}$).

The LSTM contains four activated gates, that each serve its own purpose. The input gate \cref{eq:inputG} controls whether the cell state is updated, the forget gate \cref{eq:forgetG} that determines how the current cell state is affected by the old cell state, the output gate \cref{eq:outputG} that controls how much the hidden state is affected by the cell state. The candidate cell state gate \cref{eq:candCell} computes the change in the future cell state.

\begin{equation}\label{eq:inputG}
  \overrightarrow{i_t}=\sigma ([\overrightarrow{x_{t_1}},\overrightarrow{h_{t-1}}]\cdot w_i + \overrightarrow{b_i})
\end{equation}
\begin{equation}\label{eq:forgetG}
  \overrightarrow{f_t}=\sigma ([\overrightarrow{x_{t_1}},\overrightarrow{h_{t-1}}]\cdot w_f + \overrightarrow{b_f})
\end{equation}
\begin{equation}\label{eq:outputG}
  \overrightarrow{o_t}=\sigma ([\overrightarrow{x_{t_1}},\overrightarrow{h_{t-1}}]\cdot w_o + \overrightarrow{b_o})
\end{equation}
\begin{equation}\label{eq:candCell}
  \overrightarrow{\tilde{c}_t}=tanh([\overrightarrow{x_{t_1}},\overrightarrow{h_{t-1}}]\cdot w_c+ \overrightarrow{b_c})
\end{equation}

The next part of the LSTM consist of computing the new cell state in the following equation :

\begin{equation}\label{eq:cellS}
  \overrightarrow{c_t}=\overrightarrow{f_t}\odot \overrightarrow{c_{t-1}} + \overrightarrow{i_t} \odot \overrightarrow{\tilde{c}_t}
\end{equation}

The hidden state is then determined using the current cell state \cref{eq:hiddenS}.

\begin{equation}\label{eq:hiddenS}
  \overrightarrow{h_t}=\overrightarrow{o_t}\odot tanh(\overrightarrow{c_t})
\end{equation}

Where ($w_i$,$\overrightarrow{b_i}$), ($w_f$,$\overrightarrow{b_f}$), ($w_o$,$\overrightarrow{b_o}$) and ($w_c$,$\overrightarrow{b_c}$) are the pair of weights matrixes and bias vectors for the input, forget, output and candidate cell gates respectively. $\overrightarrow{x_t}$ is the input vector and $\overrightarrow{h_t}$ is the hidden state vector.

The graphical representation of an LSTM cell is found in \cref{fig:lstmCell}.

\begin{figure}[t]
  \centering
  \includesvg[width=\columnwidth]{lstm/lstmCell.svg}
  \caption{\acs{LSTM} cell, adapted from \cite{wikiLSTM}\label{fig:lstmCell}}
\end{figure}

\subsection{Gated Recurrent Units (GRU)}

GRUs are another type of RNN. It is also known to reduce the effect of the vanishing gradient problem. It was first introduced to improve translation techniques \cite{gru}.

The \ac{GRU} is very often compared to the \ac{LSTM}, being sometimes assimilated as a type of \ac{LSTM} \cite{nbLSTM}. Their performance was found to be very similar in most situations \cite{gruVSlstm}, making those two types of \acp{RNN} coexistant in the modern machine learning world.

There are two versions of the \ac{GRU}, both are found on the internet, they are known as the encoder and decoder version \cite{gru}. They were originally designed to encode the message to translate and then decode in the translation. PyTorch only supports the decoder version \cite{gruPyTorch}, while the Keras library supports both \cite{gruKeras} chosen by changing an argument. This work only uses the encoder version, for that reason, it will be the only one described.

It contains an update gate \cref{eq:updateG}, a reset gate \cref{eq:resetG}, a candidate activation gate \cref{eq:candActivG}. The hidden state is then computed \cref{eq:gruHidG} using the previous results.

\begin{equation}\label{eq:updateG}
  \overrightarrow{z_t}=\sigma ([\overrightarrow{x_t},\overrightarrow{h_{t-1}}] \cdot w_z + \overrightarrow{b_z})
\end{equation}
\begin{equation}\label{eq:resetG}
  \overrightarrow{r_t}=\sigma ([\overrightarrow{x_t},\overrightarrow{h_{t-1}}] \cdot w_r + \overrightarrow{b_r})
\end{equation}
\begin{equation}\label{eq:candActivG}
  \overrightarrow{\hat{h_t}}=tanh(\overrightarrow{x_t}\cdot w_{hx}+(\overrightarrow{r_t}\odot\overrightarrow{h_{t-1}}) \cdot w_{hh} + \overrightarrow{b_h})
\end{equation}
\begin{equation}\label{eq:gruHidG}
  \overrightarrow{h_t}=(\overrightarrow{1}-\overrightarrow{z_t})\odot \overrightarrow{h_{t-1}} + \overrightarrow{z_t}\odot \overrightarrow{\hat{h_t}}
\end{equation}

Where ($w_z$,$\overrightarrow{b_z}$), ($w_r$,$\overrightarrow{b_r}$),($w_{hx}$,$w_{hh}$,$\overrightarrow{b_h}$) are the weights matrixes and bias vectors for the update, reset and candidate activation gates respectively.

A visual representation of the encoder \ac{GRU} cell is available in \cref{fig:encoderGruCell}.

\begin{figure}[t]
  \centering
  \includesvg[width=\columnwidth]{gru/encoderCell.svg}
  \caption{Encoder \acs{GRU} cell, legend in \cref{leg:cells}\label{fig:encoderGruCell}}
\end{figure}

\subsection{Memristors}

Memristors are the lesser known fourth fundamental passive component of electronics, among resistors, capacitors and inductor.
It was first theorized in 1971 as a missing fundamental component in \cite{TheoMemristor}. The name comes from the blend of \textit{memory} and \textit{resistance}.
The missing component linking the four fundamental circuit variables, voltage ($v$), charge ($q$), current ($i$) and flux ($\phi$). \Cref{fig:fundComp} shows the four fundamental variables are on each side of the square, with the ones on opposite sides being linked by the following equations :

\begin{equation}
  d\phi = v\cdot dt
\end{equation}

\begin{equation}
  dq = i\cdot dt
\end{equation}

Resistors, capacitors and inductors were already very established and well known components, so it was theorized that a fourth device should then exist to physically link flux ($\phi$) and charge ($q$).  The flux in this case is not a magnetic flux and is defined as such : $ d\phi=v\cdot dt \Rightarrow \phi =  \int v \,dt  $.

The component stayed theoretical until 2008 when it was implemented in a physical device for the first time \cite{memristorFab}. It took 37 years to have an actual working device.

An extension to the memristor, reffered to as the memristive device, was theorized in 1976 \cite{memrestiveDev}. The difference between the memristor and the memristive devices is its internal behavior. Memristive devices are commonly referred to as memristors as well.

\begin{figure}[t]
  \centering
  \includesvg[width=0.65\columnwidth,pretex=\tiny]{memristor/memristor}
  \caption{Fundamental passive components, adapted from \cite{memWiki}}
  \label{fig:fundComp}
\end{figure}

A memristor is used for its ability to change its internal resistance based on the current that flowed through it.

\subsection{Memristors Crossbar Arrays}

Setting memristors in a crossbar array allows to perform analog Vector Matrix Multiplication (VMM), also called Multiply and Accumulate. \Cref{fig:crossbar} shows what a typical crossbar array looks like.

\begin{figure}[b]
  \centering
  \includesvg[width=.6\columnwidth,pretex=\small]{crossbar/crossbar}
  \caption{Crossbar array schematics, inspired from \cite{xbarFigures}}
  \label{fig:crossbar}
\end{figure}

The circuit uses physical properties of electrical systems to perform analog computation. The following part will focus only on the circuit node in \cref{fig:crossNode}.

\begin{figure}[t]
  \centering
  \includesvg[wdith=0.4\columnwidth,pretex=\scriptsize]{crossbar/node}
  %\def\svgheigth{3.5cm}
  %\input{crossbar/node.pdf_tex}
  \caption{Memristor crossbar node of the $i^{th}$ line and $j^{th}$ column}
  \label{fig:crossNode}
\end{figure}

A voltage is applied on the $i^{th}$ line, and because every column is virtually grounded, the voltage applied to the memristor, with resistance $R_i$, is $V_i$. By applying Ohm's law, we know that the current flowing into the memristor ($I_{i}$) is bound by the following equation :

\begin{equation}
  V_i = R\cdot I_{i} \Rightarrow I_{i} = V_i\cdot (\frac{1}{R_i})= V_i\cdot\sigma_i
\end{equation}
With $\sigma_i$ being the conductance of memristor, defined as $\sigma_i=\frac{1}{R_i}$.

This line then joins the column where a current of $I_{i,j-1}$ is flowing, then according to Kirchhoff's current law the resulting current is :
\begin{equation}
  I_{j,i} = I_{j,i-1}+I_{i} = I_{j,i-1} + V_i\cdot\sigma_i
\end{equation}
Unfolding the equations will give the current at the bottom of the column, for example, the current at the bottom of the first column in \cref{fig:crossbar} is :
\begin{equation}
  I_1=\sigma_1\cdot V_1 + \sigma_2\cdot V_2 + \sigma_3\cdot V_3
\end{equation}
With $\sigma_1$, $\sigma_2$ and $\sigma_3$ being the conductance of the 3 memristors in the first column.

\section{Circuits}

The system will be working with a $V_{dd}$ of $1.8V$. Such a value was chosen because this is a low power system.

The way values are encoded in the analog system will be descibed here as it serves for the entire thesis.
In order for the system to support negative numbers we're going to use a $V_{cm}$ set to $\frac{V_{dd}}{2}$. That means that $V_{cm}=\frac{V_{dd}}{2}=0.9V$. This $V_{cm}$ will then describe a zero. A step of one was chosen to be $0.1V$ in the analog circuit.
\Cref{tab:valConv} shows the conversions from a real number to its voltage equivalent.

\begin{table}[b]
  \caption{Real/Voltage Conversion Table.}
  \label{tab:valConv}
  \centering
  \begin{tabular}{|c|c|}
    \hline
    \textbf{Real value} & \textbf{Voltage} \\
    \hline
    $0$ & $0.9V$ \\
    \hline
    $1$ & $1.0V$ \\
    \hline
    $x$ & $V(x)=\frac{x}{10}+V_{cm}$\\
    \hline
    $real(v)=(v-0.9)\cdot 10$ & $v$\\
    \hline
  \end{tabular}
\end{table}

Since the system cannot reach voltage outside of the operating range with the intended behavior, the voltage is then restricted to $V\in [0,1.8]$. This means that the range of real value that the systems can handle is $x\in [-9,9]$.

\subsection{Activation functions}\label{subsec:af}

Producing analog activation functions is quite important as using hard sigmoid or hard hyperbolic tangent (tanh) functions impacts the results \cite{af,hardSigm}.

The analog activation circuit thus plays an important role in the final result's quality.

The circuit used is the same as the one in \cite{thesisRef}, and is shown in \cref{circt:af}. The circuit's technology used being different, all the parameters had to be determined empirically to best fit a sigmoid shape. The parameters can be found in \cref{tab:afPar}.

Due to the nature of the functions we want to generate, we will use the same circuit for both a sigmoid and a \ac{tanh} like functions. The two different functions are generated by changing two parameters. The functions generated are the same shape and only differ by their output range.

\begin{figure}[t]
  \centering
  \includesvg[width=\columnwidth,pretex=\small]{activation/afCircuit}
  \caption{Activation functions circuit}
  \label{circt:af}
\end{figure}

\begin{table}[b]
  \caption{Circuits parameters}
  \label{tab:afPar}
  \centering
  \begin{tabular}{|c|c|c|}
    \hline
    \textbf{Parameter} & \textbf{Sigmoid} & \textbf{\ac{tanh}} \\
    \hline
    $V_1$ & \multicolumn{2}{c|}{$1.1V$}\\
    \hline
    $V_2$ & \multicolumn{2}{c|}{$635mV$}\\
    \hline
    $V_3$ & $0.8V$ & $550mV$\\
    \hline
    $i_{dc}$ & \multicolumn{2}{c|}{$150uA$}\\
    \hline
    $w$ & \multicolumn{2}{c|}{$900nm$}\\
    \hline
    $l$ & \multicolumn{2}{c|}{$60nm$}\\
    \hline
    $R_1$ & \multicolumn{2}{c|}{$5k\Omega$}\\
    \hline
    $R_2$ & \multicolumn{2}{c|}{$10k\Omega$}\\
    \hline
    $R_3$ & $2k\Omega$ & $4k\Omega$\\
    \hline
  \end{tabular}
\end{table}

Here, $w$ and $l$ are, respectively, the width and length of the two NMOS of the circuit.

The outputed voltage depends on the input passed on. The results obtained by are rather convincing and can be found in \cref{graph:af}. \Cref{graph:af} also shows the Root Mean Square Error (RMSE) of the analog results with the ideal result.

\begin{figure}[t]
  \centering
  \includesvg[width=\columnwidth, pretex=\scriptsize]{activation/afGraph}
  \caption{Input/Output graph of the activation function circuit for both sigmoid and \acs{tanh} functions}
  \label{graph:af}
\end{figure}

The symbols for the sigmoid and the \ac{tanh} are separated for better understanding and are available in \cref{sym:af}.

\begin{figure}[b]
  \centering
  \hspace*{0.8cm}
  \subfloat[Sigmoid symbol]{\includesvg[height=1cm,pretex=\small]{activation/sigmoidSymbol}}%
  \hfill
  \subfloat[\ac{tanh} symbol]{\includesvg[height=1cm,pretex=\small]{activation/tanhSymbol}}%
  \hspace*{0.8cm}
  \caption{Activation functions symbols with the input and output pins on either side depending on the flow of the current for better readability}
  \label{sym:af}
\end{figure}

These functions are still a bit different from the original functions (especially for the \ac{tanh}). However that does not matter too much as the trainning will be happening with the extracted analog functions, all weights will be set in the circuit. This is the reason why such a difference does not matter. As long as the curves have the similar shape, the result will not be drastically affected.

\subsection{Memory cell}

The memory cell is a circuit that is able to store an analog value for a limited time. It works using capacitors that have the ability to store a voltage for a short period of time.

The circuit is shown in \cref{circt:memcell}. The voltage is stored in the capacitor and is kept using \ac{CMOS} switches.

The \ac{CMOS} switches found in the circuit have width of $w=200nm$ and a length of $l=60nm$.

\begin{figure}[t]
  \centering
  \includesvg[width=\columnwidth,pretex=\small]{memcell/memCircuit}
  \caption{Memory cell circuit}
  \label{circt:memcell}
\end{figure}

%Remove first if too big
The cirucit has a two \ac{CMOS} switches design to avoid voltage leakage through the swicthes. The system has voltage leakage when only one \ac{CMOS} swicth, and thus leads to a large memory leak. This is due to the high voltage difference between the two sides of the \ac{CMOS} swicth. Using two \ac{CMOS} switches allows for this difference to be mitigated (\cref{fig:memcellLoss}).

\begin{figure}[b]
  \centering
  \includesvg[width=\columnwidth,pretex=\scriptsize]{memcell/data-loss}
  \caption{Memory conservation in a memory cell with 1 \ac{CMOS} switch vs 2 \ac{CMOS} swicthes}
  \label{fig:memcellLoss}
\end{figure}

The symbol, shown in \cref{sym:memcell}, for this circuit is designed to show a capacitor because it is its memory mechanism.

\begin{figure}[t]
  \centering
  \includesvg[height=1cm,pretex=\small]{memcell/memcellSymbol}
  \caption{Memory cell symbol with the input enable pin (top) and the output enable pin (bottom). The left and right pins are interchangebly the input and ouput for the sake of readability}
  \label{sym:memcell}
\end{figure}

\subsection{Voltage-driven crossbar circuit}\label{subsec:xbarCircuit}

The crossbar circuit theory has already been explained in the \cref{sec:state}. The actual implementation in the circuit is described here. The circuit is the one in \cref{circt:xbar}. The circuit depends on three parameters :

\begin{itemize}
  \item $n_i$ : The number of input for our crossbar array (not including bias for a more general circuit).
  \item $n_o$ : The number of parallel output for our crossbar array.
  \item $n_s$ : The serial size of our crossbar system.
\end{itemize}

\begin{figure}[b]
  \centering
  \includesvg[width=\columnwidth,pretex=\scriptsize]{crossbar/crossbarUse}
  \caption{Circuit of the crossbar array used in the final system ($n_i$, $n_o$, $n_s$)}
  \label{circt:xbar}
\end{figure}

In this work, the choice was made to use a two memristor per synapse architecture. Using two memristor per synapse doubles the area but doubles the weight range \cite{doubleMem} and allows to easily use negavite weights. The output voltage will be centered around $V_{cm}$ and be compliant with the standard set in \cref{tab:valConv}.

This design is the one that is used in \cite{doubleMem}. Let's assume that a given memristor has a resistance range of $R\in[R_{min},R_{max}]$, that means its conductance range is $\sigma \in [\sigma_{min},\sigma_{max}]$ (with $\sigma_{min}= \frac{1}{R_{max}}$ and $\sigma_{max}= \frac{1}{R_{min}}$). This design works using two \ac{opAmp} connected to $V_{cm}$ with the positive pin and the negative pin to the output of the crossbar array. \Cref{eq:doubleMem0,eq:doubleMem1,eq:doubleMem2} are describing how this architecture works. A simplified version of the double memristors per synapse circuit is also available in \cref{circt:doubleMem}.

\begin{figure}[t]
  \centering
  \includesvg[width=\columnwidth,pretex=\tiny]{crossbar/doubleMem}
  \caption{Simplified circuit of a double memristor per synapse architecture}
  \label{circt:doubleMem}
\end{figure}

Using $x_k$ as the voltage for the input line $k$. The highest \ac{opAmp} is identified as $opAmp_0$ and the lowest $opAmp_1$.

For the sake of simplicity, the following equations considers the ground to be $V_{cm}$.

\begin{equation}
  \label{eq:doubleMem0}
  V_{opAmp_0}=-R_r\cdot i_+
\end{equation}
\begin{equation}
  \label{eq:doubleMem1}
  i_{R_f}=i_-+\frac{V_{opAmp_0}}{R_r}=i_--i_+
\end{equation}
\begin{equation}
  \label{eq:doubleMem2}
  \begin{array}{2}
    V_{opAmp_1}=y_0=-R_f\cdot(i_--i_+)=R_f\cdot(i_+-i_-)\\
    =R_f\cdot\sum_{k=0}^n(\sigma_{k+}-\sigma_{k-})\cdot x_k
  \end{array}
\end{equation}
With $i_+=\sum_{k=0}^n\sigma_{k+}\cdot x_k$ and $i_-=\sum_{k=0}^n\sigma_{k-}\cdot x_k$.

This circuit has the option to be serialized with varying degrees. The idea of serializing the circuit came from \cite{thesisRef}. Serializing the circuit reduces the number of components required and thus reduces the final onChip area. Serializing the system increases the time it takes to compute the output.

Serializing the system means not computing all values of the output vector at the same time, but instead computing group by group, the groups' size are $n_o$. The first output group is computed during $e_0$ and the $i^{th}$ group is computed during $e_i$. The timing of when the outputs are available is found in \cref{tim:serpar}. Those flag control the \ac{CMOS} swicthes present in \cref{circt:xbar}, the switches control which output group is outputed.

The \ac{CMOS} switches are here to open the necessary input gates when the output is required, the \ac{CMOS} switches are controlled as in \cref{tim:serpar}.

\begin{figure}[b]
  \centering
  \begin{tikztimingtable}
    $e_0$ & x2H2L N(A1) L N(A2) 2L N(A3) L N(A4) 2Lx\\
    $e_1$ & x2L2H N(B1) L N(B2) 2L N(B3) L N(B4) 2Lx\\
    $e_i$ & x4L N(C1) L N(C2) 2H N(C3) L N(C4) 2Lx\\
    $e_{n_s-1}$ & x4L N(D1) L N(D2) 2L N(D3) L N(D4) 2Hx\\
    %foo & 2L N(A1)  4H N(A2) L\\
    \extracode
    \node[gap, at={($(A1|-A2)!0.5!(A2)$)}];
    \node[gap, at={($(A3|-A4)!0.5!(A4)$)}];
    \node[gap, at={($(B1|-B2)!0.5!(B2)$)}];
    \node[gap, at={($(B3|-B4)!0.5!(B4)$)}];
    \node[gap, at={($(C1|-C2)!0.5!(C2)$)}];
    \node[gap, at={($(C3|-C4)!0.5!(C4)$)}];
    \node[gap, at={($(D1|-D2)!0.5!(D2)$)}];
    \node[gap, at={($(D3|-D4)!0.5!(D4)$)}];
    \tablerules
    %\draw (0,0) circle (2pt); % Origin
    \begin{pgfonlayer}{background}
      \vertlines[help lines]{0.55,10.55}
      %\vertlines[red]{1.6,5.6,15.6}
      %\vertlines[blue]{3.6,9.6,15.6}
    \end{pgfonlayer}
  \end{tikztimingtable}
  \caption{Enable flags timing for any value of $n_s$ in a single time step}
  \label{tim:serpar}
\end{figure}

When the system is used fully in parallel, the \ac{CMOS} switches are not required can then be removed to lower the final onChip area.

In this work, the analog system will be simulated for the inference of the \ac{NN}, thus the weights will not have to change during the simulation. Because the weights are represented in the analog circuit by the internal resistances of the memristors, the memristors can be replaced by resistors with a set resistance for the simulation.

The symbol (\cref{sym:xbar}) defined for the voltage based memristor crossbar array used in this thesis is more compact and helps the readability of the circuits that require a crossbar array. It depends on several parameters, the number of inputs ($n_i$), the number of outputs ($n_o$) and the serial size ($n_s$).

\begin{figure}[t]
  \centering
  \includesvg[height=1cm]{crossbar/xbarSymbol}
  \caption{Symbol used for the crossbar array, the input pin is a bus of size $n_i$ and the output pin is a bus of size $n_o$}
  \label{sym:xbar}
\end{figure}

\subsection{Verilog-A models}

This work uses Verilog-A components to replace component that could not be designed due to lack of time. Those components are the operational amplifier (opAmp) and a voltage multiplier.

\subsubsection{Operational amplifier}

This component is the very famous \ac{opAmp}. This specific component was not designed for the thesis, it required a current range that did not allow to use ones already made by members of the research group. The choice was then to use a verilog-A model, and then design one if time allows it.

\begin{equation}
  \label{eq:opAmp}
  V_{out}=\mu \cdot (V_+-V_-)
\end{equation}

\Cref{eq:opAmp} is the equation used in the verilog-A model. It makes it so it behaves like an ideal \ac{opAmp}. For this thesis $\mu$ has been set to $\mu=10^5$.

\subsubsection{Voltage multiplier}

This component while far less popular than the latter, is just as useful for our specific use. It allows us to multiply, as its name implies, two voltages. It is used to compute the pointwise multiplications of the \ac{LSTM} or \ac{GRU} (\cref{fig:lstmCell,fig:encoderGruCell}).

In oder for the circuit to be compliant with the real value to voltage conversion (\cref{tab:valConv}), a multiplication needs to be as in \cref{eq:finalVoltMult}.

\begin{equation}\label{eq:finalVoltMult}
  V_{out}=10\cdot(V_{in_1}-V_{cm})\cdot (V_{in_2}-V_{cm}) + V_{cm}
\end{equation}

This is taken care of using in reality two parts, the actual voltage multiplier (\cref{eq:voltMult}) and a non inverting amplifier (\cref{eq:invAmp}).

\begin{equation}\label{eq:voltMult}
  V_{voltMult}=-(V_{in_1}-V_{cm})\cdot (V_{in_2}-V_{cm}) + V_{cm}
\end{equation}
\begin{equation}\label{eq:invAmp}
  V_{out}=-(V_{voltMult}-V_{cm})\cdot10+V_{cm}
\end{equation}

Where $V_{out}$ is the output voltage the inverting amplifier, $V_{voltMult}$ is the out voltage of the voltage multiplier itself and $V_{in_1}$ and $V_{in_2}$ are the input voltages.

The \cref{eq:voltMult} is assumed possible because of the actual voltage multiplier's datasheet available at \cite{actualVoltMult}.

The symbols for those two components are the ones in \cref{sym:models}.

\begin{figure}[t]
  \centering
  \hspace*{1.5cm}
  \subfloat[\ac{opAmp}'s symbol\label{sym:opAmp}]{\includesvg[height=1cm]{models/opAmpSymbol}}%
  \hfill
  \subfloat[Voltage multiplier's symbol\label{sym:voltMult}]{\includesvg[height=1cm]{models/voltMultSymbol}}%
  \hspace*{1.5cm}
  \caption{Symbols used for the verilog-A components}
  \label{sym:models}
\end{figure}

The \ac{opAmp} uses its IEEE symbol (\cref{sym:opAmp}) while the voltage multiplier uses a custom symbol (\cref{sym:voltMult}).

\subsection{LSTM analog implementation}

This section describes the circuit of an \ac{LSTM} with an input vector of size $n_i$, a $n_h$ hidden states, a serial size of $n_s$ and $n_{ts}$ time steps. $n_o=n_h/n_s$ is going to be used for future references in this section. In order for the crossbar array to be used $n_o$ must be an integer, in other words, $n_s$ must divide $n_h$. The circuit used for the LSTM is shown in \cref{circt:lstm}.

\begin{figure*}[h]
  \centering
  \includesvg[width=\textwidth,pretex=\tiny]{lstm/lstmCircuit}
  \caption{\ac{LSTM} circuit}
  \label{circt:lstm}
\end{figure*}

The system is built using the crossbar array from \cref{subsec:xbarCircuit} with ($n_i+n_h+1$,$n_o$, $n_s$) as parameters.

\begin{itemize}
  \item $\overrightarrow{z_t}$ is the input of the crossbar but not the input of the \ac{LSTM}. It is defined by $\overrightarrow{z_t}=(\overrightarrow{x_t},\overrightarrow{h_{t-1}},\overrightarrow{b})$.
  \item $e_{j,0}$ and $e_{j,1}$ are two enable flags that respectively represent the first and second half of $e_j$.
  \item $e_{in}$ and $e_{out}$ are the flags used to enable the hidden state values to go to the input (feedback connection) or to the output of the circuit.
  \item $e_{next}$ is the enable flag on in between two time steps.
  \item $R_{amp0}$ and $R_{amp1}$ are the two resistances used to amplify the output voltage of the voltage multipliers. Their value is set so that $\frac{R_{amp1}}{R_{amp0}}=10$.
\end{itemize}

The wires coming into the crossbar are a bus of size $n_i+n_h+1$ and the output of the crossbar is a bus of size $n_o$ (\cref{sym:xbar}). This is why everything in the system apart from the crossbar arrays is only shown once in \cref{circt:lstm} but in reality those components are present $n_o$ times. Those extra components are needed in order for the parallel channels to work.

The circuit uses two memory cells for its feedback connections not to overwrite the value of the current stored value by the value of the next time step.

The LSTM circuit can be serialized allowing to reduce the number of pointwise components in the circuit by a factor $n_s$. However serializing the circuit increases the times it takes to compute the outputs by a factor of $n_s$.

\subsection{GRU analog implementation}

This section describes the circuit of an encoder \ac{GRU} with an input vector of size $n_i$, a $n_h$ hidden states and $n_{ts}$ time steps. The decoder \ac{GRU} could also be implemented in an analog circuit, but the choice was made to focus on the encoder \ac{GRU}. The \ac{GRU} is by its nature very similar to the \ac{LSTM}. For that reason it has a similar circuit to LSTM circuit (\cref{circt:lstm}). The system is built, once again, using crossbar array with ($n_i+n_h+1$,$n_h$, $1$) as parameters.

The GRU equations requires to compute the function defined in \cref{eq:1minus0}.
\begin{equation}\label{eq:1minus0}
  f(x)=1-x
\end{equation}
The solution found to do compute the operation consist of using an \ac{opAmp} as an inverter around $V_{inv}$, the formula is available in \cref{eq:1minus1}.
\begin{equation}\label{eq:1minus1}
  f_v(v)=-(v-V_{inv})+V_{inv}=2\cdot V_{inv} -v
\end{equation}
With $V_{inv}=V_{cm}+0.05$.

The GRU circuit cannot be serialized. This is because the candidate hidden state requires a fully computed reset vector to be computed.

\section{Python tools}

\subsection{Weights generation}

The weigths were trained in python \cite{python} using the tensorflow library \cite{tensorflow}.

The weights need to be trained. The training is performed using the analog activation functions generated by the circuits in \cref{subsec:af}. The circuit is thus trained as if it were trained directly on the chip, circuit's inaccuracies aside.

The weights are constrained to make sure the voltage is within the active voltage range of the circuit. If the voltage gets out of range, the weights will not be correct as the voltage cannot get out of range in the physical circuit.

Once the weights are trained, they are saved in a file, that will later serve as the basis to generate the netlist.

The code used to generate the weights is available on my github repo \cite{lstmWei}.

\subsection{Netlist generation}

In order to be able to run the simulation with different kinds of \ac{NN} architecture of varying sizes. The point of the part is thus to explain how the netlist generator tool works.

The netlist is generated using a python script to generate a SPICE netlist. It can generate the netlist for different kinds of circuits. It can generate both LSTM and GRU circuits, with any size of input, time steps or even number of hidden states. When it is possible the serial size can also be selected.

The weights are stored in a single file. This file contains a brief description of the architecture being used. The first index stores this desciption. The following indexes are for the weights of each layer, in the order given in the description. The weights are stored in a python list and are distributed among the different resistances.

All the code used to generate the netlist can be found on my github page \cite{lstmGen}.

\section{The datasets}\label{sec:dataset}

\subsection{The format}

The values for the data needs to be set not to exceed 1. This is to make sure that the terminal voltage doesn't reach the threshold as explained in \cref{sec:xbarCircuit}.

\subsection{Airline}

The first dataset contains a time series of international airline passengers from January 1949 to December 1960, the data is recorded monthly and is in thousands. The dataset then contains twelve years of monthly data so $144$ sample points. The results of such a problem aren't useful, it was used in this thesis because of its simplicity to check whether the analog circuit is working. It is extracted from the tutorial available at \cite{airline}. The dataset is available at \cite{datasets}, the name of the file containing the dataset is \textit{airline-passengers.csv}.

This is a regression problem, the role of the \ac{LSTM} is to predict the number of passenger flying the next month being given previous months' passenger count.

\begin{figure}[H]
  \centering
  \includesvg[width=\textwidth]{datasets/airline}
  \caption{the airline dataset in a grphic. This is the curve that the regression is trying to reproduce. The vertical lines represent a full year.}
  \label{graph:airline}
\end{figure}

\subsubsection{Data format}

The dataset contains $144$ sample points. It has been transformed into $142$ data points for training. This has been done by taking each values three by three. The first two values are two timesteps for the input vector and the third value being used the target value. Two third of the dataset is being used for training and the other third is used for validation.

\subsubsection{Network configuration}

The layers used to solve this problem are listed below :

\begin{itemize}
  \item An \ac{LSTM} with four hidden states ($n_h=4$) and an input with feature size of one and two time steps.
  \item A Dense layer with an output size of one.
\end{itemize}

\Cref{fig:airlineModel} is a graphical representation of the model just described.

\begin{figure}[H]
  \centering
  \includesvg[width=\textwidth]{datasets/airlineModel}
  \caption{Model used to solve the airline passengers problem}
  \label{fig:airlineModel}
\end{figure}

\subsubsection{Results} %TODO Move to results obviously

The weights are trained using both the custom activation function (\cref{sec:genwei}) and without ones and compare their output.

After training on a digital computer using the settings described in \cref{sec:genwei}.

\subsection{\ac{C. elegans}}

This dataset is far more interresting than the latter. This data set aims to use \acp{LSTM} to mimic the behavior of real neurons. As explained in \cite{celegans}, \acf{C. elegans} are simple organisms that are getting very popular for whole brain organization studies. The point of this problem is to reproduce the nervous system of the \ac{C. elegans}. This is done using recorded data of the input of 4 neurons and the output of 4 other neurons.

The dataset is great for our study because :
\begin{itemize}
  \item It comes from a very recent paper, and means the research is happening right now.
  \item It aims at reproducing the behavior of the brain of a simple organism (\ac{C. elegans}), domain in which having a small and low powered system to mimic is an advantage.
  \item The paper was written by Dr. Barbulescu who is one of my supervisor.
\end{itemize}

\section{Results}

The target for our circuit is to reproduce, as closely as possible the digital prediction.

\subsection{Airline passengers}

The problem is solved using an LSTM or GRU with four hidden state connected to a Dense layer with an output size of one. The weights were trained for three hundreds epochs.

\begin{figure}[h]
  \centering
  \includesvg[width=\columnwidth, pretex=\scriptsize]{results/airline/analog}
  \caption{Analog predictions trained with the analog activation functions.}
  \label{graph:airlineAnalog}
\end{figure}

The analog results obtained in \cref{graph:airlineAnalog} are very close to the digital predictions. This is confirmed by the values in \cref{tab:airlineAnalog}. The results the closest to the target are the ones generated using a serial size of one ($n_s=1$). Indeed, its error ($28.4$) is lower than the error from the digital results ($52.3$) to the original target values, almost half of it. The other ones are still quite close to the targeted curve, but are still lower. The errors are higher when running the system in serial mode. This higher inaccuracy is probably due to the data staying for longer in the memory cells as the time steps get longer.

\begin{table}[h]
  \caption{\acp{RMSE} of each analog prediction to their associated digital prediction}
  \label{tab:airlineAnalog}
  \centering
  \begin{tabular}{|c|c|c|c|c|}
    \cline{2-5}
    \multicolumn{1}{c}{}& \multicolumn{3}{|c|}{Analog prediction} &Target\\
    \cline{1-4}
    Serial size & $n_s=1$ & $n_s=2$ & $n_s=4$ & data\\
    \hline
    Digital prediction & $28.4$ & $92.6$ & $68.5$ & $52.3$\\
    \hline
  \end{tabular}
\end{table}

The GRU circuit not being fully working produces very off predictions. For this reason, the GRU results are not included.

\subsection{C. elegans}

The problem is solved using an LSTM or GRU with eights hidden state connected to a Dense layer with an output size of four. The weights were trained for one thousand epochs.

\begin{figure}[h]
  \centering
  \includesvg[width=\columnwidth,pretex=\scriptsize]{results/celegans/out5}
  \caption{\acs{C. elegans} analog responses with their digital counterparts}
  \label{graph:celegansAnalog0}
\end{figure}

\Cref{graph:celegansAnalog0} combines the digital and analog of all four output neurons. However the analog predictions are quite unstable. The origin of this issue being unknown, the output are thus artificially smoothed out using software. The graph with the smoothed out curves is \cref{graph:celegansAnalog1}.

\begin{figure}[h]
  \centering
  \includesvg[width=\columnwidth,pretex=\tiny]{results/celegans/smooth5}
  \caption{\acs{C. elegans} analog responses with their digital counterparts}
  \label{graph:celegansAnalog1}
\end{figure}

The output neurons predictions are quite good, expect for the unstable analog predictions. The results are very promising.

The other sequences of inputs were ran but connot all be shown here. More details on the C . elegans analog predictions are available in the thesis this paper goes with.

\section{Conclusion}\label{sec:conc}

\subsection{Performances}

\subsubsection{\ac{LSTM}}

The circuit's performance has been evaluated in the previous chapter. The results highly depend on the parameters at play. For example the complexity of the dataset used, out of the two used in the thesis, the airline dataset is the simple dataset while the \ac{C. elegans} dataset is the more complicated datset.

The best performance is the with the airline dataset, using the circuit with a serial size of one ($n_s=1$) (\cref{tab:airlineAnalog}). While the results are not quite right, it is assumed that this issue will be elevated once inSitu training is implemented (\cref{subsec:inSitu}).

More complex datsets like \ac{C. elegans}, the issue is getting more present. Indeed, the dataset feeds to the circuit four inputs across a thousand time steps and outputs just as much data. Any inaccuracy is scaled up.

With a simple input sequence, such as sequence 5 (\cref{graph:io5Celegans}), the resulting predictions are quite good and on par with what is expected. The predictions are a bit late and it is not clear what is causing this specific issue. A theory is that the memory cells deteriorate its stored value by a very small amount every time step, thus the stored value is affect a lot after a large amount of time steps.

More complex input sequences produce even more inaccuracies. Indeed, when working with sequence 15 (\cref{graph:io15Celegans}), the inaccuracies spiral up and gives out a barely usable prediction. The curve generated does not fit its digital counterpart anymore.

The results obtained demonstrate that the \ac{LSTM} circuit block can be run with low error when dealing with simple inputs. The circuit is assumed to be ready for a full simulation, meaning also training with the analog circuit.

\subsubsection{\ac{GRU}}

The \ac{GRU} circuit, while being present, is still a work in progress. The outputed predictions are scaled down for a still unknown reason, but show the right output shape. Once those issues are dealt with and have been fixed, the \ac{GRU} circuit will be ready for an analog training as well.

\subsection{Execution time}

The inference time is quite tricky to deal with. While the analog circuit's inference time is set, it gets more complicated when trying to get the digital inference time. The digital inference time depends on the computer it is being run on. The \acp{CPU} used in embedded sytems, the main application for this work, are usually not very fast as they are intended to be low powered. This parts will thus only dicuss of the inference of the analog circuit.

\subsubsection{Airline inference}

The airline problem feeding two inputs to the circuit to get a single output, it will have to be run $144$ times to get all the output data.
The inference time depends on the serial size being used :

\begin{itemize}
  \item $n_s=1$ : The output can be read as a voltage on the output net from $t_0=18\mu s$ to $t_1=22\mu s$.
  \item $n_s=2$ : The output can be read as a voltage on the output net from $t_0=34\mu s$ to $t_1=38\mu s$.
  \item $n_s=4$ : The output can be read as a voltage on the output net from $t_0=66\mu s$ to $t_1=70\mu s$.
\end{itemize}

Those are the times it takes to compute one of the outputs. It takes $144$ times longer to get all of the results.

\subsubsection{\ac{C. elegans} inference}

This sequence is a bit more complicated, for every input sequence with a thousand time steps, there are a thousand output vectors being read. Here, the inference time also depends on the serial size.

\begin{table}[H]
  \centering
  \begin{tabular}{|c|c|c|c|c|}
    \hline
    \rowcolor{gray}
    Serial size & $n_s=1$ & $n_s=2$ & $n_s=4$ & $n_s=8$\\
    \hline
    Start of first output & $9\mu s$ & $17\mu s$ & $33\mu s$ & $65\mu s$\\
    \hline
    End of first output & $13\mu s$ & $21\mu s$ & $37\mu s$ & $69\mu s$\\
    \hline
    Output period & $9\mu s$ & $17\mu s$ & $33\mu s$ & $65\mu s$\\
    \hline
    Start of last output & $9ms$ & $17ms$ & $33ms$ & $65ms$\\
    \hline
    End of last output & $9ms +4\mu s$ & $17ms+4\mu s$ & $33ms+4\mu s$ & $65ms+4\mu s$\\
    \hline
  \end{tabular}
  \caption{First and last read times for the different serial sizes}
  \label{tab:readTimesCelegans}
\end{table}

\Cref{tab:readTimesCelegans} shows the first and last output time range with the period. All of the output times can be determined using the first time range and using the period. A sequence consists of $500ms$ as explained in \cref{subsec:celegans}, the time steps are physically $500\mu s$ appart, in the case where the circuit would be used to do real time inference, the circuit would have to be slowed down to be on sync with the inputs. The circuit could could be used in a real time use for this time sensitive problem.

\subsection{onChip area}\label{subsec:area}

The \ac{LSTM} and \ac{GRU} blocks's onChip area is impossible to estimate, as part of the circuit is made of verilog-A models. However, to get a general idea of the area of the chip, it can be assumed that the area of the circuit mainly depends on the the area of a memrisor, because the area of a memristor is much greater than the ones of the other components. Since the number of memristors is the number of weights in the circuit, the minimum area of any \ac{NN} can be determined. The area for a memristor was determined using the feature size of the memristors that can be fabricated at \ac{INESC}. The typical memristor that can be fabricated at \ac{INESC} is $3\mu m$, which would make the approximate area for a memristor $A_{memristor}=9\mu m^2=9\cdot 10^{-12} m^2$

\Cref{tab:areas} contains several \acp{NN} models with their number of parameters, a minimum onChip area and an estimated feature size. None of those values have any scientific ground, they are just here to give general idea of the kind of circuit thta could be fabricated. Those areas only considers the area of the memristors, using a two memristor per synapse architecture like the one used in the thesis.

The feature size represents the length of the side of the chip if the chip was manufactured in a square.

The values of \acp{LLM}, such as \ac{LLaMA}-2 \textbf{Meta}'s \ac{LLM} and the two last versions of \acp{GPT} the \ac{LLM} that powers \textit{ChatGPT}, were included to show how the area of the chip scales up. Those models do not use \acp{RNN}, and futhermore, give out ridiculous areas. Fabricating such cicuits in analog is not feasible witht the current technologies.

\begin{table}[H]
  \centering
  \begin{tabular}{|c|c|c|c|}
    \hline
    \rowcolor{gray}
    Model & Parameters & Minimum area & Approximate feature size \\
    \hline
    Airline \ac{GRU} $n_h=4$ & $77$ & $1386\mu m^2$ & $37.2\mu m$\\
    \hline
    Airline \ac{LSTM} $n_h=4$ & $101$ & $1818\mu m^2$ & $42.6\mu m$\\
    \hline
    \ac{C. elegans} \ac{GRU} $n_h=8$ & $348$ & $6264\mu m^2$ & $79.1\mu m$\\
    \hline
    \ac{C. elegans} \ac{LSTM} $n_h=8$ & $452$ & $8136\mu m^2$ & $90.2\mu m$\\
    \hline
    \multirow{3}{*}{\acs{LLaMA}-2} & $7\cdot 10^9$ & $1260 cm^2$ & $35.5 cm$ \\
    \cline{2-4}
    & $13\cdot 10^9$ & $2340 cm^2$ & $48.4 cm$ \\
    \cline{2-4}
    & $70\cdot 10^9$ & $1.26 m^2$ & $1.12 m$ \\
    \hline
    \acs{GPT}-3 & $175\cdot 10^9$ & $3.14 m^2$ & $1.77m$\\
    \hline
    \acs{GPT}-4 & $1\cdot 10^{12}$ & $18.0 m^2$ & $4.24m$\\
    \hline
  \end{tabular}
  \caption{Estimated onChip area for different models of \acp{NN}}
  \label{tab:areas}
\end{table}

Nevertheless, it is interresting to look at those approximation. It shows that the simple models used in this work would take a very small onChip area, even though this a very raw number. It also shows the current limit with state of the art \acp{NN} at the current time.



\bibliographystyle{IEEEtran}
\bibliography{../definitions/biblio}

\end{document}
