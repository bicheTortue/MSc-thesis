\fancychapter{TODO determine name}
\label{ap:a}

\section{Buffer}\label{apsec:buffer}

This is a classic circuit that only takes one \ac{opAmp} wired as like in figure TODO. It is used to forward a voltage without impacting the original voltage.

\section{Inverting amplifier}\label{apsec:invAmp}

This is the circuit that comes after the voltage multiplier (\cref{sec:models}), it is here to amplify the results of the voltage multiplier by 10. Sometimes, its use doubles as an adder making it an inverting summing amplifier (TODO circuit).

\section{Solving weight to resistance}\label{apsec:wei2res}

\begin{equation}\label{eq:wei2res3}
  \begin{cases}
    w=\frac{R_f}{R_+}-\frac{R_f}{2\cdot R_f -R_+}\\
    R_-=2\cdot R_f -R_+
  \end{cases}
\end{equation}


\begin{equation}\label{eq:wei2res4}
  \begin{cases}
    w=\frac{2\cdot R_f^2-2\cdot R_+\cdot R_f}{2\cdot R_f\cdot R_+ -R_+^2}\\
    R_-=2\cdot R_f -R_+
  \end{cases}
\end{equation}

\begin{equation}\label{eq:wei2res5}
  \begin{cases}
    2\cdot R_f\cdot R_+\cdot w -R_+^2\cdot w=2\cdot R_f^2-2\cdot R_+\cdot R_f\\
    R_-=2\cdot R_f -R_+
  \end{cases}
\end{equation}

\begin{equation}\label{eq:wei2res6}
  \begin{cases}
    R_+^2\cdot w - R_+\cdot 2\cdot R_f \cdot(w+1) + 2R_f^2 = 0\\
    R_-=2\cdot R_f -R_+
  \end{cases}
\end{equation}

\begin{equation}\label{eq:wei2res61}
  R_+^2\cdot w - R_+\cdot 2\cdot R_f \cdot(w+1) + 2R_f^2 = 0
\end{equation}

We must now solve \cref{eq:wei2res61}.

\begin{equation}\label{eq:wei2res61}
  \Delta = 4\cdot R_f^2\cdot(w+1)^2-8\cdot w\cdot R_f^2 = 4\cdot R_f^2\cdot(w^2+1) \ge 0
\end{equation}

Giving the two solutions :

\begin{equation}\label{eq:wei2res62}
  R_{+,0}=\frac{R_f\cdot(w+1)-R_f\cdot\sqrt{w^2+1}}{w}
\end{equation}
\begin{equation}\label{eq:wei2res63}
  R_{+,1}=\frac{R_f\cdot(w+1)+R_f\cdot\sqrt{w^2+1}}{w}
\end{equation}

\section{Hard sigmoid and \ac{tanh} functions}\label{apsec:hardFunc}

The hard sigmoid and \ac{tanh} functions are simplifications of the the classic sigmoid and \ac{tanh} used for faster computation. They come with a downside, they are less efficient then their original functions.

They are defined as \cref{eq:hSigmoid} for the hard sigmoid and \cref{eq:hTanh} for the hard \ac{tanh} \cite{hSigmoid, hTanh}.


\begin{equation}\label{eq:hSigmoid}
  \sigma (x) =
  \begin{cases}
    0& x< -2.5\\
    \frac{x}{5}+\frac{1}{2}&  2.5>x>-2.5\\
    1& x> 2.5\\
  \end{cases}
\end{equation}

\begin{equation}\label{eq:hTanh}
  tanh(x) =
  \begin{cases}
    -1& x<-1\\
    x& 1>x>-1\\
    1& x>1\\
  \end{cases}
\end{equation}

Some versions of the hard sigmoid are defined differently from \cref{eq:hSigmoid}, the slope depends on the library used \cite{hSigmoid1}.

\begin{figure}[H]
  \centering
  \includesvg[width=\textwidth]{activation/hardFunc}
  \caption{Hard sigmoid and hard \acs{tanh} with their original function}
  \label{graph:hardFunc}
\end{figure}

A visual representation helps seeing how close those functions are, \cref{graph:hardFunc} displays the two together with the function they are modeled after.

\section{Smoothing the curve}\label{apsec:smoothFunc}

\begin{lstlisting}[language=Python]
Put your code here.
\end{lstlisting}
