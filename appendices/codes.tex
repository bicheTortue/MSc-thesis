\fancychapter{Important code}
\label{ap:codes}
\pythonstyle

\section{Memristor verilog-A model}\label{apsec:memModel}

The code for the memristor model described in this thesis is available just below. The code was extracted from \cite{memCadenceModel}.

\lstinputlisting[language=Verilog, caption=Memristor Verilog-A model]{code/memristor.va}

\section{Smoothing the curve}\label{apsec:smoothFunc}

The code for smoothing the curve takes in a list and the averaging range. The function returns the averaged list.

\lstinputlisting[language=Python, caption=Python function to smooth out curve]{code/smooth.py}

\section{Saving weights to a file}\label{apsec:saveWei}

There are a few functions that show how the weights are saved into a file.
First depending on the type of layer, the weights need to be set in a list in the correct order.
Once those list are correctly set everything is saved in a binary file.

\lstinputlisting[language=Python, caption=Python functions used to save Keras' weights to a file]{code/saveWeight.py}

\section{Weights to resistances}\label{apsec:wei2res}

The function takes in a weight and returns a tuple with the two resistances values. The global variables are set to set values chosen in this work.

\lstinputlisting[language=Python, caption=Python function to convert a weight in two resistances]{code/wei2res.py}
