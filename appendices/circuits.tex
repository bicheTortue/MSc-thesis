\chapter{Circuits}
\label{ap:circuit}
\clearpage

\section{Buffer}\label{apsec:buffer}

This is a classic circuit that only takes one \ac{opAmp} wired as \cref{circt:buffer}. It is used to forward a voltage without interfering the node of the input voltage.

\begin{figure}[H]
  \centering
  \includesvg[width=.6\textwidth]{appendix/buffer}
  \caption{Buffer circuit}
  \label{circt:buffer}
\end{figure}

The input is $V_{in}$ and the output is $V_{out}$.

\section{Linear combination of inputs}\label{apsec:sumInv}

The linear combination of inputs is a very general circuit that is used in a plethera of different situations. A modulable circuit with a variable number of input ($n+1$) in available in \cref{circt:sumInv}.

\begin{figure}[H]
  \centering
  \includesvg[width=.6\textwidth]{appendix/invAmp}
  \caption{Summing inverter circuit}
  \label{circt:sumInv}
\end{figure}

This circuit links the inputs and its output with \cref{eq:sumInv}.

\begin{equation}\label{eq:sumInv}
  V_{out}=V_{cm}-R\cdot\sum_{i=0}^{n-1}\frac{V_i-V_{cm}}{R_i}
\end{equation}

In this work itself it is used in at least three different use cases :

\begin{itemize}
  \item As a voltage mirror around $V_{cm}$, using a single input ($n=1$) and using $R_0=R$. This is used in \cref{sec:af}.
  \item As a inverting amplifier after a voltage multiplier (\cref{subsec:voltmult}). Using one input ($n=1$) and $R>R_0$. It can be found in \cref{sec:lstmCircuit,sec:gruCircuit}.
  \item It is also used as a summing inverter with amplification after a voltage multiplier. Using two inputs ($n=2$) and $R_0=R_1<R$ for the resistances. This is also found in the circuit in \cref{sec:lstmCircuit,sec:gruCircuit}.
\end{itemize}
