\section{\acs{C. elegans} dataset}
\label{sec:resCelegans}

Due do the complexity of this dataset and the results associated are only trained using the analog activation function. Furthermore, the usefulness of using those activation functions was already demonstrated in \cref{sec:resAirline}.

\subsection{Network configuration}

The layers used to solve this problem are listed below :

\begin{itemize}
  \item An \ac{LSTM} with eight hidden states and an input with feature size of four (because of the four input neurons that are considered), one thousand time steps and every time steps outputs a value for the next layer.
  \item A time distributed dense layer with an output size of four. It is time distributed because it needs to compute every time steps and not only the last.
\end{itemize}

The \ac{LSTM} was chosen with a number of hidden states of eight because it gets the best results \cite{celegans}.

\Cref{fig:celegansModel} is a graphical representation of the model just described.

\begin{figure}[H]
  \centering
  \includesvg[width=\textwidth]{datasets/celegansModel}
  \caption{Model used to solve the \ac{C. elegans} problem}
  \label{fig:celegansModel}
\end{figure}

The \ac{NN} was trained for $1000$ epochs.

\subsubsection{Timing}

The timing of the flags are quite hard to keep track of, for this reason all of them are going to be shown for the full duration of the system execution to get one output.

\Cref{tim:airline} shows the time diagram of the entire execution using the parameters discussed earlier.

\begin{figure}[H]
  \centering
  \begin{tikztimingtable}
    $e_0$       & x 4H    L 4H    N(A1) 3L N(A2) 4H    x\\
    $e_{next}$  & x 4L    H 4L    N(B1) 3H N(B2) 4L    x\\
    $e_{in}$    & x 4H    L 4H    N(C1) 3L N(C2) 4H    x\\
    $e_{out}$   & x 2H 2L L 2H 2L N(D1) 3L N(D2) 2H 2L x\\
    \extracode
    \node[gap, at={($(A1|-A2)!0.5!(A2)$)}];
    \node[gap, at={($(B1|-B2)!0.5!(B2)$)}];
    \node[gap, at={($(C1|-C2)!0.5!(C2)$)}];
    \node[gap, at={($(D1|-D2)!0.5!(D2)$)}];
    \tablerules
    \begin{pgfonlayer}{background}
      \vertlines[help lines]{0.55,4.55,5.55,9.55,12.55,16.55}
      %\vertlines[red]{1.6,5.6,15.6}
      %\vertlines[blue]{3.6,9.6,15.6}
    \end{pgfonlayer}
  \end{tikztimingtable}
  \caption{Flags time diagram for the celegans problem with $n_s=1$}
  \label{tim:airline}
\end{figure}

\subsection{Digital results}
\label{subsec:digitalCelegans}

The results will be analyzed by focusing on two data sequence, the first being simple inputs expecting simple outputs and the second being more complex, those sequences are the ones in \cref{graph:io5Celegans,graph:io15Celegans}. We will identify those as sequence 5 and sequence 15 respectively.

The digital results of the \ac{C. elegans} dataset are quite numerous, not all of them can be displayed and analyzed. For this reason, only the two previously selected sequences will be displayed as the results of the dataset are not the focus of this thesis. The full work concerning the \ac{C. elegans} dataset and problem is done in \cite{celegans}.

\begin{figure}[H]
  \centering
  \begin{minipage}{\columnwidth}
    \subfloat[Analog prediction for the DB1 neuron\label{graph:digital5BD1Celegans}]{\includesvg[width=0.5\textwidth, pretex=\tiny]{results/celegans/digital5DB1}}%
    \hfill
    \subfloat[Analog prediction for the LUAL neuron\label{graph:digital5LUALCelegans}]{\includesvg[width=0.5\textwidth,pretex=\tiny]{results/celegans/digital5LUAL}}%
  \end{minipage}
  \begin{minipage}{\columnwidth}
    \subfloat[Analog prediction for the PVR neuron\label{graph:digital5PVRCelegans}]{\includesvg[width=0.5\textwidth, pretex=\tiny]{results/celegans/digital5PVR}}%
    \hfill
    \subfloat[Analog prediction for the VB1 neuron\label{graph:digital5VB1Celegans}]{\includesvg[width=0.5\textwidth,pretex=\tiny]{results/celegans/digital5VB1}}%
  \end{minipage}
  \caption{\ac{C. elegans} digital responses spread out for each output neuron}
  \label{graph:digital5Celegans}
\end{figure}

\Cref{graph:digital5Celegans,graph:digital15Celegans} are the digital predictions of the selected dat sequences. They all show a decent results, their \ac{RMSE} can be found at \cref{tab:celegansDigital}.

\begin{table}[H]
  \centering
  \begin{tabular}{|c|c|c|c|c|}
    \hline
    \rowcolor{gray}
    Neuron & BD1 & LUAL & PVR & VB1\\
    \hline
    Sequence 5 & $5.02\cdot 10^{-2}$ & $1.68\cdot 10^{-1}$ & $1.32\cdot 10^{-1}$ & $5.07\cdot 10^{-2}$\\
    \hline
    Sequence 15 & $6.98\cdot 10^{-2}$ & $1.39\cdot 10^{-1}$ & $1.86\cdot 10^{-1}$ & $6.63\cdot 10^{-2}$\\
    \hline
  \end{tabular}
  \caption{\acp{RMSE} of each neuron's prediction for the two selected sequences}
  \label{tab:celegansDigital}
\end{table}

A pattern can be observed here, the error rates are lower for the DB1 and VB1 neurons.

\begin{figure}[H]
  \centering
  \begin{minipage}{\columnwidth}
    \subfloat[Analog prediction for the DB1 neuron\label{graph:digital15BD1Celegans}]{\includesvg[width=0.5\textwidth, pretex=\tiny]{results/celegans/digital15DB1}}%
    \hfill
    \subfloat[Analog prediction for the LUAL neuron\label{graph:digital15LUALCelegans}]{\includesvg[width=0.5\textwidth,pretex=\tiny]{results/celegans/digital15LUAL}}%
  \end{minipage}
  \begin{minipage}{\columnwidth}
    \subfloat[Analog prediction for the PVR neuron\label{graph:digital15PVRCelegans}]{\includesvg[width=0.5\textwidth, pretex=\tiny]{results/celegans/digital15PVR}}%
    \hfill
    \subfloat[Analog prediction for the VB1 neuron\label{graph:digital15VB1Celegans}]{\includesvg[width=0.5\textwidth,pretex=\tiny]{results/celegans/digital15VB1}}%
  \end{minipage}
  \caption{\ac{C. elegans} digital responses spread out for each output neuron}
  \label{graph:digital15Celegans}
\end{figure}

\subsection{Analog results of sequence 5}
\label{subsec:analog5Celegans}

It has been established in \cref{sec:resAirline}, that the analog circuit with a serial size of one ($n_s=1$) shows a lower \ac{RMSE} to the digital predictions than the digital predictions.

Due to the complexity of the dataset, the simulation were only ran using a serial size of one ($n_s=1$) to get the best looking results.

This part will show the result of the inference when the weights were trained using the analog activation functions.

\begin{figure}[H]
  \centering
  \includesvg[width=\linewidth]{results/celegans/out5}
  \caption{\acs{C. elegans} analog responses with their digital counterparts}
  \label{graph:celegansAnalog0}
\end{figure}

\Cref{graph:celegansAnalog0} contains the predictions of both the digital and analog \ac{NN} for all four output neurons that are being monitored. \Cref{graph:celegansAnalog0} having all predicted responses to the stimuli on the same graph makes things very unintelligeable.

The response have thus been sperated into four graphs. One for every neuron for better readability. The target response was also added to all four graphs to highlight the objective of the \ac{NN}. Those graphs can be found in \cref{graph:spread5Celegans}.

All curves seem to all have a similar issue, the signal is late to go up. However, the reason for this behavior is still unknown. This can be adjusted manually in the graph until the reason for this issue is found.

\begin{figure}[H]
  \centering
  \begin{minipage}{\columnwidth}
    \subfloat[Analog prediction for the DB1 neuron\label{graph:out5BD1Celegans}]{\includesvg[width=0.5\textwidth, pretex=\tiny]{results/celegans/out5DB1}}%
    \hfill
    \subfloat[Analog prediction for the LUAL neuron\label{graph:out5LUALCelegans}]{\includesvg[width=0.5\textwidth,pretex=\tiny]{results/celegans/out5LUAL}}%
  \end{minipage}
  \begin{minipage}{\columnwidth}
    \subfloat[Analog prediction for the PVR neuron\label{graph:out5PVRCelegans}]{\includesvg[width=0.5\textwidth, pretex=\tiny]{results/celegans/out5PVR}}%
    \hfill
    \subfloat[Analog prediction for the VB1 neuron\label{graph:out5VB1Celegans}]{\includesvg[width=0.5\textwidth,pretex=\tiny]{results/celegans/out5VB1}}%
  \end{minipage}
  \caption{\ac{C. elegans} analog responses with their digital counterparts spread out for each output neuron}
  \label{graph:spread5Celegans}
\end{figure}

All the analog predicted responses seem to have the same problem. When the response is flat, the signal looks very good, while with a small vertical displacement. That displacement could be attributed to the difference of computations from the digital training to the analog inference (\cref{sec:resAirline}). However, when the analog response sends out a pulse, the said pulse is very unstable. It continuously flickers up and down. This can be observed by zooming on the graph like in \cref{graph:zoom5Celegans}.

This phenomenen is very strange and has not been identified so far. The main theory is that the data is not stored properly in the memory cells, thus at every \ac{LSTM} time step the supposed output is slightly altered. This would create this flickering of the outputed data.

The data is not useless for that matter. It was decided to smooth the data by averaging every output point with its neighboring points. This simple average will smooth out the curves to get a decently readable result, that can then be compared with the digitally predicted response. The python code for this function is available in \cref{apsec:smoothFunc}.

\begin{figure}[H]
  \centering
  \includesvg[width=\linewidth]{results/celegans/zoom5}
  \caption{\acs{C. elegans} DB1 analog response with zoom on the pulses}
  \label{graph:zoom5Celegans}
\end{figure}

The adjusted (smoothed out and slided to the left to match the digital prediction best) and exploitable analog predictions are available in \cref{graph:smooth5Celegans}. Theses graphs allow for a much easier and better visualisation of the results.

\begin{figure}[H]
  \centering
  \includesvg[width=\linewidth]{results/celegans/rmse5}
  \caption{\acp{RMSE} of each neuron's analog prediction and adjusted analog prediction for sequence 5}
  \label{bar:rmse5}
\end{figure}

The \cref{bar:rmse5} shows that adjusted or not, the errors are in the expected range. Adjusting the results slightly reduces the error, mainly due to the sliding of the results. Smoothing out the curve does not affect the error as it simply does a local average of each value. The erros are, once again higher when obtained with the analog simulation than with the digital inference.

\begin{figure}[H]
  \centering
  \begin{minipage}{\columnwidth}
    \subfloat[Averaged analog prediction for the DB1 neuron\label{graph:smooth5BD1Celegans}]{\includesvg[width=0.5\textwidth, pretex=\tiny]{results/celegans/smooth5DB1}}%
    \hfill
    \subfloat[Averaged analog prediction for the LUAL neuron\label{graph:smooth5LUALCelegans}]{\includesvg[width=0.5\textwidth,pretex=\tiny]{results/celegans/smooth5LUAL}}%
  \end{minipage}
  \begin{minipage}{\columnwidth}
    \subfloat[Averaged analog prediction for the PVR neuron\label{graph:smooth5PVRCelegans}]{\includesvg[width=0.5\textwidth, pretex=\tiny]{results/celegans/smooth5PVR}}%
    \hfill
    \subfloat[Averaged analog prediction for the VB1 neuron\label{graph:smooth5VB1Celegans}]{\includesvg[width=0.5\textwidth,pretex=\tiny]{results/celegans/smooth5VB1}}%
  \end{minipage}
  \caption{\ac{C. elegans} smoothed out and slided analog responses averaged with $21$ neighbors, spread out for each output neuron}
  \label{graph:smooth5Celegans}
\end{figure}

\subsection{Analog results of sequence 15}
\label{subsec:analog15Celegans}

\begin{figure}[H]
  \centering
  \includesvg[width=\linewidth]{results/celegans/out15}
  \caption{\acs{C. elegans} analog responses with their digital counterparts}
  \label{graph:celegansAnalog1}
\end{figure}

\Cref{graph:celegansAnalog1} contains the predictions of both the digital and analog \ac{NN} for all four output neurons that are being monitored. For the reasons mentioned in the previous section, for the sake of readability, the predictions are spread out on separate graphs.

The response have thus been sperated into four graphs. One for every neuron for better readability. The target response was also added to all four graphs to highlight the objective of the \ac{NN}. Those graphs can be found in \cref{graph:spread15Celegans}.

All curves seem to all have a similar issue, the signal is late to go up. However, the reason for this behavior is still unknown. This can be adjusted manually in the graph until the reason for this issue is found.

\begin{figure}[H]
  \centering
  \begin{minipage}{\columnwidth}
    \subfloat[Analog prediction for the DB1 neuron\label{graph:out15BD1Celegans}]{\includesvg[width=0.5\textwidth, pretex=\tiny]{results/celegans/out15DB1}}%
    \hfill
    \subfloat[Analog prediction for the LUAL neuron\label{graph:out15LUALCelegans}]{\includesvg[width=0.5\textwidth,pretex=\tiny]{results/celegans/out15LUAL}}%
  \end{minipage}
  \begin{minipage}{\columnwidth}
    \subfloat[Analog prediction for the PVR neuron\label{graph:out15PVRCelegans}]{\includesvg[width=0.5\textwidth, pretex=\tiny]{results/celegans/out15PVR}}%
    \hfill
    \subfloat[Analog prediction for the VB1 neuron\label{graph:out15VB1Celegans}]{\includesvg[width=0.5\textwidth,pretex=\tiny]{results/celegans/out15VB1}}%
  \end{minipage}
  \caption{\ac{C. elegans} analog responses with their digital counterparts spread out for each output neuron}
  \label{graph:spread15Celegans}
\end{figure}

\begin{figure}[H]
  \centering
  \includesvg[width=\linewidth]{results/celegans/rmse15}
  \caption{\acp{RMSE} of each neuron's analog prediction and adjusted analog prediction for sequence 15}
  \label{bar:rmse15}
\end{figure}

The \cref{bar:rmse15} shows that adjusted or not, the errors are in the expected range. The PVR output neuron seems to be hardest to reproduce during this sequence as its error is much greater than for the others.

\begin{figure}[H]
  \centering
  \begin{minipage}{\columnwidth}
    \subfloat[Averaged analog prediction for the DB1 neuron\label{graph:smooth15BD1Celegans}]{\includesvg[width=0.5\textwidth, pretex=\tiny]{results/celegans/smooth15DB1}}%
    \hfill
    \subfloat[Averaged analog prediction for the LUAL neuron\label{graph:smooth15LUALCelegans}]{\includesvg[width=0.5\textwidth,pretex=\tiny]{results/celegans/smooth15LUAL}}%
  \end{minipage}
  \begin{minipage}{\columnwidth}
    \subfloat[Averaged analog prediction for the PVR neuron\label{graph:smooth15PVRCelegans}]{\includesvg[width=0.5\textwidth, pretex=\tiny]{results/celegans/smooth15PVR}}%
    \hfill
    \subfloat[Averaged analog prediction for the VB1 neuron\label{graph:smooth15VB1Celegans}]{\includesvg[width=0.5\textwidth,pretex=\tiny]{results/celegans/smooth15VB1}}%
  \end{minipage}
  \caption{\ac{C. elegans} smoothed out and slided analog responses averaged with $21$ neighbors, spread out for each output neuron}
  \label{graph:smooth15Celegans}
\end{figure}

The results are a bit worse for this second sequence. That was expected as the inputs and the expected outputs are far more complex.
