\documentclass[14pt]{beamer}
\usetheme{defense}
\graphicspath{{../../figures/}}


\usepackage{tikz}
\usepackage{svg}
\usepackage{anyfontsize}
\setcounter{tocdepth}{1}

\title{Memristors-based recurent modules for neural computing}

\subtitle{}

\newcommand {\Supervisor} {
  {Dr. Diogo Caetano} and {Dr. Ruxandra Barbulescu}}
\newcommand {\CommitteeMembers} {
  {Prof./Dr. Lorem Ipsum}}
\newcommand {\Chairperson} {{Prof./Dr. Lorem Ipsum}}
\author[V. BARBAZA]{{Valentin BARBAZA}}

\date{17 november 2023}

\logo{
  \begin{tikzpicture}[overlay,remember picture]
    \node[left=0cm] at (current page.33){
      \includegraphics[height=1cm]{logos/inesc-mn.png}
      \hspace{1pt}
      \includegraphics[height=1cm]{logos/ist.eps}
      \includegraphics[height=1cm]{logos/inesc-id.eps}
    };
  \end{tikzpicture}
  }

  \begin{document}

  \frame{\titlepage}

  \begin{frame}
    \frametitle{Table of Contents}
    \tableofcontents
  \end{frame}

  \section{Motivation and objectives}

  \begin{frame}{\insertsection}
    \begin{columns}
      \column{.6\textwidth}
      \begin{itemize}
        \item NNs complexity grows rapidly
        \item So does its computation time
      \end{itemize}
      \column{.4\textwidth}
      \includesvg[width=\columnwidth,pretex=\scriptsize]{NN_explained.svg}
    \end{columns}
  \end{frame}

  \begin{frame}{\insertsection}
    \begin{columns}
      \column{.6\textwidth}
      \begin{itemize}
        \item Reduce computation by using analog
        \item Allows use for time sensitive problems
      \end{itemize}
      \column{.4\textwidth}
      \includesvg[width=\columnwidth,pretex=\scriptsize]{NN_explained.svg}
    \end{columns}
  \end{frame}

  \section{Introduction}

  \begin{frame}{\insertsection}{RNNs}
    RNNs are a type of NN :
    \begin{itemize}
      \item Used for time dependent data
      \item That have a feedback connection
    \end{itemize}

    \begin{equation}\label{eq:rnn}
      \overrightarrow{h_t}=f(\overrightarrow{x_t},\overrightarrow{h_{t-1}})
    \end{equation}
  \end{frame}

  \begin{frame}{\insertsection}{LSTMs}
    \includesvg[width=\columnwidth,pretex=\scriptsize]{lstm/lstmCell}
  \end{frame}

  \begin{frame}{\insertsection}{GRUs}
    \includesvg[width=\columnwidth,pretex=\scriptsize]{gru/encoderCell}
  \end{frame}

  \begin{frame}{\insertsection}{Memristors}
    \begin{columns}
      \column{.65\textwidth}
      \includesvg[width=\columnwidth,pretex=\tiny]{memristor/memristor}
      \column{.4\textwidth}
      \begin{itemize}
        \item Recent electical component
        \item Variable resistance
      \end{itemize}
    \end{columns}
  \end{frame}


  \section{The circuits}
  \subsection{Activation functions}
  \begin{frame}{\insertsection}{\insertsubsection}
    \begin{columns}
      \column{.7\textwidth}
      \includesvg[width=\columnwidth,pretex=\scriptsize]{activation/afCircuit}
      \column{.4\textwidth}
      \begin{itemize}
        \item Creates analog activation functions
        \item Is used for both tanh and sigmoid activation functions
      \end{itemize}
    \end{columns}
  \end{frame}

  \begin{frame}{\insertsection}{\insertsubsection}
    \begin{center}
      \includesvg[width=0.75\columnwidth,pretex=\tiny]{activation/afGraph}
    \end{center}
  \end{frame}

  \subsection{Memory cell}

  \begin{frame}{\insertsection}{\insertsubsection}
    \begin{columns}
      \column{.65\textwidth}
      \includesvg[width=\columnwidth,pretex=\tiny]{memcell/memCircuit}
      \column{.4\textwidth}
      \begin{itemize}
        \item Stores analog value
        \item Has two CMOS switches to avoid leakage
      \end{itemize}
    \end{columns}
  \end{frame}

  \begin{frame}{\insertsection}{\insertsubsection}
    \begin{center}
      \includesvg[width=.75\columnwidth,pretex=\tiny]{memcell/data-loss}
    \end{center}
  \end{frame}

  \subsection{Crossbar circuit}

  \begin{frame}{\insertsection}{\insertsubsection}
    \begin{columns}
      \column{.4\textwidth}
      \begin{itemize}
        \item Performs analog VMM
        \item Can be parallel or serialized
      \end{itemize}
      \column{.65\textwidth}
      \includesvg[width=\columnwidth, pretex=\tiny]{crossbar/crossbarUse}
    \end{columns}
  \end{frame}

  \begin{frame}{\insertsection}{\insertsubsection}
    \includesvg[width=\columnwidth, pretex=\tiny]{crossbar/doubleMem}
    \begin{equation}
      \label{eq:doubleMem2}
      y_0=R_f\cdot\sum_{k=0}^n(\sigma_{k+}-\sigma_{k-})\cdot x_k
    \end{equation}
  \end{frame}

  \subsection{LSTM circuit}

  \begin{frame}{\insertsection}{\insertsubsection}
    \begin{center}
      \includesvg[width=1\columnwidth,pretex=\fontsize{4}{5}\selectfont]{lstm/lstmCircuit}
    \end{center}
  \end{frame}


  \section{The tools}
  \subsection{Weight generation}
  \begin{frame}{\insertsection}{\insertsubsection}
    The weights are trained :
    \hfill\includegraphics[width = 4cm]{tensorflowKeras.jpg}
    \begin{itemize}
      \item Using Keras framework implemented in tensorflow python module
      \item In software and exported to a file
      \item Using the analog activation functions
    \end{itemize}
  \end{frame}

  \subsection{Netlist generation}
  \begin{frame}{\insertsection}{\insertsubsection}
    I created a script capable of generating modular netlist. It can take in :
    \begin{itemize}
      \item Any number of time steps
      \item Any number of inputs
      \item Any serial size
      \item Dense layers
      \item LSTM layers
      \item GRU layers
    \end{itemize}
  \end{frame}

  \section{The datasets}
  \subsection{Airline}
  \begin{frame}{\insertsection}{\insertsubsection}
    The airline dataset contains the monthly number of passengers on international airlines from 1949 to 1960.
    \includesvg[width=\columnwidth,pretex=\tiny]{datasets/airline}
  \end{frame}

  \subsection{C. elegans}
  \begin{frame}{\insertsection}{\insertsubsection}
    \begin{columns}
      \column{.35\textwidth}
      This dataset is more specific :
      \begin{itemize}
        \item It has has 4 inputs
        \item Across 1000 time steps
        \item Contains 40 sequences
      \end{itemize}
      \column{.7\textwidth}
      \includesvg[width=.95\columnwidth,pretex=\tiny]{datasets/celegans/ioValid0}
    \end{columns}
  \end{frame}

  \section{Results}
  \subsection{Airline}
  \begin{frame}{\insertsection}{\insertsubsection}
    \begin{columns}
      \column{.68\textwidth}
      \includesvg[width=\columnwidth,pretex=\tiny]{results/airline/analog}
      \column{.37\textwidth}
      \begin{itemize}
        \item Better results for $n_s=1$
        \item Very promising looking
      \end{itemize}
    \end{columns}
  \end{frame}

  \subsection{C. elegans}

  \begin{frame}{\insertsection}{\insertsubsection}
    \begin{center}
      \includesvg[width=0.8\columnwidth,pretex=\tiny]{results/celegans/out5}
    \end{center}
  \end{frame}

  \begin{frame}{\insertsection}{\insertsubsection}
    \begin{center}
      \includesvg[width=\columnwidth,pretex=\tiny]{results/celegans/smooth5}
    \end{center}
  \end{frame}

  \section{Conclusion}

  \begin{frame}{\insertsection}
    \begin{itemize}
      \item Decent results
      \item Very quick results
      \item Instability when dealing with large amount of time steps
    \end{itemize}
  \end{frame}

\end{document}
