\section{Airline dataset}
\label{sec:resAirline}

\subsection{Digital results}
\label{subsec:digitalAirline}

The digital results of the \ac{NN} once trained are very important. They are the results that the analog system will have to reproduce. Furthermore, this part is used to measure the impact of the custom activation function used for training (\cref{sec:af}).

\begin{figure}[H]
  \centering
  \includesvg[width=\linewidth]{datasets/airlineDigital}
  \caption{Graph of the digital predictions for airline dataset. The dotted vertical line shows the limit of the data used for training and the one used for validation.}
  \label{graph:airlineDigital}
\end{figure}

\Cref{graph:airlineDigital} shows the results of the \ac{NN} next to the orinal data. The results are visually very close. The error is quantified using \ac{RMSE} error function. All those curves error to each other are measured and displayed in \cref{tab:airlineDigital}.

\begin{table}[H]
  \centering
  \begin{tabular}{|c|c|c|c|}
    \hline
    %\rowcolor{gray}
    \cellcolor[HTML]{808080}\acs{RMSE} & Original data & Digital output & Custom activation functions\\
    \hline
    Original data &\cellcolor[HTML]{202020} & $45.24$ & $45.94$\\
    \hline
    Digital output & $45.24$ & \cellcolor[HTML]{202020} & $4.04$\\
    \hline
    Custom activation function & $45.94$ & $4.04$ & \cellcolor[HTML]{202020}\\
    \hline
  \end{tabular}
  \caption{\ac{RMSE} errors of each curve to the others}
  \label{tab:airlineDigital}
\end{table}

The error between the digital curves with and without the custom activation functions are visually inexistant. The \ac{RMSE} error between those curves confirms this low error as it is of only $4.04$ thousands of passengers away. Considering the values at play, this error is negligible.

Those two curves are both very close to the target curve. They have a very similar error rate to this target curve, indeed they're \ac{RMSE} error is $45.24$ for the training using the original activation functions, and $45.94$ for

\subsection{Subsection B}
\label{subsec:subbsectiona}

\begin{figure}[H]
  \centering
  \includegraphics[width=0.5\linewidth]{cover}
  \caption[Dummy Figure Caption for List of Figures.]{Dummy Figure Caption.}
  \label{fig:dummyfigure1}
\end{figure}

Remember you can change the reference style. Another dummy citation \cite{site}.
