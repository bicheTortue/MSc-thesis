\section{Airline dataset}
\label{sec:resAirline}

\subsection{Digital results}
\label{subsec:digitalAirline}

The digital results of the \ac{NN} once trained are very important. They are the results that the analog system will have to reproduce. Furthermore, this part is used to measure the impact of the custom activation function used for training (\cref{sec:af}).

\begin{figure}[H]
  \centering
  \includesvg[width=\linewidth]{datasets/airlineDigital}
  \caption{Graph of the digital predictions for airline dataset. The dotted vertical line shows the limit of the data used for training and the one used for validation.}
  \label{graph:airlineDigital}
\end{figure}

\Cref{graph:airlineDigital} shows the results of the \ac{NN} next to the orinal data. The results are visually very close. The error is quantified using \ac{RMSE} function. All those curves error to each other are measured and displayed in \cref{tab:airlineDigital}.

\begin{table}[H]
  \centering
  \begin{tabular}{|c|c|c|c|}
    \hline
    %\rowcolor{gray}
    \cellcolor[HTML]{808080}\acs{RMSE} & Target data & Digital predictions & \specialcell{Digital predictions with\\analog activation function}\\
    \hline
    Target data &\cellcolor[HTML]{202020} & $54.60$ & $52.27$\\
    \hline
    Digital predictions  & $54.60$ & \cellcolor[HTML]{202020} & $78.64$\\
    \hline
    \specialcell{Digital predictions with\\analog activation function} & $52.27$ & $78.64$ & \cellcolor[HTML]{202020}\\
    \hline
  \end{tabular}
  \caption{\acp{RMSE} of each curve to the others}
  \label{tab:airlineDigital}
\end{table}


Both predictions have similar \ac{RMSE} to the target curve. They have a \ac{RMSE} of $54.60$ for the \ac{LSTM} using the orignal activation functions (sigmoid and \ac{tanh} functions) and $52.27$ for the one using the analog activation functions (\cref{sec:af}). The error difference is so low (about two thousands passengers) that the difference may just come from the training and the initial values of the weights. They both approach the target curve pretty well. These error could be lowered by training for more epochs. This is not done because it diverges too much from the focus of the thesis.

The two predictions are very different from each other, the error rate between the two is of $78.64$, which is significantly higher than the \ac{RMSE} of the curves to the target curve.

Computing the \ac{NN} with or without the analog activation functions doesn't seem to affect the error of the predictions, experience has shown that the \ac{LSTM} using the original activation functions seem to predict higher curves than the target and the \ac{LSTM} using the analog activation functions tends to predict curves that are under the target, the reasons for such a difference are unknown. This is simply an observation after a few different runs.

\subsection{Analog results trained without analog activation function}

This subsection shows the importance of using the analog activation functions in the training phase of the \ac{NN}.

\begin{figure}[H]
  \centering
  \includesvg[width=\linewidth]{results/airline/noCustomAF}
  \caption{Analog predictions trained with the original activation functions.}
  \label{graph:airlineAnalogNoC}
\end{figure}

As expected, the analog predictions are very far off the target curve. Which, in this case, isn't the target dataset but the digital prediction ran with the default activation functions.

\begin{figure}[H]
  \centering
  \includesvg[width=\linewidth]{results/airline/noCustomAFZoomed}
  \caption{Analog predictions trained with the original activation functions.}
  \label{graph:airlineAnalogNoCZoomed}
\end{figure}
