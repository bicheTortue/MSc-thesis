\section{Motivation}
\label{sec:int_motivation}

Nowadays, it's impossible not to have heard about \acp{AI}, they are everywhere and everybody talks about them. The most recent example being \textit{ChatGPT} from \textbf{OpenAI} and is getting very popular even among casual computer users. Everybody is using those tools. It's clear that \ac{AI} is becoming more and more important in the current world.

\acp{AI} are very power hungry algorithms which limits its use in embedded systems and power efficient devices.
Furthermore, the time an \ac{AI} takes to compute its output can become quite long. So much so that the most complex ones only run on online and very powerful servers, like the aforementioned \textit{ChatGPT}.

There are several options to reduce execution speed and energy consumption such as running the algorithm on a \ac{GPU} rather than on a \ac{CPU}. Using an \ac{FPGA} or \ac{ASIC} are other ways to improve power consumption and execution time.
The latter is the most restricting (as it's name implies) but has the best results.

Using an \ac{ASIC} allows the use of analog computation. Analog computation offers great advantages compared to digital computers as it's much faster while being very power-efficient.

Making an analog chip with a very low power consumption allows to use chip like that in embedded systems such as video surveillance cameras. A small chip could be installed inside the system that could be able to do gait detection \cite{gaitDS,gaitDig,gait}. The video surveillance camera would then be able to know who is showing up on the camera and if it's not a known person send an alert.

Such a chip could also be used for video stabilization \cite{videoStab}. It would allow to directly store stabilized video, and not have to use online servers like described in \cite{videoStab}. Any action camera would benefit greatly from such an improvement since they're mostly used with a lot of movements.

Another use for \ac{AI} that works great in embedded use is for camera relocalization \cite{videoReloc}. That is useful especially for VR headsets that need to know their location in the room. Having an embedded chip to do all this computation.
