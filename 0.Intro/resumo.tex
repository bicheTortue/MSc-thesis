\begin{resumo}

  Neste trabalho, propomos uma estrutura anal\'{o}gica para m\'{o}dulos recorrentes baseados em memristors para computa\c{c}\~{a}o neurom\'{o}rfica. O sistema \'{e} totalmente anal\'{o}gico e implementa um de circuito \acs{LSTM} funcional bem como um circuito GRU, ainda em desenvolvimento. Ambos os blocos cont\^{e}m memristors para serem usados como pesos em um circuito VMM anal\'{o}gico capaz. Esses blocos permitem a execu\c{c}\~{a}o de c\'{a}lculos necess\'{a}rios para Redes Neurais Recorrentes de dimens\~{a}o arbitr\'{a}ria, e de forma rapida e eficiente, em circuito integrado. Como parte dos blocos \ac{LSTM} e \ac{GRU}, foi projetado um circuito anal\'{o}gico que implementa fun\c{c}\~{o}es de ativa\c{c}\~{a}o. Este circuito espec\'{i}fico \'{e} capaz de reproduzir fun\c{c}\~{o}es semelhantes \`{a}s sigmoid e tanh, com formas e intervalos de valor similares. O trabalho tamb\'{e}m inclui a implementa\c{c}\~{a}o de uma c\'{e}lula de mem\'{o}ria usada para armazenar um valor anal\'{o}gico por um curto per\'{i}odo de tempo. Os c\'{a}lculos do bloco LSTM podem ser executado em s\'{e}rie ou paralelo, com um grau de serializa\c{c}\~{a}o arbitr\'{a}rio. A serializa\c{c}\~{a}o do sistema permite economizar \'{a}rea no chip, a custo do tempo de execu\c{c}\~{a}o. Adicionalmente, at\'{e} \`{a} data de publica\c{c}\~{a}o esta \'{e} a primeira implementa\c{c}\~{a}o anal\'{o}gica do comportamento do \ac{C. elegans} usando o bloco \ac{LSTM} em circuito integrado. Um sistema anal\'{o}gico deste tipo estabelece a base para a implementa\c{c}\~{a}o em tempo real de sistemas nervosos.

\end{resumo}
